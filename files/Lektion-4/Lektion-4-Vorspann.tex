\cleardoublepage
\chapter*{Einleitung zur Lektion}
\addcontentsline{toc}{chapter}{Einleitung zur Lektion}
\markboth{Einleitung zur Lektion}{Einleitung zur Lektion}

In \marginline{diese Lektion und die vorherigen} Lektion~1 % todo Lektion~\ref{sec:Lektion-1}
hatten Sie schon einen ersten Überblick über die Kernprozesse des Softwareengineering erhalten. Ab dieser Lektion werden wir uns mit diesen Kernprozessen detaillierter beschäftigen. Dabei werden Ihnen auch die Vorgehensmodelle aus Lektion~1 % todo Lektion~\ref{sec:Lektion-1} 
regelmäßig wieder begegnen. Ebenso wird Sie und uns der Modell\-begriff aus Lektion~2 % todo Lektion~\ref{sec:Lektion-2}
weiter beschäftigen. Die Modellierung der Realwelt wird über die Domänenmodellierung als Aktivität innerhalb der Anforderungsermittlung wieder Thema sein. Und schließlich haben Sie in Lektion~3 gelernt, dass nicht nur Zusammenhänge zwischen Objekten der Realwelt modelliert werden, sondern dass auch ihr Verhalten und ihr Zusammenspiel Gegenstand der Modellierung ist, und dass dies, je nach Betrachtungswinkel, mit verschiedenen Modellierungssprachen möglich ist. Ab dieser Lektion werden uns die Blickwinkel Struktur und Verhalten zunehmend nicht mehr nur bezogen auf den Modellierungsgegenstand Realwelt sondern vor allem auf den Modellierungsgegenstand (zukünftiges) Softwareprodukt beschäftigen.

In \marginline{Kernprozesse} einem Softwareentwicklungsprozess spezifiziert man die Anforderungen an das zukünftige Softwareprodukt, überlegt sich dessen Struktur, implementiert und testet es. Sofern man agil vorgeht, spezifiziert man zunächst nur einen Teil der Anforderungen, verzichtet vielleicht auf einen dokumentierten Softwareentwurf und verzahnt stattdessen Entwurfs- und Implementierungstätigkeiten, testet -- möglicherweise hat man die Tests schon pa\-ra\-llel zu den Anforderungen aufgeschrieben, noch bevor man mit der Implementierung beginnt -- und startet die nächste Iteration. Die vier \mbox{Schritte} \textbf{spezifizieren, entwerfen, implementieren und testen} findet man in jedem Softwareentwicklungsprozess, unabhängig davon, nach welchem Vorgehensmodell, in welcher Reihenfolge und mit welchen Methoden gearbeitet wird. Diese Lektion beschäftigt sich mit dem Spezifizieren. Die anderen drei Schritte werden in den folgenden drei Lektionen behandelt.

Das \marginline{Requirements Engineering} Thema dieser Lektion ist somit der Umgang mit Anforderungen im Rahmen des Softwareengineering. Der englische Begriff für diesen Themenkomplex ist Requirements Engineering. Analog zum Begriff Softwareengineering soll mit \textit{Requirements Engineering} der \textbf{ingenieurmäßige, systematische Umgang} mit Anforderungen betont werden. Der Prozess des Requirements Engineering ist ein Kernprozess des Softwareengineering und umfasst die Ermittlung und Dokumentation von Anforderungen, die Prüfung und Anpassung von erfassten Anforderungen, aber auch verschiedene Aktivitäten zur Verwaltung von Anforderungen sowie zum Umgang mit Anforderungsänderungen. Einen so umfassenden deutschen Begriff wie das englische Requirements Engineering gibt es nicht. Der sich unter Umständen anbietende Begriff Anforderungsmanagement ist in der deutschen Literatur enger besetzt, indem er nur die Themenbereiche Anforderungsverwaltung und Anforderungs\-änderung adressiert. Wenn man einen deutschsprachigen Begriff verwenden möchte, wählt man meistens die Begriffskombination „Anforderungsermittlung/-analyse“ als Übersetzung für Requirements Engineering. Diesen Weg sind auch wir in den bisherigen Lektionen gegangen. Dabei bleibt aber häufig unklar, ob alle Aspekte des Umgangs mit Anforderungen gemeint sind oder nur Teilprozesse. Daher hat sich der englische Begriff Requirements Engineering auch in der deutschsprachigen Literatur etabliert. Ab dieser Lektion, in der wir uns das erste Mal systematischer mit Anforderungen beschäftigen, werden wir ihn nun ebenfalls verwenden.

\sttpUniversalkasten{Lernziele zu Lektion 4}{Nach dieser Lektion
	\begin{itemize}
		\item können Sie die Begriffe Stakeholder, Vision, Ziele, Produktumfang und Systemkontext erläutern, sie gegeneinander abgrenzen sowie sie in einen Zusammenhang zum Begriff Anforderung setzen,
		\item kennen Sie Quellen und Kategorisierungen von Anforderungen und können erklären, was Nutzeranforderungen, Systemanforderungen, domänenbasierte Anforderungen, funktionale Anforderungen und nichtfunktionale Anforderungen sind,
		\item kennen Sie die Teilprozesse des Requirements Engineering und können ihr Zusammenspiel erläutern,
		\item können Sie über das UML-Anwendungsfalldiagramm Produkt\-anforde\-rungen und Domänenaspekte modellieren.	
	\end{itemize}
}
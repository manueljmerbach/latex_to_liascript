\section{Kommentierte Literatur}
\label{sec:Kap-2.4}

%Bilder/Buchcover/Buchcover_SWEBOK.jpg
\sttpKommLitItem{Bourque/Fairley (Hrsg.)}{2014}{SWEBOK V3.0 – Kapitel 8}{swe14}{}{}
{Kapitel 8 „Software Engineering Process“ des SWEBOK, das Sie schon kennengelernt haben (s. S.~\pageref{sec:Kap-1.4:Bourque}), beschäftigt sich auch mit dem Thema der Vorgehensmodelle. Allerdings werden keine konkreten Vorgehensmodelle vorgestellt, sondern Kategorien von Vorgehensmodellen (sequentiell, inkrementell, iterativ, agil) gebildet und sehr knapp deren wichtigste Unterscheidungsmerkmale thematisiert. SWEBOK nimmt keinerlei Bewertung bezüglich der Überlegenheit oder Unterlegenheit im Vergleich der Vorgehensmodellkategorien vor. Stattdessen wird die Einzigartigkeit von Softwareentwicklungsprojekten und die daraus folgende Notwendigkeit betont, über eine individuelle Auswahl ein passendes Vorgehensmodell einzusetzen. Das Kapitel be\-inhal\-tet zudem Abschnitte zu organisationsbezogenen Softwareentwicklungsprozessen und entsprechenden Reifegradmodellen.}

%Bilder/Buchcover/Buchcover_Sommerville.jpg
\sttpKommLitItem{Sommerville}{2018}{Software Engineering}{som18}{}{}
{Kapitel 2.1 des aus der Einleitung (S.~\pageref{sec:Kap-0.3:Sommerville}) bekannten Lehrbuchs beschäftigt sich mit sequentiellen und in\-kre\-men\-tell-ite\-ra\-ti\-ven Vorgehensmodellen. Das Wasserfallmodell wird ausführlicher dargestellt, für RUP wird auf die Website zum Buch verwiesen. Die agile Softwareentwicklung bekommt in dieser 10. Auflage (engl. Auflage von 2015) ein eigenes Kapitel (Kapitel 3), in dem auch Scrum und XP thematisiert werden. Hervorzuheben sind hier eine gute Kurzzusammenfassung der wichtigsten XP-Praktiken (S. 92 f.) und ein Glossar der Scrum-Begriffe (S. 101) sowie ein eigenes Unterkapitel (Kap. 3.4) zur Skalierung agiler Methoden und der damit verbundenen Probleme. Insgesamt liegt der Schwerpunkt des Kapitels aber bei den heute relevanten agilen Methoden (unabhängig von ihrer Herkunft aus einem bestimmten Vorgehensmodell). Anhand der verschiedenen Auflagen des Lehrbuchs von Ian Sommerville lässt sich der Wandel von relevanten Themen im Softwareengineering beobachten, wie hier am Beispiel der Vorgehensmodelle: In der 9. Auflage des Lehrbuchs von 2011 lag der Schwerpunkt des Kapitels zu agiler Entwicklung noch beim konkreten Vorgehensmodell XP, während die vorhergehende 8. Auflage von 2007 die agile Softwareentwicklung noch gar nicht berücksichtigte, dafür aber ein längeres Unterkapitel zum RUP beinhaltete. Im 2020 erstmals erschienenen anderen Lehrbuch zum Thema Softwareengineering \cite{som20} vom selben Autor erhalten die Methoden der agilen Softwareentwicklung dann nochmal einen höheren Stellenwert.}

%Bilder/Buchcover/zeitung.png
\sttpKommLitItem{Royce}{1970}{Managing the Development of Large Software Systems}{roy70}{}{}
{Der Originalartikel zum ersten Wasserfallmodell [siehe auch Kap.~\ref{sec:Kap-2.2.1.1}]. Kurz und prägnant – zieht man die zahlreichen Abbildungen ab, verbleiben gerade mal \mbox{etwa} viereinhalb Seiten Text – erläutert Royce seine Vorstellung vom strukturierten Ablauf eines Softwareentwicklungsprojekts. Der Stil des Artikels ist sehr handlungsorientiert: viele Aussagen werden in Form von Faustregeln (engl. Howto) formuliert.}

%Bilder/Buchcover/Buchcover_Jacobson_Booch_Rumbaugh.jpg
\sttpKommLitItem{Jacobson/Booch/Rumbaugh}{1999}{The Unified Software Development \\ %%% für Druck
	Process}{jac99}{}{}
{Das Originalbuch zum Unified Process von dessen Erfindern. Der Unified Process verwendet für alle Prozesse als Modellierungssprache die UML, die von denselben Autoren entwickelt wurde. Das Buch stellt dementsprechend bei der Beschreibung der Bestandteile des Unified Process auch die UML vor. Aufbauend auf dem Unified Process sind verschiedene ausdifferenzierte (werkzeugunterstützte) Versionen entstanden, unter anderem der Rational Unified Process [siehe Kap.~\ref{sec:Kap-2.2.2.1} und \cite{kru99}].}

%Bilder/Buchcover/Buchcover_Kruchten.jpg
\sttpKommLitItem{Kruchten}{1999}{Der Rational Unified Process}{kru99}{}{}
{Originalbuch zum Rational Unified Process in deutscher Übersetzung. Das Buch beschreibt detailliert den Rational Unified Process sowie die zugrundeliegenden Best Practices, die der allgemeine Unified Process \cite{jac99} vorgestellt hatte.
}

%{Bilder/Buchcover/Buchcover_Marciniak.jpg}{S. 1804–1806}
\sttpKommLitItem{Alhir}{2002}{Unified Process}{alh02}{}{}
{Überblicksartikel aus einer Enzyklopädie des Softwareengineering zum Rational Unified Process (RUP). Der Artikel stellt den RUP, seine Charakteristika und Besonderheiten auf wenigen Seiten zusammenfassend dar. Er findet damit einen guten Mittelweg zwischen den oft zu verkürzten Darstellungen des RUP in Büchern oder Kapiteln zu Vorgehensmodellen und den vielen, bis ins kleinste Detail gehenden Darstellungen der exklusiv dem RUP gewidmeten Monographien. Der Titel des Artikels ist allerdings irreführend, da er sich nicht mit dem allgemeinen Unified Process, sondern mit der von Rational Software entwickelten Version beschäftigt.}

%Bilder/Buchcover/Buchcover_Beck.jpg
\sttpKommLitItem{Beck}{2003}{Extreme Programming: Das Manifest}{bec03}{}{}
{Die deutsche Übersetzung des Originalbuchs zu Extreme Programming von dessen Erfinder. Das Buch trägt den Untertitel „Das Manifest“ und dementsprechend ist auch die Art der Darstellung. Der Schwerpunkt liegt auf der Beschreibung von Zielen und Philosophie von XP. Die einzelnen Praktiken werden recht allgemein beschrieben oder anhand von einzelnen Anekdoten und Beispielen. Das Buch ist dementsprechend kein Nachschlagewerk, wie genau die Prinzipien von XP in konkreten Projekten angewendet werden können, sondern eher die Beschreibung einer umfassenderen Vision.}

%Bilder/Buchcover/Buchcover_Schwaber_Sutherland.png
\sttpKommLitItem{Schwaber/Sutherland}{2020}{Der Scrum Guide}{sch20}{}{}
{Etwa fünfzehnseitige Darstellung zu Scrums Regeln, Rollen und Artefakten von den Entwicklern von Scrum. Darstellungen zu Scrum findet man heute in zahlreicher Literatur, vielfach sind diese aber uneinheitlich oder sogar widersprüchlich. Da Scrum nur ein Rahmenwerk für agiles Projektmanagement ist und keine eigenen Entwicklungsmethoden (\zb in welcher Form wird getestet?, wie ist das Verhältnis zwischen Programmcode und Dokumentation?) anbietet, wird es in der praktischen Verwendung mit typischerweise agilen Entwicklungsmethoden (\zb Test-First Development) anderer Quellen kombiniert. In der Literatur zu Scrum vermischt sich dadurch häufig die Darstellung der eigentlichen Scrum-Inhalte mit der Darstellung der üblicherweise im Rahmen von Scrum-Projekten verwendeten Methoden. Der Scrum Guide ist das offizielle Scrum-Regelwerk, das bedeutet Regeln, Methoden oder Praktiken, die hier nicht aufgeführt werden, gehören auch nicht zu Scrum. Die Scrum-Autoren sind in dieser Hinsicht etwas puristisch. Sie weisen in einer Schluss\-bemerkung zum Guide zudem daraufhin, dass es sich nicht um eine Scrum-Entwicklung handelt, wenn man nur Teile von Scrum einsetzt. Dies ist, inklusive dem Abändern von Scrum-Regeln, in der Praxis aber durchaus üblich [siehe Kap.~\ref{sec:Kap-2.3}].}

%{Bilder/Buchcover/Buchcover_Gonzalez.jpg}{84-1 bis 84–28}
\sttpKommLitItem{Favaro}{2014}{Agile}{fav14}{}{}
{Dieser Artikel aus dem Softwareengineering-Handbuch \cite{gon14} stellt die Prinzipien des Agilen Manifests, daraus resultierende agile Methoden der Softwareentwicklung sowie konkrete agile Modelle – unter anderem XP und Scrum – vor. Unter einem recht weiten Blickwinkel werden dabei sowohl Vorläufer und Entstehungshintergrund einzelner Methoden, Gründe und Anlässe für ihre Aufnahme in den agilen Methodenkanon als auch die Konsequenzen ihres Einsatzes für die konkrete Ausgestaltung von Softwareentwicklungsprojekten dargestellt. Zudem stellt der Autor vor, welchen Weg – auch in nicht-agile Ansätze – die verschiedenen Methoden in der Softwareentwicklung seit der Veröffentlichung des Agilen Manifests genommen haben. Im zweiten Teil des Artikels gibt der Autor einen zusammenfassenden Überblick über Themen und Ergebnisse aktueller (2014) Forschungsfelder im agilen Umfeld. Der Artikel schließt mit einer umfangreichen Literaturliste und ist als Einstieg für die weitergehende Beschäftigung mit agilen Methoden und Modellen sehr zu empfehlen.}

\sttpKommLitItem{Sommerville}{2020}{Modernes Software-Engineering}{som20}{}{}
{Dieses Lehrbuch steht parallel zu dem regelmäßig aktualisiertem Standardlehrbuch zum Softwareengineering  \cite{som18} des Autors. Es verfolgt einen anderen Ansatz und setzt einen produktbezogenen anstelle eines projektbezogenen Fokus. Insbesondere geht es dabei mehr von Softwareentwicklungen für den Verkauf aus und weniger von einem Kunden, der eine Spezifikation vorgibt, aufgrund derer eine Softwarefirma ihm in einem Entwicklungsprojekt eine maßgeschneiderte Software erstellt. Das Buch beschäftigt sich sehr ausführlich mit agiler Softwareentwicklung und ihren wichtigsten Methoden.}

%{Bilder/Buchcover/Buchcover_Laplante.jpg}{S. 29–46}
\sttpKommLitItem{Ambler}{2011}{Agile Software Development}{amb11}{}{}
{Ein deutlich pro-agil gefärbter – der Autor ist ein überzeugter Vertreter und Entwickler agiler Methoden – aber nichtsdestotrotz sehr informativer Artikel über agile Softwareentwicklung. Der Artikel stellt verschiedene agile Praktiken vor und beschreibt, worin sich agile Ansätze von klassischen Softwareentwicklungsansätzen unterscheiden und worin sich beide Arten ähneln. Zielgruppe sind Projekte, die von anderen Vorgehensmodellen auf agile Vorgehensweisen wechseln möchten. Dafür beschäftigt sich der Artikel auch mit der Skalierung agiler Ansätze, die in der Regel für kleine und mittlere Teamgrößen konzipiert wurden, auf große und sehr große Entwicklungsteams.}

%/Bilder/Buchcover/Buchcover_Cockburn.jpg
\sttpKommLitItem{Cockburn}{2003}{Agile Software-Entwicklung}{coc03}{}{}
{Der Autor gehört zu den Begründern des Agilen Manifests und hat mit den sogenannten Chrystal-Family-Methoden auch selber ein agiles Vorgehensmodell vorgelegt, das in einem Kapitel dieses Buchs thematisiert wird. Der Schwerpunkt des Buchs liegt aber auf einer allgemeineren Beschreibung von Zielen, Philosophie und Methodik agiler Softwareentwicklung, insbesondere in den Bereichen Stellenwert von Individualität im Entwicklungsteam und Kommunikation. Ähnlich wie im Buch von Beck über XP \cite{bec03} geht es hier um die Darstellung einer Vision und weniger um Faktenwissen zu agilen Methoden. Sehr lesenswert ist das Kapitel über das Agile Manifest (S. 281ff), in dem aus einer Beteiligtenperspektive über Gründe und Zufälle für das Entstehen der Werte und Prinzipien des Manifests referiert wird.}

%/Bilder/Buchcover/zeitung.png
\sttpKommLitItem{Kneuper}{2017}{Sixty Years of Software Development Life Cycle Models}{kne17}{}{}
{Der Zeitschriftenartikel beschäftigt sich mit der Geschichte der Vorgehensmodelle seit den 1950er Jahren und stellt dabei auch die Verbindungen zwischen den Modellen (welches Modell ist aus welchem hervorgegangen) dar. Von den hier im Text betrachteten Vorgehensmodellen werden das Wasserfallmodell von Royce und der Rational Unified Process ausführlicher betrachtet. Die agilen Modelle werden nur allgemein behandelt. Dafür berücksichtigt der Artikel auch die frühen Modelle der 1950er und 1960er Jahre und stellt zum Beispiel das Modell von Benington, das wir im Zusammenhang mit dem Wasserfallmodell nur kurz erwähnt haben, detaillierter vor.}

%Bilder/Buchcover/Buchcover_Balzert.jpg
\sttpKommLitItem{Balzert}{2008}{Softwaremanagement}{bal08}{}{}
{Der dritte Band des dreibändigen Lehrbuchs zur Softwaretechnik beschäftigt sich mit Themen des Softwaremanagements. Einen großen Raum nimmt dabei der Abschnitt zu Vorgehensmodellen (unter dem Begriff Prozessmodell), Qualitätsmodellen und Reifegradmodellen ein (Kap. 16-24). Die Kategorisierung der Vorgehensmodelle unterscheidet sich von derjenigen von SWEBOK \cite{swe14}, die uns als Grundlage diente. Die hier im Text behandelten Vorgehensmodelle finden sich aber alle auch bei Balzert – entsprechend seiner Kategorisierung verteilt über verschiedene Kapitel.}

\sttpKommLitItem{Studienreihe Status Quo Agil – 3. Studie}{2017}{}{kom17}{}{}
{\href{https://www.gpm-ipma.de/wissen/studien/status-quo-agile}{https://www.gpm-ipma.de/wissen/studien/status-quo-agile}

Die Studienreihe „Status Quo Agile“ [siehe auch Kap.~\ref{sec:Kap-2.3}], wird vom „BPM-Labor für Business Process Management und Organizational Excellence“ der Hochschule Koblenz in Zusammenarbeit mit Scrum.org, GPM Deutsche Gesellschaft für Projektmanagement e.V. und teilweise weiteren wechselnden Partnern durchgeführt. Die Studienreihe erfragt seit 2012 alle paar Jahre in einer nicht-repräsentativen Studie die Nutzung agiler Methoden in Unternehmen. Schwerpunktmäßig werden Projekte der Softwareentwicklung oder anderer IT-naher Bereiche befragt, es finden sich aber auch Angaben aus IT-fernen Projekten. Etwa dreiviertel der Studienteilnehmer sind deutsche Unternehmen. Die einzelnen Studien der Reihe legen unterschiedliche Schwerpunkte, ein Großteil der Fragen bleibt aber ähnlich (teilweise veränderte Antwortkategorien).}

\sttpKommLitItem{Studienreihe Status Quo Agil – 4. Studie}{2019/20}{}{kom20}{}{}
{\href{https://www.gpm-ipma.de/wissen/studien/status-quo-agile}{https://www.gpm-ipma.de/wissen/studien/status-quo-agile}

Die Studie von 2020 setzt einen inhaltlichen Schwerpunkt auf den Themenbereich Skalierung agiler Ansätze  \cite[95 \psqq]{kom20}.}

%Bilder/Buchcover/zeitung.png
\sttpKommLitItem{Hunt et al.}{2011}{Agile @ 10}{hun11}{}{}
{Anlässlich des 10. Geburtstages des Agilen Manifests 2011 sendeten zehn der siebzehn Urheber ihre Gedanken zu den Fortschritten und Herausforderungen der agilen Softwareentwicklung für einen kurzen Zeitschriftenartikel einer Onlinezeitschrift ein. Die Beiträge der Autoren sind unterschiedlich lang und meist im Stil einer Kolumne gehalten. Sie beschreiben pointiert und sehr unterschiedlich enthusiastisch den Stellenwert der agilen Methoden innerhalb der Softwareentwicklung zehn Jahre nach der Veröffentlichung des Agilen Manifests.}

%Bilder/Buchcover/Buchcover_Boehm_Turner.jpg
\sttpKommLitItem{Boehm/Turner}{2004}{Balancing Agility and Discipline}{boe04}{}{}
{Eine sehr gute Gegenüberstellung zwischen plangesteuerter und agiler Softwareentwicklung bezüglich deren Charakteristika, Stärken und Schwächen, geeigneten Einsatzgebieten etc. [siehe auch Kap.~\ref{sec:Kap-2.3}]. Darüber hinaus werden ausführliche Fallbeispiele aus der Praxis vorgestellt, die die Grenzen von rein plangesteuertem und rein agilem Vorgehen aufzeigen sollen. In diesem Kontext sehr nett zu lesen ist auch die – natürlich entsprechend konstruierte – Fabel von Elefant (plangesteuerte Modelle) und Affe (agile Modelle), mit der die Autoren ihr Buch beginnen und die ihr Credo beinhaltet: Elefant und Affe sind erst dann erfolgreich, wenn sie die Stärken des jeweils anderen anerkennen und einen Weg gefunden haben, ihre Stärken zu verbinden. Die Autoren Barry Boehm (Wasserfallmodell, Spiralmodell) und Richard Turner (Reifegradmodell CMMI) verband man zur Entstehungszeit des Buchs (2004) eher mit plangesteuerter als mit agiler Softwareentwicklung. Die von Pragmatismus statt von Ideologie geleitete Herangehensweise der Autoren zeigt sich auch daran, dass Alistair Cockburn (einer der Autoren des Agilen Manifests) ein sehr positives Vorwort zum Buch verfasst hat.}

%Bilder/Buchcover/Buchcover_Meyer.jpg
\sttpKommLitItem{Meyer}{2014}{Agile! The Good, the Hype and the Ugly}{mey14}{}{}
{Der Autor Bertrand Meyer gehört zu den Großen in der Informatik: er hat wichtige Prinzipien der objektorientierten Softwareentwicklung entworfen, die Programmiersprache Eiffel erfunden und das in der Softwareentwicklung wichtige Konzept „Design by Contract“ entwickelt. Im Umfeld der agilen Softwareentwicklung war er bis zu diesem Buch weniger präsent. Das Buch stellt agile Prinzipien, Rollen und Methoden (unter anderem diejenigen von XP und Scrum) vor und bewertet sie hinsichtlich ihrer Vor- und Nachteile, ihrer Eignung für die praktische Softwareentwicklung und ihres Innovationspotenzials. Der Schwerpunkt des Buchs liegt auf der Bewertung der agilen Ansätze [siehe auch Kap.~\ref{sec:Kap-2.3}]. Die dafür notwendige Beschreibung der Ansätze ist in der Darstellungsart dabei teilweise stringenter und verständlicher als in vieler agiler Originalliteratur.}
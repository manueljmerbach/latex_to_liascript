\subsection{Das Domänenmodell}
\label{sec:Kap-3.2.2}

Wie aus der Darstellung des Modellbegriffs in Abschnitt~\ref{sec:Kap-3.1} bekannt, kann man Modellierung im Softwareengineering in sehr unterschiedlichen Kontexten einsetzen. Für die Vorbereitung der konkreten Implementierung könnte man zum Beispiel ein Modell erstellen, das sehr technisch gehalten ist und genau diejenigen Elemente enthält, die in der ausgewählten Programmiersprache implementiert werden sollen. Am anderen Ende der Skala kann Modellierung auch verwendet werden, um Strukturen und Abläufe der Realwelt zu modellieren, ohne Implementierungsbedürfnisse zu berücksichtigen. Im objektorientierten Softwareengineering nennt man letzteres Modell ein Domänenmodell. 

Der Begriff der \textit{Domäne} \marginline{Domäne}
wird in unterschiedlichen Wissensgebieten äußerst unterschiedlich definiert. Im Softwareengineering ist eine Domäne derjenige Ausschnitt der Realwelt, der (entsprechend abstrahiert) in der zukünftigen Software abgebildet werden soll. Die Domäne umfasst also Objekte, Strukturen, Arbeits-, Geschäfts\-prozesse und sonstige Abläufe, Personengruppen, Interaktionen, Beziehungen etc. eines abgegrenzten Bereichs der Realwelt. Zum Beispiel würde für eine zu erstellende Schulverwaltungssoftware die Domäne Schule, für ein Fußball-Wettportal die Domäne Fußball relevant sein. Anstelle des Begriffs Domäne wird manchmal auch von Fachgebiet, Anwendungsbereich oder fachlichem Problembereich gesprochen. Ein \textit{Domänenmodell}
\marginline{Domänen-\\modell}
(engl. domain model) – in englischsprachiger Literatur teilweise auch semantic (information) model oder conceptual (information) model – beschreibt entsprechend (Abstraktionen der) Objekte, Strukturen und Abläufe der Domäne und zwar in den Begrifflichkeiten der Domäne, sodass das Domänenmodell auch von Personen außerhalb des Softwareentwicklungsteams verstanden werden kann. In der objektorientierten Softwareentwicklung werden zur Erstellung von Domänenmodellen häufig die Diagrammtypen der Unified Modeling Language (UML) verwendet.

In der Regel besteht ein Domänenmodell sowohl aus statischen (Objekte, Beziehungen, Strukturen) als auch aus dynamischen (bestehende bzw. zu gestaltende Pro\-zesse, Abläufe und Interaktionen) Perspektiven, sogenannten \textit{Sichten},
\marginline{Sicht}
auf die \mbox{Domäne.} Dafür werden meist unterschiedliche Arten von Diagrammen sowie informelle textuelle Beschreibungen kombiniert. Bei  Einsatz von UML als Modellierungssprache ist das Kernstück des Domänenmodells das \textit{Domänenklassendiagramm}, das eine statische Sicht auf die Domäne liefert. Darüber hinaus umfasst ein Domänenmodell häufig verschiedene Arten grafischer oder textueller Beschreibungen, zum Beispiel zur Struktur von Organisationseinheiten (statische Sicht), zur Interaktion zwischen Elementen der Domäne (dynamische Sicht) oder zu Geschäftsprozessen der Domäne (dynamische Sicht).

\vspace{2.2mm} %%% für Druck

\minisec{Anforderungsermittlung und Domänenmodell}
\phantomsection
\label{sec:Kap-3.2-2:anforderungsermittlung}

\vspace{1.1mm} %%% für Druck

Mit der Erstellung eines Domänenmodells wird parallel zu der Anforderungsermittlung begonnen. Oft lassen sich Tätigkeiten nicht eindeutig der Anforderungsermittlung oder der Domänenmodellerstellung zuordnen. So können zum Beispiel modellierte Geschäftsprozesse sowohl Auskunft über Arbeitsabläufe der Domäne geben als auch direkte Anforderungen sein, wenn das zu entwickelnde Softwareprodukt genau diese Geschäftsprozesse unterstützen soll. In Softwareentwicklungsprojekten, die nach sequentiell orientierten Vorgehensmodellen arbeiten, wird das Domänenmodell innerhalb der Phase der Anforderungsermittlung und -analyse abgeschlossen, in agilen Softwareentwicklungsprojekten unterliegt das Domänenmodell – sei es implizit oder explizit vorhanden – genau wie die Anforderungen kontinuierlichen Veränderungen während der gesamten Entwicklungszeit des Softwareprodukts. 

Das Domänenmodell wird vom Softwareentwicklungsteam erstellt, unter enger Einbindung von Personen, die sich mit der Domäne auskennen (Domänenexperten).
In der Regel kommen die Domänenexperten aus dem Umfeld des Auftraggebers des Softwareprodukts (\zb Mitarbeiter verschiedener Fachabteilungen). \marginline{Ziel} Hauptzweck eines Domänenmodells ist die Schaffung eines gemeinsamen (Begriffs)Verständnisses von Softwareentwicklungsteam und Kunde über die Domäne. Adressat des Domänenmodells ist zum einen der Kunde. Dieser muss kontrollieren können, ob das entstandene Modell die Domäne adäquat abbildet. Dementsprechend nicht-technisch muss das Domänenmodell gestaltet sein. Aber auch das Entwicklungsteam (bzw. zumindest einzelne Personen des Teams) ist Zielgruppe des Domänenmodells. Dies ist ein Aspekt, der in Softwareentwicklungsprojekten leider oft vernachlässigt wird. Vom Kunden kann nicht erwartet werden, im weiteren Verlauf des Softwareentwicklungsprozesses (technische) Entwurfsentscheidungen der Software oder gar fertigen Programmcode daraufhin zu überprüfen, ob die Gegebenheiten der Domäne berücksichtigt worden sind. Das ist Aufgabe des Entwicklungsteams und dieses muss dafür die Domäne verstanden haben. 

In klassisch organisierten Softwareentwicklungsprojekten gibt es häufig die Rolle \textit{\mbox{Requirement} Engineer}. 
\marginline{Requirement Engineer}
Diese bildet das Bindeglied zwischen Kunde und Entwicklungsteam. Der Requirement Engineer hat die Aufgabe, die Anforderungen des Kunden und in diesem Zusammenhang auch dessen Wissen über die Domäne zu verstehen und sie für die Softwareentwickler im Team so aufzubereiten, dass diese das Softwareprodukt erstellen können, ohne sich in allen Einzelheiten mit der Domäne auseinandergesetzt zu haben. In agilen Projekten, in denen es in der Regel weniger feste Rollenzuordnungen innerhalb des Teams gibt, ist die Verfügbarkeit des notwendigen Domänenwissens bei den Softwareentwicklern schwieriger sicherzustellen. 

Die finale Version des Domänenmodells enthält genau die Aspekte der Realwelt, die eine Relevanz für das zukünftige Softwareprodukt haben. Die Erstellung eines Domänenmodells ist ein (in der Regel sehr kommunikationsintensiver) Prozess. Es erfordert viel Abstimmung zwischen Kunde und Entwicklungsteam und oft mehrere Modell-Zwischen\-ver\-sionen bis ein Domänenmodell fertiggestellt ist. In vielen Softwareentwicklungsprojekten spielt dabei auch eine Rolle, dass zu Beginn zunächst die Systemgrenzen des zukünftigen Softwareprodukts abgestimmt bzw. verhandelt werden müssen und in diesem Zusammenhang auch erstmal noch unklar sein kann, welcher Ausschnitt der Realwelt denn die Domäne bildet.

Zum Schluss sei noch darauf hingewiesen, dass eine Domäne nicht unveränderlich ist. Zum einen können sich Strukturen oder Abläufe (\zb Arbeitsprozesse) während der Erstellungszeit des Softwareprodukts ändern. Zum anderen könnte es gewünscht sein, Arbeitsprozesse etc. einer Domäne zu verändern, sobald das Softwareprodukt zur Verfügung steht, das diese Arbeitsprozesse übernehmen oder unterstützen kann. In diesem Fall ist die Domäne, die im Rahmen der Erstellung des Softwareprodukts modelliert wird, nicht die aktuell existierende, sondern die zukünftige Domäne. Beispiel: In der Beschreibung ihres Zoos (s. Lektion 1, Fallbeispiel Zoo) spricht die Zoodirektorin davon, dass aktuell sowohl Gehege als auch Käfige für die Tiere verwendet würden, dass es zukünftig aber ausschließlich Gehege geben solle. Bei der Domänenmodellierung wäre in diesem Beispiel daher zu klären, inwiefern die Käfige überhaupt berücksichtigt werden sollen.
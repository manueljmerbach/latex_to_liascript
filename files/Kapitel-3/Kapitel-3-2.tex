\section{objektorientierte Modellierung}
\label{sec:Kap-3.2}

Die  Bezeichnung objektorientierte Modellierung ist ein Sammelbegriff für den Einsatz objektorientierter Konzepte in den Software\-engineering-Prozessen Anforde-
\linebreak %%% für Druck
rungsermittlung und -analyse und Entwurf. Diese entstanden verstärkt seit Ende der 1980er und Anfang der 1990er Jahre. Oft findet sich in der Literatur auch die Aufteilung in die Begriffe \textit{objektorientierte Analyse} (engl. object oriented analysis, 
\marginline{OOA, OOD} 
OOA) und \textit{objektorientierter Entwurf} (engl. object oriented design, OOD). OOA und OOD sind dabei selbst wieder Oberbegriffe, unter denen eine Vielzahl konkreter und unterschiedlicher Methoden für das objektorientierte Softwareengineering subsumiert werden. Dementsprechend unterschiedlich werden OOA und OOD in der Literatur teilweise dargestellt. Gemeinsam ist den unterschiedlichen Methoden, dass für die Modellierungszwecke im Rahmen der Anforderungsermittlung und -analyse und derjenigen im Rahmen des Entwurfs dieselbe Notation – überwiegend die UML – verwendet wird. Beachten Sie, dass je nach verwendetem Vorgehensmodell für ein Softwareprojekt die Modellierung einen unterschiedlich großen Anteil einnimmt. So haben zum Beispiel sequentielle und agile Vorgehensmodelle durchaus verschiedene Vorstellungen über Umfang und Ausgestaltung der benötigten Modelle während des Softwareentwicklungsprozesses.

In dieser Lektion betrachten wir aus dem Themenbereich der objektorientierten Modellierung zunächst nur die Modellierung von Realweltzusammenhängen. 


\section{Kommentierte Literatur}
\label{sec:Kap-3.3}

\sttpKommLitItem{Stachowiak}{1973}{Allgemeine Modelltheorie}{sta73}{Bilder/Buchcover/Buchcover_Stachowiak.jpg}{}
{Eine sehr umfassende Auseinandersetzung mit vorher vorhandenen Erkenntnissen zum Bereich des Modells und darauf aufbauend die Entwicklung einer eigenen sehr ausdifferenzierten Modelltheorie. Diese hat sich heutzutage als Standard für wissenschaftliche Modellbegriffe etabliert. Für den Zweck dieses Kapitels benötigen wir nur die Grundlagen von Stachowiaks Modelltheorie, wie sie in Kapitel 2.1 des Buchs dargestellt werden.
}

\sttpKommLitItem{Balzert}{2005}{Lehrbuch der Objektmodellierung}{bal05}{Bilder/Buchcover/Buchcover_Balzert.jpg}{}
{Das Lehrbuch beschäftigt sich mit Methoden der objektorientierten Analyse (OOA) und des objektorientierten Entwurfs (OOD). In diesem Zusammenhang werden mit vielen Beispielen die Notationselemente der UML vorgestellt – jeweils in dem für OOA bzw. OOD benötigten Detaillierungsgrad. Kapitel 1 enthält zudem eine recht ausführliche Beschreibung der Entwicklungsgeschichte der UML.}

\sttpKommLitItem{Kecher/Salvanos/Hoffmann-Elbern}{2018}{UML 2.5}{kec18}{Bilder/Buchcover/Buchcover_Kecher_Salvanos.jpg}{}
{Die offizielle Spezifikation der UML, die frei von der Website der OMG heruntergeladen werden kann (\url{https://www.omg.org/spec/UML/About-UML/}), ist gerade für UML-Anfänger wenig übersichtlich. Sekundärliteratur zur UML ist daher sehr hilfreich. Dieses Buch bietet eine sehr gute und übersichtliche Darstellung der Diagrammarten und Elemente der UML 2.5 sowie eine kurze Übersicht zur Geschichte der UML. Es ist bereits die 6. Auflage des Buchs. Bisher ist mit jeder größeren Änderung der UML eine neue Auflage erschienen. Jede Diagrammart wird in einem eigenen Kapitel behandelt und nach ähnlichem Schema vorgestellt. Dazu gehören die Beschreibung der einzelnen Notationselemente anhand zahlreicher unterschiedlicher Lebensweltbeispiele genauso wie Beispiel-Implementierungen von in UML modellierten Konzepten in Java- und C\#-Programmcode. Zudem findet sich zu Beginn jedes Kapitels eine kurze Übersicht, zu welchen Zwecken das jeweilige Diagramm im Rahmen von Softwareengineering-Prozessen üblicherweise verwendet wird.}

\sttpKommLitItem{Bourque/Fairley (Hrsg.)}{2014}{SWEBOK V3.0 – Kapitel 9}{swe14}{Bilder/Buchcover/Buchcover_SWEBOK.jpg}{}
{Der schon bekannte Software Engineering Body of Knowledge (SWEBOK) behandelt in Kapitel 9 „Software Engineering Models and Methods“ auch das Thema Modelle im Softwareengineering. Die Konzepte Modell und Abstraktion werden beschrieben, verschiedene Modelltypen vorgestellt sowie über den Sinn von Modellierung und über Gütekriterien für Modellierungssprachen referiert. Wie in den anderen SWEBOK-Kapiteln ist der Stil sehr konzeptionell, teilweise abstrakt, gehalten. Als erster Einstieg in das Thema Modellierung ist der Artikel daher eher weniger geeignet, stattdessen aber sehr gut, wenn einem die Begrifflichkeiten des Themas bekannt, die Abgrenzungen zwischen einzelnen Konzepten aber etwas unklar geblieben sind.}

\sttpKommLitItem{Ambler}{2002}{Agile Modeling}{amb02}{Bilder/Buchcover/Buchcover_Ambler_02.jpg}{}
{Der Autor Scott W. Ambler ist ein starker Befürworter agiler Softwareentwicklung. Nichtsdestotrotz legt er großen Wert auf den Einsatz von Modellen im Rahmen von Softwareentwicklungsprojekten. In diesem Buch stellt er seine Methode eines agilen Modellierens vor, mit der agile Prinzipien und Tätigkeiten der Modellierung in Einklang gebracht werden sollen. Auf dieser Methode basiert auch Amblers spätere Veröffentlichung zum objektorientierten Modellieren. \cite{amb04}.}

\sttpKommLitItem{Ambler}{2004}{The Object Primer}{amb04}{Bilder/Buchcover/Buchcover_Ambler_04.jpg}{}
{Das Buch stellt vor, in welcher Weise objektorientierte Modellierungstechniken mit agilen Prinzipien verbunden werden können. In Kapitel 8 des Buches werden
	\linebreak %%% für Druck
	Domänenmodelle thematisiert, Zusammenhänge zwischen Domänenmodell und Anforderungen aufgezeigt und verschiedene Methoden vorgestellt, wie Domänenwissen in agilen Projekten modelliert werden kann. Interessanterweise findet man dort auch Methoden, die nicht objektorientiert sind. Der Autor plädiert dafür, auch für objektorientiertes Softwareengineering zu einem konkreten Projekt passende Methoden zu verbinden, unabhängig davon, ob sie objektorientiert sind oder nicht.}

\sttpKommLitItem{Oestereich/Scheithauer}{2013}{Analyse und Design mit der UML 2.5}{oes13}{Bilder/Buchcover/Buchcover_Oesterreich_Scheithauer.jpg}{}
{Abschnitt 4 (S. 9 ff.) gibt einen kurzen Überblick über die Geschichte der UML mit einer Übersicht (S. 10), welche objektorientierten Methoden und Notationen in die UML eingeflossen sind.}

\sttpKommLitItem{Schäfer}{2010}{Softwareentwicklung}{sch10}{Bilder/Buchcover/Buchcover_Schaefer.jpg}{}
{Kapitel 2 des Lehrbuchs beschäftigt sich auf wenigen Seiten, aber mit hohem Informationsgehalt mit dem Modellbegriff und dem Zusammenhang zwischen Modellen und Systemen.}

\sttpKommLitItem{Hunt et al.}{2011}{Agile @ 10 }{hun11}{Bilder/Buchcover/zeitung.png}{}
{In dem aus Kapitel 2 bekannten  Artikel mit Beiträgen der Urheber des Agilen Manifests zu dessen 10. Geburtstag findet sich auch ein kurzes Statement von Stephen J. Mellor (S. 17 ff.). Dieser beschreibt anekdotenhaft und nett zu lesen die Diskussionen und Irritationen über den Modellbegriff in der agilen Softwareentwicklung zwischen ihm, dessen Arbeiten sich um Modelle drehten, und den anderen Teilnehmern des Treffens, die Modelle eigentlich ablehnten.}

\sttpKommLitItem{Dumke}{2003}{Software Engineering}{dum03}{Bilder/Buchcover/Buchcover_Dumke.jpg}{}
{Das aus Lektion 1 bekannte Kapitel 1.1 des Buches stellt grundlegende Begriffe des Softwareengineering vor. Zu diesen gehören ausführlich dargestellt der Begriff der Methode und knapper gehalten der Begriff der Domäne.}

\clearpage %%% für Druck

\sttpKommLitItem{Ludewig/Lichter}{2023}{Software Engineering}{lud23}{}{}
{Das Lehrbuch zum Softwareengineering enthält auch ein Kapitel über Modellierung (Kap. 1). Es stellt verschiedene Arten von Modellen in der Informatik und im Softwareengineering vor. Der Blickwinkel ist allgemeiner gehalten als wir ihn für diesen Text benötigt haben. Insofern finden Sie dort Aspekte zum Thema, wenn Sie sich allgemeiner mit Modellierung beschäftigen möchten. Näher an den hier im Text gesetzten Schwerpunkten sind \cite{som18} und \cite{sch10}.}
\subsection{Die Unified Modeling Language (UML)}
\label{sec:Kap-3.2.3}

Der zentrale Faktor für die Qualität von Modellen, die im Rahmen des Softwareentwicklungsprozesses erstellt werden, ist ihre Eignung für die anvisierte Zielgruppe und den vorgesehenen Einsatzweck. Das bedeutet, dass die Menschen, die das Modell verstehen sollen, es auch verstehen können müssen und anhand des Modells die Entscheidungen treffen können, die sie treffen sollen. Dies kann aber nur gelingen, wenn die eingesetzte Modellierungssprache zum einen eindeutig genug ist, um Missverständnisse zu vermeiden und zum anderen mächtig genug ist, um alle benötigten Konzepte abbilden zu können. Idealerweise sollte sie zudem für die gegebenenfalls auch aus technisch nicht-versierten Personen bestehende Zielgruppe leicht erlernbar sein. Eine wichtige Aufgabe zu Beginn eines Softwareentwicklungsprojekts ist daher die Verständigung auf die einzusetzende(n) Modellierungssprache(n). Eine in vielen objektorientierten Softwareentwicklungsprojekten eingesetzte Modellierungssprache ist die Unified Modeling Language (UML). Die UML ist eine sehr mächtige Modellierungssprache, die die gängigen Konzepte der Objekt\-orien\-tierung abbilden kann. Gleichzeitig bietet sie sehr unterschiedliche Diagrammtypen an, mit denen sich verschiedene Sichten auf die Domäne oder das bestehende bzw. zu entwickelnde Softwareprodukt modellieren lassen. 

In den späten 1980er und frühen 1990er Jahren waren von unterschiedlichen Autorinnen und Autoren verschiedene Methoden und zugehörige Notationen für objekt\-orientiertes Softwareengineering vorgestellt worden. Zu den bekannteren gehörten die Object Modeling Technique (OMT) von James Rumbaugh  \cite{rum91}, die Booch-Methode von Grady Booch \cite{boo94} und das Object-Oriented Software Engineering (OOSE) von Ivar Jacobsen \cite{jac92}. Mitte der 1990er Jahre führten Booch, Jacobson und Rumbaugh – zu diesem Zeitpunkt mittlerweile alle drei bei Rational Software beschäftigt – ihre Ansätze zusammen, integrierten Notationen weiterer Autorinnen und Autoren und schufen mit der UML eine einheitliche Modellierungssprache für die objektorientierte Softwareentwicklung. 

Die Version 1.0 der UML wurde 1997 veröffentlicht. Die UML verbreitete sich \mbox{rasant}, auch weil sie seit Ende der 1990er Jahre als Standard der Object Management Group  (OMG), in der zahlreiche Unternehmen der Computerindustrie vertreten sind, herausgegeben wird und seitdem unter Führung der OMG kontinuierlich weiterent\-wickelt wird. 

Die UML definiert (in erster Linie grafische) Symbole, mit denen objektorientierte Konzepte (wie \zb Objekt, Klasse, Schnittstelle, Generalisierung, Nachrichten\-versand) dargestellt werden können. Die grafischen Notationselemente sind bewusst sehr einfach gehalten, um UML-Diagramme auch für Nicht-Programmierer verständlich zu gestalten. Vereinfacht gesagt, bestehen UML-Diagramme fast ausschließlich aus Kästchen und Pfeilen zwischen den Kästchen. 

Mit dem Wechsel auf Version 2.0 im Jahr 2005 wurde die UML komplett neu strukturiert. Ein Ziel war es, die UML allgemeingültiger objektorientiert zu halten, indem Notationselemente noch unabhängiger von den Bedürfnissen konkreter objektorientierter Programmiersprachen gestaltet wurden. Außerdem erweiterte die Version 2 die UML um zusätzliche Diagrammarten, vor allem um dynamisches Verhalten eines Softwaresystems noch besser modellieren zu können. Mit der Object Constraint Language 
(OCL) wurde zudem eine formale Beschreibungssprache in die UML integriert, mit der man grafische Notationselemente mit zusätzlichen Zusicherungen und Bedingungen versehen kann. Die aktuelle Version der UML ist heute (2024) die Version 2.5.1, die im Dezember 2017 veröffentlicht wurde.\footnote{Die jeweils aktuelle UML-Spezifikation kann frei von der Website der OMG heruntergeladen werden: \url{https://www.omg.org/spec/UML/About-UML/}}

Die UML bietet sieben sogenannte Strukturdiagramme an, mit denen sich aus einer statischen Sicht Elemente (\zb Module, Klassen, Objekte) und ihre Beziehungen zueinander modellieren lassen. Hinzu kommen sieben sogenannte Verhaltens\-diagramme, mit denen aus einer dynamischen Sicht Prozessabläufe sowie Inter\-aktionen zwischen Elementen modelliert werden können.

Die UML ist eine reine Modellierungs\textbf{notation}, sie ist \textbf{keine Methode} für objekt\-orientiertes Softwareengineering. Das bedeutet, sie stellt verschiedene Diagrammtypen zur Verfügung und gibt vor, welche Notationselemente in welchen Diagrammen und in welchen Kombinationen verwendet werden können. Sie legt aber nicht fest, welche Diagramme, in welcher Detailtiefe und in welcher Reihenfolge für die Durchführung der Prozesse des Softwareengineering erstellt werden sollen. Daher besitzen die verschiedenen UML-Diagrammtypen auch explizit keine feste Zuordnung zu konkreten Tätigkeiten des Softwareengineering, sondern können für verschiedene Software\-engi\-neering-Prozesse verwendet werden. So eignen sich zum Beispiel UML-Klassen\-dia\-gramme in entsprechend angepassten Abstraktionsgraden sowohl für die Domänenmodellierung als auch für die Prozesse der Anforderungsermittlung und -analyse und des Entwurfs sowie für die Prozesse der (Vorbereitung der) Implementierung und des Testens. In dieser
\marginline{Durchgängigkeit der Konzept\-modellierung}
Durchgängigkeit der Konzeptmodellierung über verschiedene Prozesse des Softwareengineering liegt einer der großen Vorteile der UML. UML-Diagramme aus frühen Phasen der Softwareentwicklung können durch Ergänzungen und Verfeinerungen in geeignete Diagramme für spätere Phasen überführt werden. Medienbrüche können dabei weitgehend vermieden werden.

\sttpKasten{\textbf{UML-Werkzeuge}

Parallel zur Entwicklung und Weiterentwicklung der UML entstanden Werkzeuge zur Erstellung von UML-Diagrammen. Grundsätzlich lassen sich UML-Diagramme mit allen gängigen Zeichenprogrammen erstellen, mit denen man verschiedene Arten von Kästen und Pfeilen zeichnen kann. Explizite UML-Zeichenprogramme bieten aber den Vorteil, dass die verschiedenen UML-Notationselemente direkt ausgewählt werden können und dass die Programme bei der Einhaltung der Syntax-Regeln (welche Notationselemente dürfen in welchen Kombinationen verwendet werden) unterstützen. Seit der Version 2 der UML existiert mit dem XML Metadata Interchange-Format (XMI) zudem ein standardisiertes Austauschformat für die Speicherung von UML-Diagrammen. Sofern die UML-Werkzeuge dieses Format unterstützen, kann auf diese Weise ein UML-Diagramm, das mit einem bestimmten UML-Werkzeug erstellt wurde, auch mit einem anderen UML-Werkzeug angezeigt und weiter bearbeitet werden.

Neben den reinen UML-Zeichenprogrammen gibt es UML-Werkzeuge, die auf Grundlage eines im Entwurfsprozess erstellten UML-Klassendiagramms des zukünftigen Softwareprodukts Programmcode einer bestimmten objektorientierten Programmiersprache erzeugen. Diese Automatisierung des Implementierungsprozesses bietet – zumindest theoretisch – den Vorteil, dass Inkonsistenzen zwischen Entwurfsmodell und Programmcode verhindert werden können, indem Änderungen nur am Modell vorgenommen werden und der Programmcode \mbox{daraus} automatisiert erzeugt wird.}

Sie werden verschiedene Diagramme der UML kennenlernen, die in den Prozessen des Softwareengineering eingesetzt werden können. Es geht uns explizit nicht darum, im Rahmen dieses Textes die UML in ihrer Gänze darzustellen. Die Notationselemente der UML interessieren (nur) vor dem Hintergrund ihrer Einsatzmöglichkeiten für Aktivitäten im Softwareengineering und werden daher in Zusammenhang mit den – und begrenzt auf die – in der jeweiligen Lektion gerade behandelten Themen des Softwareengineering vorgestellt. 




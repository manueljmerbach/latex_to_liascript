\section{Testen}
\label{sec:Kap-11-1}

Testen heißt zunächst einmal Ausprobieren, auch wenn im Software Engineering mit den Begriffen "`Test"' und "`testen"' viel mehr verbunden wird. Man testet ein beliebiges technisches Produkt, indem man ausprobiert, ob es das tut was es soll, funktionsfähig bleibt, andere Qualitätsanforderungen erfüllt usw. Umgekehrt ist nicht jedes Ausprobieren gleich ein Test: Man kann auch ein Produkt ausprobieren, um herauszufinden, was es tut oder um jemand anderem das Produkt zu demonstrieren. Beim Testen geht es dagegen grundsätzlich darum, Fehler zu finden oder Fehler\-freiheit wahrscheinlicher zu machen, wobei ein Fehler zunächst einmal ein Abweichen von der Vorstellung ist, wie das getestete System sich in einem bestimmten Fall verhalten sollte.

Beschränken wir uns auf Software. Wie bei dem Vergleich von Vorgehensmodellen in Lektion~1 % TODO Lektion~\ref{sec:Lektion-1}
diskutiert, betraf das Testen ursprünglich die fertige, implementierte Software, die erst relativ spät im Entwicklungsprozess vorliegt. Findet man hier einen Fehler, also ein anderes Verhalten als von der Spezifikation vorgegeben, hat man ein Problem: Liegt der Fehler an der Implementierung oder ist die Spezifikation fehlerhaft? Während ein Implementierungsfehler als Abweichung von der Spezifikation definiert werden kann, ist ein Spezifikationsfehler ein Abweichen von der Vorstellung der Stakeholder. Wie in Lektion~4 % TODO Lektion~\ref{sec:Lektion-4}
diskutiert, gleicht diese "`Vorstellung"' einem Wattebausch, der sich auch noch verändern kann. Aus diesem Grund werden Spezifikationen möglichst früh und immer wieder validiert, erste Prototypen mit den Anforderungen verglichen und in agilen Verfahren möglichst früh Software bereitgestellt, die von Beginn an Tests unterzogen wird, um den tatsächlichen Bedürfnissen der Stakeholder möglichst nahe zu kommen.
Wir wollen von den Überlegungen, wann und was im Rahmen verschiedener Prozesse bzw. Vorgehensmodelle getestet wird, in dieser Lektion weitgehend abstrahieren, und stattdessen Grundsätzliches zum Testen von Software herausstellen.

Es gibt zu Tests, Testverfahren, Testwerkzeugen, usw. etliche Bücher, die aber fast alle sehr praxisorientiert sind oder sich auf eine konkrete Programmiersprache beziehen. Insbesondere geht es vielfach um die Aufgaben und Qualifikationen zertifizierter Software-Tester, siehe zum Beispiel \cite{spi19}. Unsere Perspektive ist etwas anders, mit höherer Flughöhe und mehr an Standards und Prinzipien orientiert. An speziellen Test-Lehrbüchern haben wir uns nicht orientiert, sondern eher an Kapiteln zum Thema Test aus Büchern zum Softwareengineering. Als lesenswerte Beispiele empfehlen wir \cite{bal11}, \cite{lig09}, \cite{zel09} und \cite{lud23}. An vielen Stellen waren wir inspiriert von einem Vorlesungsskript von Martin Glinz \cite{gli05}.
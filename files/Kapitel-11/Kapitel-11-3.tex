\section{Kommentierte Literatur}
\label{sec:Kap-11.3}

\sttpKommLitItem{Ludewig/Lichter}{2023}{Software Engineering}{lud23}{}{}
{In Kapitel~18 "`Programmtest"' werden generelle Konzepte des Testens im Software\-entwicklungs\-prozess angesprochen. Dazu zählen das Einordnen von Tests in den Gesamtlebenszyklus, grundlegende Teststrategien sowie ein Überblick über verschiedene Teststufen. Die Darstellung ist eher breit gefasst und soll ein allgemeines Verständnis vermitteln.}

\sttpKommLitItem{Spillner/Linz}{2019}{Basiswissen Softwaretest}{spi19}{}{}
{Eine ausführliche Betrachtung des Programmtests. Dieses Buch ist als etablierter Standard für die Vorbereitung auf den ISTQB-Certified-Tester bekannt. Ent\-sprechend bietet es einen tieferen Einblick in methodische und praktische Aspekte des Software-Testens. Es finden sich ausführliche Darstellungen von Testfallentwurfsverfahren, dynamischen und statischen Testtechniken, Testdokumentation, Test\-organisation, Rollen und Verantwortlichkeiten im Testprozess sowie ein stärkerer Fokus auf Normen, Standards und das professionelle Testmanagement.}

\sttpKommLitItem{Hoare}{1969}{An Axiomatic Basis for Computer Programming}{hoa69}{}{}
{Hoare legt in seinem grundlegenden Artikel den theoretischen Grundstein für formale Korrektheitsbeweise von Programmen. Historische Grundlagenliteratur.}

\sttpKommLitItem{Huth/Ryan}{2004}{Logic in Computer Science}{hut04}{}{}
{Kapitel~4 "`Program verification"' bietet einen vertieften Einblick in die formale Verifikation von Programmen. Behandelte Themen umfassen unter anderem Hoare-Tripel, partielle und totale Korrektheit sowie grundlegende Beweisregeln und Verifikations\-strategien.}
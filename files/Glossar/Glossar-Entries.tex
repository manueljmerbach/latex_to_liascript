%%%%%%%%%%%%%%%%%%%%%%%%%%%%%%%%%%%%%%%%%%%%%%%%%%%
%%%%% Glossar vorerst auf Kommentar
%%%%%%%%%%%%%%%%%%%%%%%%%%%%%%%%%%%%%%%%%%%%%%%%%%%

% Diese Datei enthält alle Glossar-Einträge für den gesamten Kurs.

% Im Moment müssen Umlaute und das ß noch gesondert ausgezeichnet werden!!!

\newglossaryentry{Anwendungsfalldiagramm}
{
	name=Anwendungsfalldiagramm,
	description={TODO: }
}

\newglossaryentry{Anwendungsprogrammierer}
{
    name=Anwendungsprogrammierer,
    description={TODO: Glossartext f\"ur den Begriff Anwendungsprogrammierer}
}

\newglossaryentry{CASE_Tools}
{
    name=CASE Tools,
    description={TODO: Glossartext f\"ur den Begriff CASE Tools}
}

\newglossaryentry{Domaenenexperte}
{
    name=Dom\"anenexperte,
    description={Eine Person, die das fachliche Anwendungsgebiet (die Dom\"ane), f\"ur das das Softwareprodukt erstellt werden soll, sehr gut kennt und daher beurteilen kann, ob die technische Umsetzung den fachlichen Erfordernissen entspricht.}
}

\newglossaryentry{Kapselung}
{
    name=Kapselung,
    description={TODO: Glossartext f\"ur den Begriff Kapselung}
}

\newglossaryentry{Klassendiagramm}
{
	name=Klassendiagramm,
	description={TODO: }
}

\newglossaryentry{Komponente}
{
	name=Komponente,
	description={TODO: }
}

\newglossaryentry{Komponentendiagramm}
{
	name=Komponentendiagramm,
	description={TODO: }
}

\newglossaryentry{Konsistenz}
{
    name=Konsistenz,
    description={TODO: Glossartext f\"ur den Begriff Kapselung}
}

\newglossaryentry{Kontextdiagramm}
{
	name=Kontextdiagramm,
	description={TODO: kommt aus der Strukturierten Analyse}
}

\newglossaryentry{Library}
{
    name=Library,
    description={TODO: Glossartext f\"ur den Begriff Library}
}

\newglossaryentry{Objektdiagramm}
{
	name=Objektdiagramm,
	description={TODO: }
}

\newglossaryentry{Prototyp}
{
	name=Prototyp,
	description={TODO: Glossartext f\"ur den Begriff Prototyp}
}

\newglossaryentry{Rollen}
{
    name=Rollen,
    description={TODO: Glossartext f\"ur den Begriff Rollen}
}

\newglossaryentry{Schnittstelle}
{
    name=Schnittstelle,
    description={TODO: Glossartext f\"ur den Begriff Schnittstelle}
}

\newglossaryentry{Skalierbarkeit}
{
    name=Skalierbarkeit,
    description={TODO: Glossartext f\"ur den Begriff Skalierbarkeit}
}

\newglossaryentry{Softwarearchitekt}
{
    name=Softwarearchitekt,
    description={TODO: Glossartext f\"ur den Begriff Softwarearchitekt}
}

\newglossaryentry{Softwareprodukt}
{
    name=Softwareprodukt,
    description={Das Ergebnis eines Softwareentwicklungsprozesses.}
}

\newglossaryentry{Softwaretests}
{
    name=Softwaretests,
    description={TODO: Integrationstests überprüfen das Zusammenspiel verschiedener Komponenten.
    	im Systemtest wird geprüft, ob die Software die Gesamtheit der definierten Anforderungen erfüllt}
}

\newglossaryentry{Systemarchitektur}
{
    name=Systemarchitektur,
    description={TODO: Glossartext f\"ur den Begriff Systemarchitektur}
}

\newglossaryentry{Validierung}
{
    name=Validierung,
    description={TODO: Glossartext f\"ur den Begriff Validierung}
}

\newglossaryentry{Verifikation}
{
    name=Verifikation,
    description={TODO: Glossartext f\"ur den Begriff Verifikation}
}

\newglossaryentry{Vision}
{
	name=Produktvision,
	description={TODO: Glossartext f\"ur den Begriff Vision/Produktvision}
}

\newglossaryentry{Wartbarkeit}
{
    name=Wartbarkeit,
    description={Ein Qualit\"atskriterium f\"ur Software. Es gibt an, mit wieviel Aufwand und mit welchem Erfolg \"Anderungen an der bestehenden Software vorgenommen werden k\"onnen.}
}
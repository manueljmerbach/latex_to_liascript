%%-----------------------------------------------------
%%---------- Makros für Lektion 3 (Petrinetze) --------
%%-----------------------------------------------------

% wird für die Darstellung der Petrinetze in Lektion 3 benötigt
\usetikzlibrary{petri, positioning, angles, quotes, decorations.pathreplacing, calc, arrows.meta}

% zentrales Setzen von Farben
\colorlet{colDummyLine}{white} % Dummy-Linie über diesen Schalter sichtbar (rot) bzw. unsichtbar (weiß) machen

\colorlet{COLTRANS}{black}
\definecolor{COLPLACE}{RGB}{0,102,102} % FernUni-Grün
\colorlet{COLRED}{red!80!black}

\newlength{\PNTransWidth}
\newlength{\PNTransHeight}
\newlength{\PNPlaceSize}

\setlength{\PNTransWidth}{2.2cm}
\setlength{\PNTransHeight}{1.5cm}
\setlength{\PNPlaceSize}{0.9cm}

%--------------------------------------
% Makro für Petrinetze
%--------------------------------------
% Das folgende Kommando verwenden wir für die Darstellung der Petrinetze in Lektion 3.
% Voraussetzung hierfür ist die Verwendung des Package "tikz" und einiger Libraries.
%
% \usepackage{tikz}
% \usetikzlibrary{petri, positioning, angles, quotes, decorations.pathreplacing, calc, arrows.meta}
%--------------------------------------
\newcommand{\petrinetz}[1]{
	\begin{tikzpicture}[
		>=Stealth, % Latex-Pfeilspitzen verwenden
		line width=0.9pt, % Dicke der Linien
		every place/.style={minimum size=0.75cm, inner sep=0pt}, % Größe der Stellen, ohne Polsterung
		every transition/.style={minimum size=0.75cm, inner sep=0pt}, % Größe der Transitionen, ohne Polsterung
		every node/.append style={font=\boldmath}, % Fett und größer für alle Labels
		shorten >=0pt,
		shorten <=0pt, % Pfeile verkürzen
		post/.style={->, line width=0.9pt, arrows={-Stealth[length=8pt,width=6pt]}} % Pfeilspitzen größer
		]
		#1 % Zeichnung
	\end{tikzpicture}
}
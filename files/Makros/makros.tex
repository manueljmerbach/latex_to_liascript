%%-----------------------------------------------------
%%---------- Makros für deutsche Abkürzungen ----------
%%-----------------------------------------------------
\newcommand{\dasHeisst}{d.h.\@\xspace}
\newcommand{\sa}{s.a.\@\xspace}
\newcommand{\so}{s.o.\@\xspace}
\newcommand{\su}{s.u.\@\xspace}
\newcommand{\zb}{z.B.\@\xspace}

% Über dieses Makro wird die Kapitelüberschrift für das Fallbeispiel erzeugt.
% Darin enthalten sind auch der Eintrag für das Inhaltsverzeichnis sowie die Angaben für linke und rechte Kopfzeile.
\newcommand{\FallBeispielZoo}{
	\chapter*{Fallbeispiel Zoo ~~~ \raisebox{-0.35cm}{\includegraphics[height=1.0cm]{Bilder/elephant_emoji.png}}}
	\addcontentsline{toc}{chapter}{Fallbeispiel Zoo ~~~ \raisebox{-0.25cm}{\includegraphics[height=0.8cm]{Bilder/elephant_emoji.png}}}
	\markboth{Fallbeispiel Zoo ~ \raisebox{-0.15cm}{\includegraphics[height=0.6cm]{Bilder/elephant_emoji.png}}}{\raisebox{-0.15cm}{\includegraphics[height=0.6cm]{Bilder/elephant_emoji2.png}} ~ Fallbeispiel Zoo}
}

% neues Kommando für farblich hervorgehobenen Text 
% Im Kurstext wird diese Hervorhebung verwendet, wenn der erste Satz eines Absatzes hervorgehoben werden soll. Funktioniert aber auch an jeder beliebigen Stelle.
% Hervorgehobener Text wird in FernUni-MI-green dargestellt.
% Das Kommando erhält 1 Parameter
%   - Text, der farblich hervorgehoben wird
\newcommand{\sttpHervorhebung}[1]{\textcolor{FernUni-MI-green}{#1}}

%% Glossar vorerst auf Kommentar
%% neues Kommando für einen STTP-eigenen Glossareintrag (sttpgls -> gls => Glossar)
%% Der Glossar-Begriff zu dem Glossar-Schlüsselwort wird im Seitenrand angezeigt und mit einem entsprechenden Icon versehen.
%\newcommand{\sttpgls}[1]{\marginline{\includegraphics[height=0.3cm]{Bilder/gluehbirne.png} \gls{#1}}}

% neues Kommando für einen Kapitelverweis
% Ein Kapitelverweis wird im Seitenrand positioniert und sieht wie folgt aus: "`Begriff"' -> "`Kap. 1.1"' (mit Verlinkung).
% Der Kapitelverweis erhält 2 Parameter
%   - Begriff
%   - Verweis (der eigentliche Verweis soll allerdings beim Aufruf gemacht werden)
\newcommand{\sttpkapitelverweis}[2]{\marginline{#1 $\rightarrow$~#2}}

%--------------------------------------
% neues Kommando für einen Eintrag in der 'Kommentierten Literatur' (KommLitItem -> Komm(entierte) Lit(eratur) Item (Eintrag))
% enthält 7 Parameter
%   - Autor(en)
%   - Jahr
%   - Titel
%   - Key für die Literatur-Referenz
%   - #5 wird derzeit nicht verwendet
%     Dateinamen der Grafik für das Buchcover
%     Wenn der Parameter leer bleibt ('{}'), wird kein Buchcover angezeigt.
%   - #6 wird derzeit nicht verwendet
%     Seitenangabe, wenn es nur ein Artikel aus dem Buchcover ist.
%     Wenn der Parameter leer bleibt ('{}'), wird der Text nicht angezeigt.
%     Wird auch nur angezeigt, wenn ein Buchcover angegeben ist (Parameter #5).
%   - Kommentar zu diesem Eintrag der Literaturliste
%--------------------------------------
\newcommand{\sttpKommLitItem}[7]{
	\needspace{2\baselineskip} \noindent 
	\textit{#1 
%		\ifthenelse{\equal{#5}{\empty}}
%			{} % kein Buchcover angegeben
%			{
%				\ifthenelse{\equal{#6}{\empty}}
%				{\sttpMarginPicture{#5}} % Anzeige Buchcover ohne Text
%				{\sttpMarginPictureWithText{#5}{#6}} % Anzeige Buchcover mit Text
%			} % else
		(#2). #3
	}~\cite{#4}\smallskip\\#7\\
}
%--------------------------------------
% Dieses Kommando ist ein Helfs-Befehl, mit dem eine zusätzliche die Fußnote hinter der Lietratur-Referenz in der 'kommentierten Literaturliste' erzeugt werden kann (dies kommt beim ersten Eintrag in der Einleitung vor).
%   - Fußnoten-Text (8. Parameter)
\newcommand{\sttpKommLitItemMitFussnote}[8]{
	\needspace{2\baselineskip} \noindent 
	\textit{#1 
%		\ifthenelse{\equal{#5}{\empty}}
%			{} % kein Buchcover angegeben
%			{
%				\ifthenelse{\equal{#6}{\empty}}
%				{\sttpMarginPicture{#5}} % Anzeige Buchcover ohne Text
%				{\sttpMarginPictureWithText{#5}{#6}} % Anzeige Buchcover mit Text
%			} % else
		(#2). #3
	}~\cite{#4}\footnote{#8}\smallskip\\#7\\
}
%--------------------------------------

%--------------------------------------
% Ich verwende ein neues Kommando für Bilder, die im Seitenrand (marginpar) an
% der oberen Kante eines Absatzes positioniert werden.
% Der Befehl hat 1 Parameter
%   - Dateinamen der Grafik (Buchcover)
% Benutzt wird dies für die Buchcover der Werke, die in den kommentierten Literaturlisten vorgestellt werden.
% Quelle: Bild in Marginpar an oberer Kante des Absatzes positionieren
% https://texwelt.de/fragen/11095/bild-in-marginpar-an-oberer-kante-des-absatzes-positionieren
%--------------------------------------
\newcommand{\sttpMarginPicture}[1]{%
	\sttpMarginPictureWithText{#1}{}
}
%--------------------------------------
% Den Befehl gibt es auch noch in einer etwas abgewandelten Form, so dass eine zusätzliche Seitenangabe gemacht werden kann.
% Der Befehl hat 2 Parameter
%   - Dateinamen der Grafik (Buchcover)
%   - Text für die Seitenangabe
%     Diese wird unterhalb des Buchcovers angezeigt.
%--------------------------------------
\newcommand{\sttpMarginPictureWithText}[2]{%
  \leavevmode\marginline{%
    \raisebox{\dimexpr-\totalheight+\ht\strutbox\relax}%
      [\ht\strutbox][\dp\strutbox]{
					\parbox[b]{2.6cm}{\centering \includegraphics[width=2.2cm]{#1}\\#2}
		}
	}
}
%--------------------------------------
% Das folgende Makro ist für ein Bild im Seitenrand (marginpar) gedacht.
% Das Makro sttpMarginPicture kann dafür nicht verwendet werden, da das für die Buchcover gebastelt wurde.
% Auch hier sollen Bilder im Seitenrand (marginpar) an der oberen Kante eines Absatzes positioniert werden.
% Der Befehl hat 1 Parameter
%   - komplette Angabe für das einzufügende Bild (samt "\includegraphics[width...]{}")
% Benutzt wird dies erstmal für die kleinen bunten Kästen, die aus der Abbildung 3.1 (MindMap) stammen.
%
% Quelle: Bild in Marginpar an oberer Kante des Absatzes positionieren
% https://texwelt.de/fragen/11095/bild-in-marginpar-an-oberer-kante-des-absatzes-positionieren
%--------------------------------------
\newcommand{\sttpForImageInMargin}[1]{%
	\leavevmode\marginline{%
		\raisebox{\dimexpr-\totalheight+\ht\strutbox\relax}%
		[\ht\strutbox][\dp\strutbox]{
			\parbox[b]{\marginparwidth}{#1}
		}
	}
}
%--------------------------------------

% neues Kommando für einen Autoren-Kasten
% In einem Autorenkasten wird das Bild des Autors, der Name, das Geburtsdatum und ein erklärender Text angezeigt.
% enthält 5 Parameter
%   [1] - Name des Autors
%   [2] - Geburtsjahr
%   [3] - Todesjahr (wird nur angezeigt, wenn nicht leer)
%   [4] - erweiterter Text
%   [5] - Bilddatei
%   [6] - Datum der Aufnahme (nur Jahr)
%   [7] - Quellenangabe
\newcommand{\sttpAutorenkasten}[7]{
	\vspace{2ex}
	\phantomsection % damit hyperref korrekt funktioniert
	\addcontentsline{lofr}{section}{Foto #1 #6. #7} % Eintrag fuer das Foto in das Quellenverzeichnis
	\begin{tcolorbox}[
			center,
			width=0.8\textwidth,
			enhanced,
			%--Beginn Bildquelle--Angaben für die Bildquelle im "title"
			flip title={sharp corners},
			coltitle=black,
			colbacktitle=gray!5,
			fonttitle=\tiny,% andere mögliche Größen: \small, \footnotesize, \scriptsize, \tiny
			title={Bildquelle: #7 (#6)},
			%--Ende Bildquelle--
			sidebyside, % upper and lower part nebeneinander statt untereinander
			segmentation empty, % keine Trennlinie
			lefthand width=4.0cm,
			boxrule=2.4pt,
			left=2mm, right=2mm, top=2mm, bottom=2mm, middle=2mm,
			titlerule=0pt,
			colback=gray!5,
			colframe=FernUni-MI-green!30,
			sharp corners,
			drop fuzzy shadow
		]
		% -------------------------------------------------
		\begin{center}
			\includegraphics[width=2.5cm]{#5}%
			\\
			\vspace{2mm}
			\textbf{\mbox{#1}}
			\\
			\textbf{\footnotesize{\mbox{(\**#2\ifthenelse{\equal{#3}{\empty}}{}{,~\dag#3})}}}
		\end{center}
		\tcblower % -------------------------------------------------
		#4
		% -------------------------------------------------
	\end{tcolorbox}
	\vspace{2ex}
}

%%--------------------------------------
% neues Kommando für einen Definitions-Kasten
% In einem Definitions-Kasten wird ein Begriff, ein Kurztext (kursiv) und ein Langtext (normal) angezeigt.
% Das Kommando enthält 4 Parameter
%   - relative Breite des Kastens zur Textbreite
%   - Begriff, wird fett in einer farblich unterlegten Box dargestellt
%   - Kurztext, wird kursiv dargestellt (bei leerem Parameter entfällt dieser Text)
%   - Langtext, wird normal dargestellt (bei leerem Parameter entfällt dieser Text)
%%--------------------------------------
% Alle Definitionskästen haben eine Breite von 0.8\textwidth.
% In Kapitel 5 (Petrinetze / PN) werden die Kästen mit 0.9 skaliert.
\newcommand{\sttpDefinitionskastenSkalierungsfaktor}{0.8}
\newcommand{\sttpDefinitionskastenSkalierungsfaktorKapPN}{0.9}
%%--------------------------------------
\newcommand{\faktorBreite}{\sttpDefinitionskastenSkalierungsfaktor}
\newcommand{\sttpDefinitionskasten}[4]{
	\vspace{2ex}
	\begin{tcolorbox}[
		center,
		width=#1\textwidth,
		enhanced,
		adjusted title=#2,
		% Das kleine "i"-Symbol in der Titelzeile des Kastens entfernen.
		%after title={\hfill\includegraphics[height=0.3cm]{Bilder/i.png}},
		fonttitle=\sffamily\bfseries,
		coltitle=FernUni-MI-green!80!black,
		enlarge top initially by=0mm,
		enlarge bottom finally by=2mm,
		left=2mm,right=2mm,middle=0.5mm,
		segmentation hidden,
		sharp corners,
		boxrule=2.4pt,
		colback=gray!5,
		colframe=FernUni-MI-green!30,
		drop fuzzy shadow,
	]
% TODO AF: nochmal prüfen
%		% -------------------------------------------------
%		% Kurztext nur anzeigen, wenn nicht leer
%		\ifthenelse{\equal{#2}{\empty}}{}{\textit{#2}}
%		% -------------------------------------------------
%		% den Wechsel "`tcblower"' nur dann einfügen, wenn beide Elemente nicht leer sind
%		\ifthenelse{\equal{#2}{}}{}{\ifthenelse{\equal{#3}{}}{}{\tcblower}}
%		% -------------------------------------------------
%		% Langtext nur anzeigen, wenn nicht leer
%		\ifthenelse{\equal{#3}{\empty}}{}{#3}
%		% -------------------------------------------------
%
		% -------------------------------------------------
		% Kurztext nur anzeigen, wenn nicht leer
		\ifx&#3&\empty\else\textit{#3}\fi
		% -------------------------------------------------
		% Den Wechsel "`tcblower"' nur dann einfügen, wenn beide Elemente nicht leer sind
		\ifx&#3&\empty\else\ifx&#4&\empty\else\tcblower\fi\fi
		% -------------------------------------------------
		% Langtext nur anzeigen, wenn nicht leer
		\ifx&#4&\empty\else#4\fi
		% -------------------------------------------------
	\end{tcolorbox}
	\vspace{2ex}
}

%%--------------------------------------
% neues Kommando für einen Kasten mit Theorem
% In einem Kasten mit Theorem wird der angegebene Text angezeigt, zusammen mit dem String "Theorem:".
% Das Kommando enthält 2 Parameter
%   - relative Breite des Kastens zur Textbreite
%   - Textinhalt des Theorems
%%--------------------------------------
% Ein Kasten für ein Therorem hat standardmäßig die Breite 0.9\textwidth.
\newcommand{\sttpTheoremSkalierungsfaktor}{0.9}
%%--------------------------------------
\newcommand{\sttpTheorem}[2]{
	\vspace{1ex}
	\begin{tcolorbox}[
		center,
		width=#1\textwidth,
		colback=FernUni-MI-green!5!white,
		colframe=FernUni-MI-green!40!white
	]
	\textbf{Theorem:} #2
	\end{tcolorbox}
	\vspace{1ex}
}

% neues Kommando für einen Hinweis-Kasten
% In einem Hinweis-Kasten wird - ähnlich wie in einem Definitions-Kasten - ein Kurztext (fett) und ein Langtext (normale Schriftgröße) angezeigt.
% Zusätzlich erhält der Kasten 'Hinweis' als Überschrift.
% Das Kommando enthält 3 Parameter
%   - relative Breite des Kastens zur Textbreite
%   - Kurztext, wird fett dargestellt (bei leerem Parameter entfällt dieser Text)
%   - Langtext, wird normal dargestellt (bei leerem Parameter entfällt dieser Text)
\newcommand{\sttpHinweiskasten}[3]{
	\vspace{2ex}
	\begin{tcolorbox}[
		center,
		width=#1\textwidth,
		enhanced,
		title=Hinweis,
		attach boxed title to top left={xshift=-1mm,yshift=-\tcboxedtitleheight+1mm},
		boxed title style={colframe=FernUni-MI-green,colback=FernUni-MI-green!50},
		fonttitle=\sffamily\bfseries,
		top=\tcboxedtitleheight, % Abstand oben, gleiche Höhe wie 'height of the boxedtitle'
		middle=0.5mm,
		segmentation hidden,
		colback=gray!5,
		colframe=FernUni-MI-green!30,
		drop fuzzy shadow
	]
		% -------------------------------------------------
		% Kurztext (fett) nur anzeigen, wenn nicht leer
		\ifthenelse{\equal{#2}{\empty}}{}{\textbf{#2}}
		% -------------------------------------------------
		% den Wechsel Upper/Lower (tcblower) nur dann einfügen, wenn beide Elemente nicht leer sind
		\ifthenelse{\equal{#2}{}}{}{\ifthenelse{\equal{#3}{}}{}{\tcblower}}
		% -------------------------------------------------
		% Langtext (normal) nur anzeigen, wenn nicht leer
		\ifthenelse{\equal{#3}{\empty}}{}{{#3}}		
		% -------------------------------------------------
	\end{tcolorbox}
	\vspace{2ex}
}

% neues Kommando für einen Kasten
% In einem Kasten wird - ähnlich wie in einem Hinweis-Kasten - ein Text angezeigt.
% Der Kasten geht allerdings über die gesamte Textbreite, die Hintergrundfarbe ist weiß und der Rahmen ist etwas schmaler.
% Das Kommando enthält 1 Parameter
%   - Text, der angezeigt werden soll
\newcommand{\sttpKasten}[1]{
	\vspace{1ex}
	\begin{tcolorbox}[width=\textwidth,
		left=2mm,right=2mm,middle=0.5mm,
		% breakable, % breakable führt dazu, dass die Fußnoten innerhalb einer tcolorbox dargestellt werden und nicht am Seitenende
		bicolor,sharp corners,boxrule=1.5pt,colback=white,colbacklower=white,colframe=FernUni-MI-green!50]
		#1
	\end{tcolorbox}
	\vspace{1ex}
}
% Bei diesem Kasten ist zudem Seitenumbruch innerhalb des Kastens erlaubt.
% Aber Achtung: dann sind keine "normalen" Fußnoten möglich.
\newcommand{\sttpKastenBreakable}[1]{
	\vspace{1ex}
	\begin{tcolorbox}[width=\textwidth,
		left=2mm,right=2mm,middle=0.5mm,
		breakable, % breakable führt dazu, dass die Fußnoten innerhalb einer tcolorbox dargestellt werden und nicht am Seitenende
		bicolor,sharp corners,boxrule=1.5pt,colback=white,colbacklower=white,colframe=FernUni-MI-green!50]
		#1
	\end{tcolorbox}
	\vspace{1ex}
}

% neues Kommando für ein Zitat (wichtiges Zitat, dass direkt zum Text gehört)
% Das Zitat selbst wird kursiv dargestellt.
% Es soll kein Einzug der ersten Zeile vorgenommen werden (\noindent).
% Die Quelle erscheint rechtsbündig und in etwas kleinerer Schriftart darunter.
% Vor und nach dem Zitat wird ein vertikaler Abstand zum restlichen Text eingefügt.
% ---
% Anmerkung von Maren:
% Quelle kommt direkt hinter das Zitat in den ersten Parameter, 
% zweiter Parameter bleibt leer, wenn nur Quelle. 
% Zweiter Parameter für inhaltliche Zusätze 
% ---
% Das Kommando enthält 2 Parameter
%   - Zitattext (wird kursiv dargestellt)
%   - Quelle (wird rechtsbündig dargestellt) (kann auch leer bleiben)
\newcommand{\sttpzitat}[2]{
	\vspace{1ex}
	\noindent
	\textit{#1}

	\ifthenelse{\equal{#2}{\empty}}
	{} % keine Ausgabe, wenn #2 leer ist
	{
		{\raggedleft
			\small #2\\ % Anmerkung: die abschließenden "`\\" werden benötigt, damit der Text wirklich rechtsbündig ausgegeben wird
		}
	}
	\vspace{1ex}
}

% neues Kommando für ein Seitenrand-Zitat
% Ein Seitenrand-Zitat reicht nicht über die gesamte Seitenbreite.
% Es wird in einem Kasten dargestellt, der "`Außenrand-bündig"' ausgerichtet ist.
% Das Zitat selbst wird kursiv dargestellt.
% Die Quelle erscheint rechtsbündig und in etwas kleinerer Schriftart darunter.
% Vor und nach dem Zitat wird ein vertikaler Abstand zum restlichen Text eingefügt.
% Das Kommando enthält 2 Parameter
%   - Zitattext (wird kursiv dargestellt)
%   - Quelle (wird rechtsbündig dargestellt)
\newcommand{\sttpseitenrandzitat}[2]{
	\begin{tcolorbox}[
		if odd page={flush right}{flush left}, % ACHTUNG: die Unterscheidung linke Seite / rechte Seite geschieht manchmal erst nach dem 2. TeX-Durchlauf!
		width=1.0\textwidth,
		%width=0.75\textwidth, % alte Einstellung
		top=1.5mm,bottom=1.5mm,left=1.5mm,right=1.5mm,middle=0mm,
		bicolor,sharp corners,boxrule=1pt,colback=white,colbacklower=white,colframe=FernUni-MI-green!30,
		halign lower=right]
		% \tcbifoddpage{Odd}{Even} page!\\
		\textit{#1}
		\tcblower
		{\small #2}
	\end{tcolorbox}
}

%%% TODO kommentieren
% neues Kommando für die Leserführung (Einsatz erstmal nur in Kapitel 2)
% Die beiden Grafiken werden mittels zweier Minipages nebeneinander positioniert.
% Das Kommando enthält 2 Parameter
%   - Dateiname der linken Grafik
%   - Dateiname der rechten Grafik
\newcommand{\sttpLeserfuehrung}[2] {
	\vspace{1ex}
	\begin{center}
		\begin{minipage}[c]{.34\textwidth} 
			\includegraphics[scale=0.25]{#1}
		\end{minipage}
		\begin{minipage}[c]{.65\textwidth}
			\centering
			\includegraphics[scale=0.9]{#2}
		\end{minipage}
	\end{center}
	\vspace{1ex}
}


%--------------------------------------
% Das Kommando \sttpUMLText ist ein zentrales Marko, mit dem
% Text auf eine bestimmte Art (Typewriter-Schriftart ohne 
% Umbruch) dargestellt werden kann.
% Aktuell wird der Text mit \texttt{...} und ohne Umbruch
% (\mbox{...}) formatiert.
%
% Das Kommando enthält 1 Parameter
%   - Text, der entsprechend dargestellt werden soll
%--------------------------------------
% Achtung: es dürfen keine Leerzeichen und Zeilenumbrüche innerhalb
% des folgenden Makros enthalten sein.
\newcommand{\sttpUMLText}[1]{\mbox{\texttt{#1}}}
%--------------------------------------

%--------------------------------------
% neues Kommando für einen Kasten, in dem eine Anforderung dargestellt wird
% Der Text wird in Schreibmaschinenschrift dargestellt.
% Der Text wird zentriert dargestellt.
% Der Text wird in einem Kasten (100% Textbreite) mit hellgrünem Rand dargestellt.
%
% Das Kommando enthält 1 Parameter
%   - Text, der entsprechend dargestellt werden soll
%--------------------------------------
\newcommand{\sttpAnforderungText}[1]{
	\begin{tcolorbox}[
		width=\textwidth,
		top=1.5mm,bottom=1.5mm,left=1.5mm,right=1.5mm,middle=0mm,
		sharp corners,
		boxrule=10pt,
		colback=white,
		colframe=FernUni-MI-green!30,
		halign=flush center,
		]
		\texttt{#1}
	\end{tcolorbox}
}


%--------------------------------------
% neues Kommando für eine farbige kleine Box mit enthaltenem Text
% TODO kommentieren ...
%--------------------------------------
\newtcbox{\sttpMindMapText}[1][red]{
	on line,
	arc=0pt,
	outer arc=2pt,
	colback={#1},
	colframe={#1},
	coltext=white,
	boxsep=0pt,
	left=4pt,right=4pt,top=2pt,bottom=1pt,
	boxrule=1pt, 
}
% folgende Farben werden für die Texte benötigt
\definecolor{colMindMap1}{RGB}{153,0,0}
\definecolor{colMindMap2}{RGB}{74,74,186}
\definecolor{colMindMap3}{RGB}{0,102,204}
\definecolor{colMindMap4}{RGB}{200,112,4}
\definecolor{colMindMap5}{RGB}{38,115,77}


%--------------------------------------
% neues Kommando für eine Karteikarte
% Das Kommando enthält die folgenden 9 Parameter:
%   - Breite der Karteikarte (3cm oder ..., bitte nicht in "0.8\textwidth" angeben, das funktioniert z.B. in Tabellen nicht)
%   - Skalierungsfaktor
%   - Rotierungswinkel
%   - Kategorie (Kopfzeile des Kastens)
%   - Überschrift im 1. Teil
%   - Text zum 1. Teil
%   - Überschrift im 2. Teil
%   - Text zum 2. Teil (wenn leer, dann ohne Anzeige vom 2. Teil (auch ohne Überschrift #7))
%   - Ersteller-Name (wenn leer, dann ohne Anzeige dieser Zeile)
%--------------------------------------
% fuer die Karteikarte verwenden wir eine Schreibschrift
\newcommand*\schreibschrift{\fontfamily{wela}\selectfont}
% Für die Karteikarte können diese "Variablen" für den Skalierungsfaktor und den Rotierungswinkel verwendet werden.
% Das macht insbesondere dann Sinn, wenn ein und dieselbe Karteikarte an verschiedenen Stellen in verschiedenen
% Skalierungsgrößen und verschieden rotiert verwendet werden soll.
\newcommand{\sttpKarteikarteSkalierungsfaktor}{1.0}
\newcommand{\sttpKarteikarteRotierungswinkel}{0}
%--------------------------------------
\newcommand{\sttpKarteikarte}[9]{
	\smallskip
	\begin{tcolorbox}[
		skin=bicolor,
		center,
		width=#1,
		%enhanced,
		adjusted title=#4,
		fonttitle=\large\sffamily\bfseries,
		toptitle=2pt,
		bottomtitle=2pt,
		coltitle=FernUni-MI-green!80!black,
		colback=white,%gray!5,
		colframe=FernUni-MI-green!30,
		colbacklower=FernUni-MI-green!10,
		boxrule=1pt,
		rounded corners,
		segmentation style={solid,line width=1pt}, % Trennlinie 
		% drop fuzzy shadow, % Schatten
		%
		% mit folgenden Zeilen kann man Karopapier in Hintergrund anzeigen
		%underlay={
		%    \begin{tcbclipinterior}
		%        \draw[help lines,step=4mm,FernUni-MI-green!10,shift={(interior.north west)}]
		%        (interior.south west) grid (interior.north east);
		%    \end{tcbclipinterior}
		%},
		scale=#2,
		rotate=#3,
		]
		% -------------------------------------------------
		\textsf{\textbf{#5}}\\
		\schreibschrift{\textbf{#6}}
		% -------------------------------------------------
		\ifthenelse{\equal{#8}{\empty}}
		{
			% Text zum 2. Teil leer -> keine Anzeige
		}
		{
			\tcbline
			% -------------------------------------------------
			\textsf{\textbf{#7}}\\
			\schreibschrift{\textbf{#8}}
		}
		% -------------------------------------------------
		% Wenn der Parameter für den Ersteller-Namen leer ist, fällt die Anzeige dieser zeile weg.
		\ifthenelse{\equal{#9}{\empty}}
		{
			% Ersteller-Name leer -> keine Anzeige
		}
		{
			% Ersteller-Name nicht leer -> Anzeige mittels tcblower
			\tcblower
			% -------------------------------------------------
			\small \textsf{\textbf{Ersteller:}} #9
			% -------------------------------------------------
		}
	\end{tcolorbox}		
	\smallskip
}

%--------------------------------------
% neues Kommando für einen Universal-Kasten
% In einem Universal-Kasten wird eine Überschrift (mittig in der Kopfzeile) und 
% ein Text (normal im Kasten) angezeigt. Der Kasten ist so breit wie der Text.
% Das Kommando enthält 2 Parameter
%   - Überschrift, wird fett und zentriert in einer farblich unterlegten Box dargestellt
%   - Text, wird normal dargestellt
% Ergänzung: Wenn der erste Parameter leer bleibt (keine Überschrift),
%            dann wird die Titelzeile im Kasten gar nicht angezeigt.
%--------------------------------------
\newcommand{\sttpUniversalkasten}[2]{
	\ifthenelse{\equal{#1}{\empty}}
	{
		% Ausgabe ohne Titelzeile, wenn #1 leer ist
		\begin{tcolorbox}[
			center,
			width=\textwidth,
			enhanced,
			%adjusted title=#1,
			%halign title=center,
			% Das kleine "i"-Symbol in der Titelzeile des Kastens entfernen.
			%after title={\hfill\includegraphics[height=0.3cm]{Bilder/elephant_emoji.png}},
			%fonttitle=\sffamily\bfseries,
			%coltitle=FernUni-MI-green!80!black,
			enlarge top initially by=0mm,
			enlarge bottom finally by=2mm,
			left=2mm,right=2mm,middle=0.5mm,
			segmentation hidden,
			sharp corners,
			boxrule=2.4pt,
			colback=gray!5,
			colframe=FernUni-MI-green!30,
			drop fuzzy shadow,
		]
		#2
		\end{tcolorbox}
	}
	{
		\vspace{2ex}
		% Ausgabe mit Titelzeile, wenn #1 NICHT leer ist
		\begin{tcolorbox}[
			center,
			width=\textwidth,
			enhanced,
			adjusted title=#1,
			halign title=center,
			% Das kleine "i"-Symbol in der Titelzeile des Kastens entfernen.
			%after title={\hfill\includegraphics[height=0.3cm]{Bilder/elephant_emoji.png}},
			fonttitle=\sffamily\bfseries,
			coltitle=FernUni-MI-green!80!black,
			enlarge top initially by=0mm,
			enlarge bottom finally by=2mm,
			left=2mm,right=2mm,middle=0.5mm,
			segmentation hidden,
			sharp corners,
			boxrule=2.4pt,
			colback=gray!5,
			colframe=FernUni-MI-green!30,
			drop fuzzy shadow,
		]
		#2
		\end{tcolorbox}
		\vspace{2ex}
	}
}


%--------------------------------------
% Die folgenden Einstellungen sind für die "algorithm"-Umgebung,
% die in Kurseinheit 7 / Kapitel 11 verwendet wird.
% z.B. Farben in der algorithm-Umgebung
%--------------------------------------
% Text-Style in Conditions (If, While, Repeat, ...)
% Wird gebraucht für "if odd(n)". Ansonsten ist "odd" immer kursiv.
% Quelle: https://tex.stackexchange.com/questions/301793/in-algorithm2e-how-to-force-non-italic-font-in-the-condition-block-of-while
\SetArgSty{textnormal}
%--------------------------------------
% Einzelne Zeilen können farblich hervorgehoben werden.
% Quelle: https://tex.stackexchange.com/questions/149779/how-can-i-colourfuly-highlight-some-lines-of-an-algorithm-using-algorithm2e
\def\HiLi{\leavevmode\rlap{\hbox to \hsize{\color{FernUni-MI-green!5}\leaders\hrule height .8\baselineskip depth .5ex\hfill}}}
%--------------------------------------
% Titelzeile wird farblich unterlegt.
% Quelle: https://tex.stackexchange.com/questions/176057/algorithm2e-and-color
\makeatletter
\renewcommand{\algocf@makecaption@ruled}[2]{%
	\global\sbox\algocf@capbox{\colorbox{FernUni-MI-green!20}{\hskip\AlCapHSkip% .5\algomargin%
			\parbox[t]{\hsize}{\algocf@captiontext{\strut#1}{\strut#2\strut}}\hskip 1.5\algomargin}}% then caption is not centered
}%
\setlength{\interspacetitleruled}{0pt}%
\makeatother
%--------------------------------------
% "ggt" wird in Formeln damit nicht kursiv dargestellt
\newcommand{\ggt}{{\text{ggT}}}
%--------------------------------------

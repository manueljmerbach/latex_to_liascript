\section{Kommentierte Literatur}
\label{sec:Kap-4.4}


\sttpKommLitItem{Lahres/Raýman/Strich}{2018}{Objektorientierte Programmierung}{lah18}{Bilder/Buchcover/Buchcover_Lahres_Rayman_Strich.png}{}
{Häufig sind Bücher zu objektorientierter Programmierung eigentlich nur Einführungen in konkrete objektorientierte Programmiersprachen. Das ist hier nicht so. Das Buch beschäftigt sich sehr systematisch und konzeptionell und dabei gut verständlich mit den Prinzipien, der Methodik, den Elementen und dem Einsatz objektorientierter Softwareentwicklung, unabhängig von konkreten Programmiersprachen. Den Schwerpunkt bilden Implementierungsaspekte, dennoch findet man auch viele Informationen zur Objektorientierung im Allgemeinen und Aspekten der objektorientierten Realweltmodellierung. Neben Kapitel~2, das die Objektorientierung im Vergleich zu anderen Programmiermethodiken vorstellt, ist für die Inhalte dieser Lektion vor allem Kapitel~4 relevant. Dieses stellt die Konzepte Objekt und Klasse, ihre Charakteristika und ihre Einsatzmöglichkeiten umfassend vor.}

\sttpKommLitItem{Kecher/Salvanos/Hoffmann-Elbern}{2018}{UML~2.5}{kec18}{Bilder/Buchcover/Buchcover_Kecher_Salvanos.jpg}{}
{Das Kapitel zum UML-Klassendiagramm (S.~37~ff.) ist fast hundert Seiten lang. Sämtliche Elemente, die in einem Klassendiagramm vorkommen können, werden hier gut verständlich mit Realweltbeispielen beschrieben. Bei vielen Elementen wird sogar zusätzlich eine mögliche Umsetzung in Programmcode mit angegeben. Für Anfänger besonders hilfreich sind die Hinweise, für welche Modellierungszwecke sich welche Elemente besonders anbieten.}

\sttpKommLitItem{Sommerville}{2018}{Software Engineering}{som18}{Bilder/Buchcover/Buchcover_Sommerville.jpg}{}
{Kapitel~5 des Lehrbuchs stellt verschiedene Modellarten der UML und ihre Einsatzzwecke im Softwareengineering vor, das UML-Klassendiagramm wird in Kapitel~5.3 behandelt. Im Unterschied zu anderen Lehrbüchern beziehen sich die UML-Beispiele in diesem Buch immer auf eines der vier durchgehenden Fallbeispiele, die in Kapitel~1 vorgestellt werden und sich durch alle Themen des Buchs ziehen. Das hat den Vorteil, dass man sich nicht ständig in neue Kontexte einarbeiten muss, wenn man auch andere Softwareengineering-Themen mit diesem Buch lernen möchte; aber auch den Nachteil, dass es nicht immer die am intuitivsten passenden Beispiele für ein konkretes UML-Konzept sind. Insgesamt ist es gerade für Objektorientierung- und UML-Anfänger sehr zu empfehlen, mit mehreren Büchern zu arbeiten, da jede Autorin/ jeder Autor einen etwas (oder auch sehr) anderen Zugang zum Thema hat.}

\sttpKommLitItem{Brügge/Dutoit}{2006}{Objektorientierte Softwaretechnik}{bru06}{Bilder/Buchcover/Buchcover_Bruegge_Dutoit.jpg}{}
{Kapitel~2.3 des Lehrbuchs stellt unter anderem die Begriffe Modellierung, Objekt, Klasse, Datentyp und Domäne vor. Kapitel~2.4 beschäftigt sich mit unterschiedlichen Diagrammarten der UML, unter anderem mit dem Klassendiagramm. Hier wird auf knapp zehn Seiten ein guter erster Einblick in die wichtigsten Aspekte des UML-Klassendiagramms anhand verschiedener Beispiele gegeben. Für eine intensivere Beschäftigung mit Klassendiagrammen – und anderen UML-Diagrammen – sollte man aber immer auch die speziellen UML-Bücher wie \cite{kec18} zu Rate ziehen.}

\sttpKommLitItem{Oestereich/Scheithauer}{2013}{Analyse und Design mit der UML~2.5}{oes13}{Bilder/Buchcover/Buchcover_Oesterreich_Scheithauer.jpg}{}
{Abschnitt~7 (S.~17~ff) enthält eine Einführung in die Objektorientierung mit den wichtigsten Aspekten in Bezug auf Klassen und Objekte.}


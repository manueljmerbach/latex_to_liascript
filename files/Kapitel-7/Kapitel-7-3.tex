\section{Kommentierte Literatur}
\label{sec:Kap-7.3}


\sttpKommLitItem{Sommerville}{2020}{Modernes Software-Engineering}{som20}{}{}
{Kapitel 4 behandelt das Thema Softwarearchitektur mit Schwerpunkt auf Einflussfaktoren und Architekturmustern. Dieses Kapitel haben wir als Grundlage für die Themen in Abschnitt~\ref{sec:Kap-7.1} verwendet. Deshalb werden Sie hier fündig, wenn Sie zu diesen Themen noch weiter ins Detail gehen möchten. Entwurfsprinzipien werden nur am Rande im Zusammenhang mit Cross-Cutting Concerns behandelt.}

\sttpKommLitItem{Lahres/Raýman/Strich}{2018}{Objektorientierte Programmierung}{lah18}{Bilder/Buchcover/Buchcover_Lahres_Rayman_Strich.png}{}
{Kapitel 3 stellt Entwurfsprinzipien der objektorientierten Programmierung vor. 
\linebreak %%% für Druck
Außer dem Interface Segregation Principle kommen alle im Lerntext behandelten Prinzipien vor, teilweise mit etwas anderen Namen. Zudem werden weitere Entwurfs\-prinzipien thematisiert. Die Darstellung der Prinzipien ist gut zu lesen, weil viel mit Beispielen und mit Analogien aus der Realwelt gearbeitet wird.}

\sttpKommLitItem{Broy/Kuhrmann}{2021}{Einführung in die Softwaretechnik}{bro21}{}{}
{Sehr ausführliche Darstellung der Prozesse des Softwareengineering. Dem Prozess Entwurf sind die Kapitel acht bis zehn gewidmet, insgesamt fast 120 Seiten. Einen großen Stellenwert bekommt dort auch das Thema Wiederverwendung bewährten Wissens. Innerhalb von Kapitel 8 und von Kapitel 10 befinden sich kurze, gut verständliche Zusammenfassungen vieler verschiedener Entwurfsprinzipien (S. 329-341 und 405-417).}

\sttpKommLitItem{Ousterhout}{2021}{Prinzipien des Softwaredesigns}{ous21}{}{}
{Sehr auf den praktischen Einsatz orientierte Darstellung zahlreicher Entwurfs-
\linebreak %%% für Druck
prinzipien, die auf die niedrige Abstraktionsebene der Klassen fokussiert. Eher kein Nachschlagewerk für einzelne Prinzipien, sondern auf sequentielles Lesen ausgerichtet. Hilfreich sind die guten Zusammenfassungen am Ende jedes Kapitels.}

\sttpKommLitItem{Sommerville}{2018}{Software Engineering}{som18}{Bilder/Buchcover/Buchcover_Sommerville.jpg}{}
{Im Überblickskapitel über Softwareprozesse (Kap. 2) befindet sich ein kurzer, sehr gut zusammenfassender Abschnitt über den Ablauf und die Aufgaben im Software\-engineering-Prozess Entwurf (S. 67-69). Kapitel 6 befasst sich ausführlich mit dem Thema Softwarearchitektur. Hier finden Sie auch noch weitere Architekturmuster.}


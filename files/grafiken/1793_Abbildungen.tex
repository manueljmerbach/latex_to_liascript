% das Papierformat zuerst
% \documentclass[a4paper, 11pt]{article}
\documentclass[11pt]{scrartcl}

% für Umlaute
\usepackage{ucs}
\usepackage[utf8x]{inputenc}
\usepackage[T1]{fontenc}

% für "Deutsch"
\usepackage[ngerman]{babel}

% für Grafiken
\usepackage{graphicx}

\usepackage{caption}
\captionsetup{format=plain}

% Das Paket wird für die Verwendung der foreach-Schleife benötigt.
\usepackage{pgffor}

% Das hyperref-Paket wird üblicherweise als letztes Paket in der Präambel 
% geladen, da es sich so auf alle zuvor geladenen Pakete einstellen kann.
\usepackage{hyperref}

% für die Titelseite
\title{Kurs 1793 - Abbildungen}
\author{Maren Stephan}
\date{\today{}, Hagen}

% hier beginnt das Dokument
\begin{document}

\maketitle

\newpage

% Inhaltsverzeichnis anzeigen
\tableofcontents

\clearpage

%-------------------------------------------------------------------------
%%--- Zu jedem Kapitel sollen alle Grafiken ausgegeben werden.
%%--- Der Tipp für die Umsetzung stammt von hier:
%%--- https://tex.stackexchange.com/questions/7653/how-to-iterate-through-the-name-of-files-in-a-folder

\newcommand*{\MaxNumOfChapters}{10}% Adjust these two settings for your needs.
\newcommand*{\MaxNumOfFigures}{50}%

\foreach \c in {1,2,...,\MaxNumOfChapters}{%

		% Kapitelueberschrift
		\section{Abbildungen zu Kapitel \c}

    \foreach \f in {1,2,...,\MaxNumOfFigures}{%
        \IfFileExists{Kapitel-\c/Abb-\c-\f.pdf} {%
					\subsection{Abbildung \c-\f}
					\begin{figure}[htbp]
						\centering
						\setlength{\fboxsep}{5pt}
						\setlength{\fboxrule}{2pt}
						\fbox{\includegraphics{../grafiken/Kapitel-\c/Abb-\c-\f.pdf}}
						\caption{Abb-\c-\f.pdf}
						\label{fig:Abb-\c-\f}
					\end{figure}
				}{%
					% files does not exist, so nothing to do
				}%
    }%
}%

%%-----------------------------------
%%--- Abbildungsverzeichnis, etc. ---
%%-----------------------------------
\clearpage
\listoffigures
\clearpage

\end{document}
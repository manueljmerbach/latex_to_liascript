\section{Kommentierte Literatur}

\sttpKommLitItem{Weske}{2024}{Business Process Management}{wes24}{}{}
{Sehr gute Grundlagenliteratur zum Thema Prozessmanagement. \textit{Part II Business Process Modelling} gibt einen Überblick über (Geschäfts-)Prozessmodellierung mit Fokus auf BPMN und Workflow-Management. Auf Englisch.}

\sttpKommLitItem{Aalst}{2016}{Process Mining}{aal16}{}{}
{Einführung in die Themen Data Science und Process Mining. Es werden verschiedene Techniken erläutert, wie aus Daten Prozessmodelle entwickelt werden, aber auch, wie man diese analysieren kann. Auf Englisch.}

\sttpKommLitItem{Reisig}{2010}{Petrinetze}{rei10}{}{}
{Grundlagenliteratur zum Thema Petrinetze in einfacher und gut verständlicher Sprache. Hier wird von High-Level Petrinetzen als allgemeiner Fall ausgegangen -- ein sehr interessanter Ansatz.}

\sttpKommLitItem{Kecher/Salvanos/Hoffmann-Elbern}{2018}{UML~2.5}{kec18}{}{}
{Wie für die UML-Klassendiagramme ist dieses Buch durch seine zahlreichen Lebens\-welt\-beispiele und beispielhaften Programmcodeumsetzungen auch für Zustands-
	\linebreak %%% für Druck
	diagramme, Sequenzdiagramme und Aktivitätsdiagramme gerade für Anfänger der UML eine gute Wahl.}

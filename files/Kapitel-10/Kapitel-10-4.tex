\section{Entwurfsmuster vs. Frameworks vs. Bibliotheken}
\label{sec:Kap-10.4}

Es ist nicht sinnvoll, in jedem Softwareentwicklungsprojekt den Programmcode aus den basalen Elementen der Programmiersprache von Grund auf neu zu schreiben. Im Idealfall sollten sich die Entwickler in einem Projekt stattdessen auf diejenigen Aspekte konzentrieren können, die für das jeweilige Projekt spezifisch sind und für alles andere von bereits geleisteten Arbeiten und Erfahrungen profitieren.

Die Verwendung von Entwurfsmustern ist solch eine Möglichkeit, von Erfahrungen zu profitieren. Doch auch weniger abstrakte Vorarbeiten in Form von schon vor\-ge\-fertigten Codebausteinen und Programmlogiken werden verwendet, um die benötigte Software in der gewünschten Qualität zu entwickeln. \textit{Bibliotheken} und \textit{Frameworks} sind Möglichkeiten für die eigene zu erstellende Software auf bewährte, gut getestete und in zahlreichen anderen Entwicklungsprojekten eingesetzte Bestandteile zurückzugreifen.

Bibliotheken \marginline{Bibliotheken} (auch Klassenbibliotheken oder Programmbibliotheken genannt) bestehen aus wiederverwendbaren Klassen, die nützliche und allgemeine Funktionalitäten zur Verfügung stellen, die aus dem eigenen Programmcode heraus aufgerufen werden können. Viele Programmiersprachen bringen bereits eigene Bibliotheken mit. So besteht die Java-Standardbibliothek aus mehreren tausend Klassen mit Operationen für viele Problemstellungen. Eine Bibliothek ist kein eigenständig lauf{}fähiges Programm. Sie ist eine Sammlung von schon in Quellcode umgesetzten Funktionalitäten, von denen Entwickler die für ihr Programm benötigten Teile einsetzen. Die Verwendung einer Bibliotheksklasse bringt natürlich nur dann die genannten Vorteile, wenn sie genau zu der Anforderung passt und vom Programmierer genau so verwendet wird, wie vom Ersteller vorgesehen. Andernfalls steigt das Risiko für fehlerhafte Software sogar, wenn fremde Klassen blind zusammengewürfelt werden.
 
Beim Einsatz von Bibliotheken entwerfen die Entwickler das Grundgerüst für ihr Programm (Klassen und ihre Beziehungen, Ablauf des Programms) selber, evtl. ausgerichtet auf abstrakt beschriebene Entwurfsmuster, und binden die durch eine Bibliothek zur Verfügung gestellten Funktionalitäten an den geeigneten Stellen ihres Programmgerüsts ein.

Bei der Arbeit \marginline{Frameworks} mit Frameworks geht man den umgekehrten Weg. Ein Framework liefert bereits das Grundgerüst für das Programm. Die Entwickler müssen innerhalb dieses Grundgerüstes noch die Aspekte programmieren, die spezifisch für ihre Software sind (wobei auch an dieser Stelle aus Bibliotheken eingebunden werden kann).

Um es noch etwas komplizierter zu machen: In der Regel basieren Frameworks auf einer spezifischen Zusammenstellung mehrerer Entwurfsmuster. Und eigentlich ist es insofern die Kombination der verwendeten Entwurfsmuster, die das Programmgerüst bestimmt und das Framework nur die Umsetzung dieser Kombination. Des Weiteren können auch Bibliotheken auf Entwurfsmustern basieren, und wenn man diese einbindet, hat das evtl. doch Auswirkungen auf das eigene Programmgerüst.

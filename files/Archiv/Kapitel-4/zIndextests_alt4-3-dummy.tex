\section{Dummy-Section}

\lipsum[1][1-5]

\textbf{An dieser Stelle steht ein ganz normaler Indexeintrag.}
\index{Eintrag}

\lipsum[1][6-10]

\textbf{Und hier ist ein Unter-Eintrag enthalten.}
\index{Eintrag!Untereintrag} 

\lipsum[1][11-15]

\textbf{Hier beginnt ein "`LangIndexEintrag"'.}
\index{LangIndexEintrag|(}


\subsection{Dummy-Subsection}

\lipsum[2][1-3]  

\subsection{Dummy-Subsection}

\lipsum[3][1-5]

\textbf{Indexeinträge mit Umlauten müssen gesondert ausgezeichnet werden.}
\index{Aepfel@Äpfel}

\lipsum[3][6-10]

\subsection{Dummy-Subsection}

\lipsum[4][1-3]

\textbf{Ein Haupteintrag wird im Indexverzeichnis "`fett"' dargestellt.}
\index{Eintrag|textbf}

\lipsum[4][4-5]

\textbf{Man kann auch Unter-Unter-Einträge erstellen.}
\index{Eintrag!Untereintrag!Unter-Untereintrag} 

\lipsum[5][1-3]

\textbf{Hier endet der "`LangIndexEintrag"'.}
\index{LangIndexEintrag|)}

\minisec{weitere Tests für Indexeinträge}

An dieser Stelle stehen weitere Einträge für den Index. Ich will testen, wie es mit der Sortierung der Umlaute klappt.

\index{Apfel}

\index{Ampel}

\index{Ast}

\index{Aenderung@Änderung}
\index{Aepfelchen@Äpfelchen}
\index{Aeste@Äste}

\index{Abteilung}

Beim Befehl \texttt{makeindex} gibt es die Option \texttt{-g} (für \textit{german odering}).
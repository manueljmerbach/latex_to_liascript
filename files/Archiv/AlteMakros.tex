%%% Seitenrandkästen werden nicht mehr verwendet. Stattdessen Definitions-Kästen
%% neues Kommando für einen STTP-eigenen Seitenrandkasten (sttpsrk -> srk => Seitenrandkasten)
%% Dieser Seitenrandkasten wird immer zu einem Begriff im Text gebildet, der dann dort kursiv dargestellt wird.
%% Ein Seitenrandkasten erhält 4 Parameter
%%   - Begriff (der Begriff im Text wird kursiv dargestellt)
%%   - Überschrift (zentriert und fett)
%%   - Kurztext (zentriert)
%%   - Langtext
%\newcommand{\sttpsrk}[4]{\textit{#1} \marginpar{\fbox{\parbox{3cm}{\begin{center}\textbf{#2}\end{center}\begin{center}#3\end{center}#4}}}}

%%% Seitenrandkästen mit Bild werden nicht mehr verwendet.
%% neues Kommando für einen STTP-eigenen Seitenrandkasten mit Bild (sttpsrkbild -> srkbild => Seitenrandkasten mit Bild)
%% Ein Seitenrandkasten mit Bild erhält 3 Parameter
%%   - Überschrift (zentriert und fett)
%%   - Dateinamen der Grafik
%%   - Langtext
%\newcommand{\sttpsrkbild}[3]{\sttpsrk{}{#1}{\includegraphics[width=2.9cm]{#2}}{#3}}

%%% Reinragende Seitenrandkästen werden nicht mehr verwendet.
%% neues Kommando für einen STTP-eigenen reinragenden Seitenrandkasten (sttprrsrk -> rr => reinragend, srk => Seitenrandkasten)
%% Ein reinragender Seitenrandkasten erhält 3 Parameter
%%   - Überschrift (zentriert und fett)
%%   - Kurztext
%%   - Langtext
%\newcommand{\sttprrsrk}[3]{
	%\begin{wrapfigure}{o}[3cm]{12,5cm}%
	%\fbox{\parbox{12,2cm}{\begin{center}\textbf{#1}\end{center}\begin{center}#2\end{center}#3}}
	%\end{wrapfigure}
	%}

%%--------------------------------------
%% Kommando für einen Abbildungskasten
%% In einem Abbildungskasten wird eine Abbildung ähnlich wie in einem Autorenkasten dargestellt
%% (gleiche Farbe, Rahmen, Schatten, ...).
%% 
%% enthält 2 Parameter
%%   - Bilddatei
%%   - Skalierungsfaktor
%\newcommand{\sttpAbbildungskasten}[2]{
	%	\vspace{2em}
	%	\tcbox[
	%		center,							% zentriert
	%		enhanced,						% ohne enhanced kein "Schatten"
	%		boxrule=2.4pt,					% Rahmendicke
	%		left=2mm,						% Abstand zum Rand
	%		right=2mm,						% Abstand zum Rand
	%		top=2mm,						% Abstand zum Rand
	%		bottom=2mm,						% Abstand zum Rand
	%		colback=gray!5,					% Hintergrundfarbe
	%		colframe=FernUni-MI-green!30,	% Rahmenfarbe
	%		sharp corners,					% alle Ecken -> scharfe Ecken
	%		drop fuzzy shadow				% Schatten
	%	]
	%	{
		%		\includegraphics[scale=#2]{#1}%
		%	}
	%	\vspace{2em}
	%}
%%--------------------------------------

%%--------------------------------------
%% neues Kommando für das Logo im Seitenrand (STTP - Abbildung - Logo)
%\newcommand{\sttpabblogo}[1]{
	%	\marginline{\includegraphics[height=10pt]{#1}}
	%}
%%--------------------------------------

%%% TODO kommentieren
% neues Kommando für die Leserführung "MindMap (Einsatz hier: Kapitel 3)
% Das Makro sttpLeserfuehrungMindMap funktioniert wie das Makro \sttpLeserfuehrung.
% Es werden lediglich andere Sklaierungswerte verwendet.
% Die beiden Grafiken werden mittels zweier Minipages nebeneinander positioniert.
% Das Kommando enthält 2 Parameter
%   - Dateiname der linken Grafik
%   - Dateiname der rechten Grafik
%\newcommand{\sttpLeserfuehrungMindMap}[2] {
	%	\begin{center}
		%		\begin{minipage}[c]{.4\linewidth} 
			%			\includegraphics[width=\linewidth]{#1}
			%		\end{minipage}
		%		\hspace{.1\linewidth}% Abstand zwischen den beiden Bildern
		%		\begin{minipage}[c]{.4\linewidth}
			%			\includegraphics[width=\linewidth]{#2}
			%		\end{minipage}
		%	\end{center}
	%	\vspace{1ex}
	%}
%%--------------------------------------

%--------------------------------------
% neues Kommando für einen Überleitungs-Kasten
% Das Kommando enthält 4 Parameter
%   - Titel (Kopfzeile des Kastens)
%   - Themen (werden rechts neben dem Schriftzug "Themen" angezeigt)
%   - Text, der im Hauptfeld des Kastens steht
%   - Text in der sogenannten Statuszeile (für die URL)
%--------------------------------------
%\newcommand{\sttpUeberleitungskasten}[4]{
	%	\begin{tcolorbox}[
		%			center,
		%			width=0.9\textwidth,
		%			enhanced,
		%			adjusted title=#1,
		%			fonttitle=\sffamily\bfseries,
		%			coltitle=FernUni-MI-green!80!black,
		%			colback=gray!5,
		%			colframe=FernUni-MI-green!30,
		%			boxrule=2.4pt,
		%			sharp corners,
		%			segmentation style={FernUni-MI-green,solid,opacity=0.3,line width=2.4pt},
		%			drop fuzzy shadow,
		%		]
		%		% -------------------------------------------------
		%		\begin{labeling}{\textsf{\textbf{Themen:}}}
			%			\item [\textsf{\textbf{Themen:}}] #2
			%		\end{labeling}
		%		% -------------------------------------------------
		%		\tcblower
		%		% -------------------------------------------------
		%		#3
		%		% -------------------------------------------------
		%		\tcbline
		%		% -------------------------------------------------
		%		\footnotesize{\sttpHervorhebung{$\Rightarrow$} #4}
		%		% -------------------------------------------------
		%	\end{tcolorbox}		
	%}
%--------------------------------------


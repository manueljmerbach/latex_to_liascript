Kapitel 3.1

%\textcolor[RGB]{153,0,0}{\textbf{Modellierungsgegenstand}} von Modellen 
%\marginline{\includegraphics[width=\marginparwidth]{Bilder/Kapitel-3/Abb-3.1-MindMap-Kasten-Modellierungsgegenstand.pdf}}
%können Softwareentwicklungsprozesse sein (in der Abbildung oben links). Vertreter solcher Modelle haben Sie in Kapitel 2 %~\ref{sec:Kap-2}
%mit den Vorgehensmodellen schon kennengelernt. Anstelle des ...
%
%\textcolor[RGB]{153,0,0}{\textbf{Modellierungsgegenstand}} von Modellen 
%\sttpMarginPicture{Bilder/Kapitel-3/Abb-3.1-MindMap-Kasten-Modellierungsgegenstand.pdf}
%können Softwareentwicklungsprozesse sein (in der Abbildung oben links). Vertreter solcher Modelle haben Sie in Kapitel 2 %~\ref{sec:Kap-2}
%mit den Vorgehensmodellen schon kennengelernt. Anstelle des ...
%
%\textcolor[RGB]{153,0,0}{\textbf{Modellierungsgegenstand}} von Modellen 
%\sttpMarginPictureNeu{\includegraphics[width=\marginparwidth]{Bilder/Kapitel-3/Abb-3.1-MindMap-Kasten-Modellierungsgegenstand.pdf}}
%können Softwareentwicklungsprozesse sein (in der Abbildung oben links). Vertreter solcher Modelle haben Sie in Kapitel 2 %~\ref{sec:Kap-2}
%mit den Vorgehensmodellen schon kennengelernt. Anstelle des ...
%
%\includegraphics[height=2em]{Bilder/Kapitel-3/Abb-3.1-MindMap-Kasten-Modellierungsgegenstand.pdf} von Modellen 
%\marginline{Modellierungs\-gegenstand}
%können Softwareentwicklungsprozesse sein (in der Abbildung oben links). Vertreter solcher Modelle haben Sie in Kapitel 2 %~\ref{sec:Kap-2}
%mit den Vorgehensmodellen schon kennengelernt. Anstelle des ...
%
%\fcolorbox[RGB]{153,0,0}{153,0,0}{\textcolor{white}{\textsf{Modellierungsgegenstand}}} von Modellen 
%\marginline{Modellierungs\-gegenstand}
%können Softwareentwicklungsprozesse sein (in der Abbildung oben links). Vertreter solcher Modelle haben Sie in Kapitel 2 %~\ref{sec:Kap-2}
%mit den Vorgehensmodellen schon kennengelernt. Anstelle des ...








%TODO hier angepasste Abbildung aus Abb 3.1., ohne Abbildungsunterschrift, Einsatzzweck ist Leserführung, evtl. also sogar eingepasst in den Leserführungsumrandung 

%\sttpLeserfuehrungMindMap{Bilder/Kapitel-3/Abb-3-1-MindMap-Leserfuehrung-Illustration-Modellierungsgegenstand.pdf}{Bilder/Kapitel-3/Abb-3-1-MindMap-Leserfuehrung-Modellierungsgegenstand.pdf}

%Mögliche weitere Leserführungen für spätere Kapitel ???
%
%\sttpLeserfuehrungMindMap{Bilder/Kapitel-3/Abb-3-1-MindMap-Leserfuehrung-Illustration-Zielgruppe.pdf}{Bilder/Kapitel-3/Abb-3-1-MindMap-Leserfuehrung-Zielgruppe.pdf}
%
%\sttpLeserfuehrungMindMap{Bilder/Kapitel-3/Abb-3-1-MindMap-Leserfuehrung-Illustration-Perspektive.pdf}{Bilder/Kapitel-3/Abb-3-1-MindMap-Leserfuehrung-Perspektive.pdf}
%
%\sttpLeserfuehrungMindMap{Bilder/Kapitel-3/Abb-3-1-MindMap-Leserfuehrung-Illustration-Geltungsbereich.pdf}{Bilder/Kapitel-3/Abb-3-1-MindMap-Leserfuehrung-Geltungsbereich.pdf}
%
%\sttpLeserfuehrungMindMap{Bilder/Kapitel-3/Abb-3-1-MindMap-Leserfuehrung-Illustration-Modellierungszweck.pdf}{Bilder/Kapitel-3/Abb-3-1-MindMap-Leserfuehrung-Modellierungszweck.pdf}
%Kapitel 1.1

%Bild Computer
\begin{figure}[h!]
    \centering
    \includegraphics[scale=0.5]{Bilder/platzhalter/platzhalter.png}
    \caption{Computermodell der 1950er Jahre}
    \label{fig:bild_rechner_1950er}
\end{figure}

%Autorenkasten Bauer
\sttpAutorenkasten{Bilder/platzhalter/mann2.png}{Friedrich L. Bauer}{1924}{2015}{Deutscher Mathematiker. Entwickelte das Prinzip des Kellerspeichers (engl. stack) und erhielt daf�r den Computer Pioneer Award. Er forschte zu verschiedenen Themen der Mathematik sowie zu Programmiersprachen und -methoden, Softwareengineering und Kryptologie. Lehrte viele Jahre als Informatikprofessor an der TU M�nchen.}

%Autorenkasten Hoare

\sttpAutorenkasten{Bilder/platzhalter/mann1.png}{Sir Tony Hoare}{1934}{}{Britischer Informatiker. Entwickelte den Quicksort-Algorithmus und die Prozessalgebra CSP. Mit dem von ihm entwickelten Hoare-Kalk�l l�sst sich die Korrektheit von Algorithmen beweisen. F�r grundlegende Errungenschaften zur Definition und Entwicklung von Programmiersprachen wurde er mit dem Turing Award ausgezeichnet. Lehrte viele Jahre als Informatikprofessor an der Universit�t Oxford.}

%Autorenkasten Wirth
\sttpAutorenkasten{Bilder/platzhalter/mann4.png}{Niklaus Wirth}{1934}{}{Schweizer Informatiker. Entwickelte neben Pascal weitere Programmiersprachen. So war er an der Entwicklung von Euler und der Weiterentwicklung von Algol beteiligt und entwarf Modula-2 und Oberon. F�r seine Arbeiten im Bereich der Programmiersprachen wurde er mit dem Turing Award ausgezeichnet. Lehrte viele Jahre als Informatikprofessor in Z�rich.}
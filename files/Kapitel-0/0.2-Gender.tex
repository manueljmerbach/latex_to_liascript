\cleardoublepage
\chapter*{Die in diesem Text verwendete Sprache}
\addcontentsline{toc}{chapter}{Die in diesem Text verwendete Sprache}
\markboth{Die in diesem Text verwendete Sprache}{Die in diesem Text verwendete Sprache}
\label{sec:Kap-0.2}

\vspace{1cm} %%% für Druck

Liebe Leserin, lieber Leser,

oder sollte ich schreiben „Lesende“, was in diesem Fall ja tatsächlich mal zutrifft? In der Debatte um gendergerechte Schreibweisen von personenbezogenen Wörtern gibt es viele Vorschläge, und ich muss als Professor und Autor spätestens hier Farbe bekennen und mir überlegen, wie ich es in diesem Text mit Sprache und Gender halten werde. Das Ergebnis dieser Überlegungen möchte ich Ihnen hier vorstellen. Einerseits bin ich als Professor und Mitglied einer staatlichen Universität gehalten auch bezüglich Sprache die Vorgaben aus Politik und Hochschulleitung (und Gleich\-stellungsstelle) zu befolgen. Diese sind aber leider nicht eindeutig und machen teilweise Vorschläge, die ich einfach nur falsch finde (wie die Verwendung von „Studierende“ auch für gerade nicht studierende Studenten und Studentinnen). \mbox{Andererseits} unterliegt gerade ein Lehrtext – als Analogon zu einer Vorlesung – der zu Recht gesetzlich verbrieften Freiheit von Lehre. Dies ist nicht nur Privileg, sondern auch Verpflichtung. Gerade als Professor kann ich nicht etwas verbreiten und mich zugleich davon distanzieren mit der Begründung, ich hätte dies ja sagen müssen. Die Verwendung von Sprachformen bestimmt aber auch deren Inhalt, jedenfalls kann sie politisch relevant sein, und im Falle der Berücksichtigung von Genderformen ist sie dies gewiss.

In einem früheren Lehrtext habe ich für Singularformen stets das generische Femininum und für den Plural stets das generische Maskulin verwendet: eine Studentin, zwei Studenten usw. Es war mehr Experiment als Überzeugung, und kam nicht überall gut an. Ein Kollege bietet seinen Lehrtext in zwei Formen an, eine durchgehend männliche und eine durchgehend weibliche. Er erhält trotz des erheblichen Aufwands Wutmails, er würde die Genderfrage verhöhnen. Ich lerne daraus, dass man es sicher nicht jeder und jedem Recht machen kann und immer von irgendwem kritisiert wird. Ich lerne aber auch, dass es nicht reicht genervt irgendetwas zu machen mit der Begründung, es sei doch egal. Gleich welches Ergebnis meine Über\-legungen haben, man wird und soll sie mir persönlich zuschreiben. Um ein Statement zur Gendersprachenfrage komme ich als Autor nicht herum.

Ich habe für dieses Thema das lesenswerte Buch „Die Teufelin steckt im Detail - zur Debatte um Gender und Sprache“ (herausgegeben von André Meinunger und Antje Baumann, erschienen 2017 im Kulturverlag Kadmos) studiert, in dem 14 Fachleute ein Plädoyer für den jeweils von ihnen favorisierten Umgang mit Sprache darstellen. Am meisten überzeugt hat mich die Argumentation der Linguistin Heide Wegener in ihrem Artikel „Grenzen gegenderter Sprache – warum das generische Maskulinum fortbestehen wird, allgemein und insbesondere im Deutschen“. Die Zielrichtung wird schon durch den Titel deutlich. Hinzu kommt, dass der Wunsch der Sichtbarkeit von Frauen in allen Bereichen durch Abkehr von generischen Formen dazu geführt hat, dass jede(r) Lesende sich nur angesprochen fühlen kann, wenn er/sie zuvor die Frage klärt, ob er/sie denn nun als Mann oder als Frau angesprochen wird, auch wenn diese Frage mit dem gelesenen Text rein gar nichts zu tun hat. In jüngerer Zeit wird aber zu Recht gerade dies kritisiert; Menschen, die weder von außen geschlechtsbezogen verortet werden wollen noch selbst gerade jetzt diese Entscheidung treffen wollen, fühlen sich nur durch generische Formen angesprochen. Dem zuweilen vorgetragenen Versuch, diese generischen Formen durch einen Unterstrich oder ein Sternchen und der Verwendung weiblicher Formen wieder einzuführen, kann ich nichts abgewinnen, spätestens beim Sprechen entsprechender Texte scheitert er.

Ich habe viele Gespräche zu diesem Thema geführt, meine Ko-Autorin und ich haben gemeinsam beraten, ein entscheidender Hinweis kam aber von einer von mir sehr geachteten Kollegin (Informatikprofessorin), der sicher niemand mangelndes Bewusst\-sein bei diesem Thema vorwerfen würde (meiner Ko-Autorin ebenfalls nicht). Um diesen Vorschlag zu verstehen, sind Informatikkenntnisse hilfreich, aber da bin ich ja bei Ihnen an der richtigen Adresse. Das Verständnis dieser Idee hat sogar sehr viel mit Softwareengineering zu tun! Wir unterscheiden hier die Begriffe Menge und Klasse, und ich möchte den Unterschied an einem Beispiel verdeutlichen: bei der Modellierung eines Unternehmens mag es die Klasse Mitarbeiter geben, die sich nicht verändert, während die Menge der Mitarbeiter und Mitarbeiterinnen des Unternehmens mit jeder Neueinstellung und Kündigung variiert. Bezieht sich ein personen\-bezogener Bezeichner auf eine Klasse, werden wir das generische Maskulin verwenden, bezieht er sich auf eine Menge, dann werden wir dies unterlassen. Wir würden also nicht sagen „die Lehrer der Hotzenplotz-Schule in Dingsda~\ldots“, wenn sich in dem Lehrerkollegium auch Frauen befinden, aber eben Lehrerkollegium, Lehrerparkplatz usw.  

Wir treffen diese Unterscheidung auch deshalb, weil gerade bei der Anforderungsermittlung für Softwaresysteme sprachliche Präzision extrem wichtig ist. Aus Text soll sein formaler Inhalt extrahiert werden (das werden Sie in Übungs- und Klausur\-aufgaben leisten). Aus „Professoren und Professorinnen“ ist zu schließen, dass von jeder dieser Untergruppen wenigstens zwei existieren, weil die Pluralform gewählt wurde. Da der Lehrkörper eines Faches aus beliebig vielen männlichen und weiblichen Professoren bestehen kann, bräuchten wir so aber eine Fallunterscheidung aus acht Fällen (sofern wir die leere Menge ausschließen): „die Professorinnen oder die Professorinnen und der Professor oder die Professorinnen und die Professoren oder die Professorin und der Professor oder~\ldots“ – nicht lesbar und eigentlich ohne die Verwendung von Klammern auch mehrdeutig. Nun kommt hinzu, dass bei der Verwendung von „oder“ auf eine Alternative geschlossen werden kann und sich die Frage stellt, woher die Entscheidung kommt. Sie sehen schon: in der Fachsprache funktioniert das nicht. Wir möchten nicht Regeln für die Verwendung von Sprache ausgeben und ihre Einhaltung einfordern, an die wir uns in der Lehrsprache nicht halten wollen.

Jörg Desel
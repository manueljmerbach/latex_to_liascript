\section{Prinzipien}
\label{sec:Kap-9.1}

Sie haben in Lektion 5 eine Komponente als Einheit zusammengehöriger Funktionalität kennengelernt. Im Rahmen der Implementierung steht in Bezug auf die Komponenten die niedrige Abstraktionsebene der Klassen und Objekte im Fokus. Und anstelle des Begriffs Komponente verwendet man im Prozess der Implementierung eher den Begriff \textit{Modul}. Ein Modul ist eine zusammengehörige Einheit innerhalb des Programmcodes. Aus einer praktischen Perspektive gesehen definiert sich ein Modul als das, was „ein Programmierer als eine Einheit betrachtet, die als Ganzes bearbeitet und verwendet wird“ \cite[42]{lah18}. In objektorientierten Programmiersprachen entspricht ein Modul häufig einer einzigen Klasse oder einem Paket mehrerer logisch eng zusammengehöriger Klassen. 

Das Prinzip der Modularisierung, \marginline{Modularisierung} also der modulare Aufbau von Programmcode in voneinander unabhängige, aber zusammenarbeitende Einheiten, gehört zu den wichtigsten Techniken, um die Komplexität in der Softwareentwicklung beherrschbar zu halten. Es gibt verschiedene Prinzipien und Heuristiken, die vorschlagen, wie Module strukturiert sein sollten und wie sie miteinander interagieren. In Kapitel 7 haben Sie schon einige grundsätzliche Entwurfsprinzipien kennengelernt. Daran knüpfen wir in den folgenden Abschnitten an und sehen uns diese und weitere in der Umsetzung auf Modulebene an. Wir vereinfachen im weiteren Text, gehen davon aus, dass ein Modul genau einer Klasse entspricht und sprechen daher häufig konkret von der Klasse. Alle betrachteten Aspekte lassen sich aber auch auf Module übertragen, die aus mehreren Klassen bestehen.



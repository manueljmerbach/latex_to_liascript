\section{Kommentierte Literatur}
\label{sec:Kap-9.4}


\sttpKommLitItem{Kecher/Salvanos/Hoffmann-Elbern}{2018}{UML~2.5}{kec18}{Bilder/Buchcover/Buchcover_Kecher_Salvanos.jpg}{}
{Das Thema Sichtbarkeiten in UML wird direkt zu Beginn des schon bekannten \mbox{Kapitels} über Klassendiagramme bei der Einführung von Attributen behandelt. Im weiteren Verlauf des Kapitels finden Sie Beispiele, wie die UML-Sichtbarkeiten in Java-Programmcode umgesetzt werden, inklusive der Verwendung von Sichtbar\-keiten im Rahmen abstrakter Klassen und Interfaces (S. 45, 51, 97f., 107-109).}

\sttpKommLitItem{Lahres/Raýman/Strich}{2018}{Objektorientierte Programmierung}{lah18}{Bilder/Buchcover/Buchcover_Lahres_Rayman_Strich.png}{}
{Kopplung und Kohäsion werden im Rahmen der allgemeinen Entwurfsprinzipien der objektorientierten Programmierung explizit nur kurz behandelt (S. 46). Das \mbox{Thema} der Modulgestaltung zieht sich dann aber durch fast alle Kapitel des Buchs, so dass Sie auch im weiteren Verlauf inhaltlich immer wieder auf diese Aspekte treffen. Den Themen Vererbung und Polymorphie (und damit auch Generalisierung und Substituier\-barkeit) ist ein eigenes knapp fünfzigseitiges Kapitel (S. 155ff) gewidmet, das diese Aspekte sehr ausführlich behandelt.} 

\sttpKommLitItem{Ousterhout}{2021}{Prinzipien des Softwaredesigns}{ous21}{}{}
{Das ganze Buch beschäftigt sich sehr programmcodenah mit Prinzipien und Heuristiken zur guten Modulgestaltung. Insofern finden Sie die Aspekte Kopplung und Kohäsion in fast jedem Kapitel.}
	


\cleardoublepage
\chapter*{Einleitung zur Lektion}
\addcontentsline{toc}{chapter}{Einleitung zur Lektion}
\markboth{Einleitung zur Lektion}{Einleitung zur Lektion}

Diese Lektion beschäftigt sich mit dem Kernprozess \textbf{Entwurf} des Software-
\linebreak %%% für Druck
engineering. Auch beim Entwurf geht es wieder um Modellierung. Der Modellierungs\-gegenstand ist nun in erster Linie das zu entwickelnde Softwaresystem. Domänen\-objekte, die zu Objekten im System werden sollen, spielen jedoch weiterhin eine große Rolle. Die Domänenmodellierung als Ganzes ist aber nur noch dann relevant, wenn den Ergebnissen aus dem Requirements Engineering die notwendige Ausdifferenzierung fehlt, um konkrete Klassen für den späteren Programmcode zu spezifizieren, oder wenn sie sich als inkonsistent herausgestellt haben. 

Wichtige Aufgaben im Rahmen des Entwurfs sind die Festlegung der Software\-architektur sowie die Bestimmung der einzelnen Komponenten des Software\-systems und deren Schnittstellen. Eine weitere Aufgabe ist der Entwurf für die Daten\-haltung/""Datenbank. Diese behandeln wir in diesem Modul nicht, hierfür gibt es eigene \mbox{Module}. 

Diese Lektion besteht aus zwei Kapiteln. Kapitel~\ref{sec:Kap-7} behandelt das Thema Software\-architektur und betrachtet damit das zukünftige Softwaresystem auf einer hohen Abstraktions\-ebene. Kapitel~\ref{sec:Kap-8} beschäftigt sich auf einer niedrigen Abstraktions\-ebene mit Klassen und Objekten im Kontext des Entwurfs. Das Thema Entwurfs\-prinzipien, das wir mit in Kapitel~7 platziert haben, verbindet diese beiden sowie die Abstraktionsebenen dazwischen.

\clearpage

\sttpUniversalkasten{Lernziele zu Lektion 5}{Nach dieser Lektion
	\begin{itemize}
		\item wissen Sie, was eine Softwarearchitektur ist und kennen wichtige Faktoren für ihre Auswahl und Ausgestaltung,
		\item können Sie erklären, was Architekturmuster sind und wozu diese gut sind. Sie kennen die Schichtenarchitektur und Client-Server-Architekturen als Beispiele für Architekturmuster und können deren grundsätzliche Funktions\-weise erläutern,
		\item können Sie erklären, was Entwurfsprinzipien sind. Sie kennen die grundsätzlichen Einsatzzwecke der in der Lektion vorgestellten Entwurfs\-prinzipien und die Unterschiede zwischen ihnen,
		\item können Sie erklären, in welchen Zusammenhängen und mit welchen \mbox{Bedeutungen} die Begriffe Komponente und Schnittstelle im Rahmen des Software\-entwurfs verwendet werden,
		\item können Sie mit Elementen des UML-Klassendiagramms Klassen, ihre Eigen\-schaften, ihr Verhalten und ihre Beziehungen mit dem Fokus auf Softwareentwurf modellieren.		
	\end{itemize}
}
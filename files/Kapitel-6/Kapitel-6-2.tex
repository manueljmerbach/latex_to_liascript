\section{Anforderungen}
\markboth{\thechapter~Requirements Engineering}{\thesection~Anforderungen}% Muss explizit gesetzt werden, da nach Kapitel Zoo kaputt.
\label{sec:Kap-6.2}

Das "`Software and Systems Engineering Vocabulary"' (SEVOCAB; siehe Kap.~1.1) % TODO Kap.~\ref{sec:Kap-1.1}
definiert den Begriff Anforderung (engl. requirement) aus verschiedenen Blick-
\linebreak %%% für Druck
winkeln. Aus der Perspektive eines Nutzers -- womit sowohl menschliche Nutzer als auch andere Softwareprodukte gemeint sind -- eines (zukünftigen) Softwareprodukts ist eine Anforderung ein

\sttpzitat{"`statement that translates or expresses a need and its associated constraints and conditions"'.}{\url{https://www.computer.org/sevocab}; Eintrag: requirement (1)}

\vspace{3mm} %%% für Druck

Aus den Bedürfnissen (need) der Nutzer ergeben sich -- sofern sie berücksichtigt werden -- die Funktionalitäten und Eigenschaften des zu entwickelnden Software\-produkts.

SEVOCAB, das Begriffsdefinitionen aus unterschiedlichen Standards gleichwertig nebeneinander stellt, führt für den Begriff requirement noch eine zweite Defini\-tion aus einem Standard zur Qualitätssicherung von Software auf. Danach ist eine Anforderung eine

\sttpzitat{"`condition or capability that must be met or possessed by a system, system component, product, or service to satisfy an agreement, standard, specification, or other formally imposed documents"'.}{\url{https://www.computer.org/sevocab}; Eintrag: requirement (2)}

\vspace{3mm} %%% für Druck

Hier wird der Begriff Anforderung aus Perspektive des zu erstellenden Softwareprodukts definiert. Danach beschreiben Anforderungen, welche Bedingungen oder Fähigkeiten ein System aufweisen bzw. erfüllen muss, um vorliegenden formellen Dokumenten (wie einem Vertrag oder einer Spezifikation) gerecht zu werden. Auf die Unterscheidung zwischen sogenannten Nutzeranforderungen und Systemanforderungen kommen wir in Abschnitt~\ref{sec:Kap-6.2.1} zurück.

\vspace{2mm} %%% für Druck

\minisec{Anforderungsquellen}

Anforderungen an ein zu erstellendes Softwareprodukt können aus unterschiedlichen Quellen stammen. Eine wichtige Kategorie von Quellen sind verschiedene Arten von Dokumenten, wie zum Beispiel Unternehmenshandbücher, die Organisations- und Arbeitsprozesse spezifizieren, die die zu erstellende Software abbilden bzw. unter\-stützen soll. Eine andere Kategorie von Anforderungsquellen sind existierende Softwareprodukte, deren Funktionalitäten in einer aktualisierten Version verbessert werden sollen bzw. Konkurrenzprodukte, deren Funktionalitäten auch im eigenen Softwareprodukt zur Verfügung stehen sollen. Auch Schnittstellenbeschreibungen benachbarter Softwareprodukte können Anforderungen enthalten bzw. generieren. Eine sehr wichtige Quelle von Anforderungen sind zudem Domänenmodelle. Letztendlich sind die Quellen von Anforderungen aber immer die Stakeholder. Und zwar zum einen, weil sie entscheiden, welche der anderen Quellen überhaupt Infor\-mations\-grundlage für die Anforderungsermittlung sein sollen. Zum zweiten und vielleicht auch vor allem, weil sie Bedürfnisse haben, die als Anforderungen an das Softwareprodukt formuliert werden müssen und die sich größtenteils nicht aus anderen (nichtmenschlichen) Quellen extrahieren lassen.
\section{Anforderungen ermitteln und dokumentieren}
\label{sec:Kap-6.3}

Ein nicht zu unterschätzendes Problem in Softwareentwicklungsprojekten ist, wenn Auftraggeber und/oder Entwicklungsteam zu naiv davon ausgehen, dass der Auftrag\-geber genau weiß, was er möchte; dass alle Anforderungen, die er im Vorfeld schon aufgeschrieben hat vollständig und verständlich sind; und dass maximal noch die Bedeutung einiger vorkommender Fachbegriffe geklärt werden muss, bevor mit dem Entwurf für das Softwareprodukt begonnen werden kann. Nachdem seit den 2000er Jahren der Prozess des Requirements Engineering immer stärker in den Vordergrund gerückt ist, ist heute -- auch in sequentiell arbeitenden Software\-entwicklungs\-projekten -- eine ganz strikte Aufgabentrennung „der Kunde erstellt die Anforderungen, das Entwicklungsteam setzt sie um“ nicht mehr der Regelfall. Die Bedeutung des gemeinsam von Auftraggeber und Entwicklungsteam durchgeführten systema\-tischen Requirements Engineering für den Erfolg des Projekts wird aber trotzdem oft noch unterschätzt.

Wie in Lektion~1 % todo Lektion~\ref{sec:Lektion-1}
in Zusammenhang mit den Vorgehensmodellen schon angesprochen, ist die Vertragsgestaltung zwischen Auftraggeber und Entwicklungsprojekt durchführendem Auftragnehmer schwierig, wenn für den Vertragsabschluss keine formell spezifizierten Anforderungssammlungen vorliegen. Und dieser Aspekt steht im Wider\-spruch zu einem gemeinsam von Aufraggeber und Auftragnehmer durchgeführten Requirements Engineering. Unternehmen gehen mit dieser Problematik unterschiedlich um. Manche haben eigene Mitarbeiter oder sogar ganze Abteilungen, die sich auf Requirements Engineering spezialisiert haben und die Fachprojekte des Hauses bei der Erstellung von Anforderungsspezifikationen für Auftragsvergaben unterstützen. Andere Unternehmen führen für die Ermittlung von Anforderungen ein Vorprojekt vor dem eigentlichen Entwicklungsprojekt durch und lassen sich dabei von externen Requirements Engineering-Experten unterstützen. Im agilen Umfeld findet man zudem teilweise auch abweichende Vertragsmodelle, die versuchen, der Problematik von der anderen Richtung aus zu begegnen.

Zur Ermittlung von Anforderungen kann man ganz klassische Techniken aus dem Umfeld der Ist- und Soll-Analysen des Projektmanagements nutzen. Das können zum Beispiel der Einsatz von Fragebögen oder die Durchführung von Interviews sein, in denen die Stake\-holder über ihre Arbeitsabläufe, Organisationsstrukturen, vorhandene Probleme oder Wünsche an das neue Softwareprodukt befragt werden. Eine andere Möglichkeit ist, Beobachtungstechniken einzusetzen, wobei die konkreten Arbeitstätigkeiten der Stake\-holder vor Ort beobachtet, dokumentiert und anschließend gemeinsam analysiert werden, mit dem Ziel herauszufinden, an welchen Stellen das geplante Softwareprodukt unterstützen kann. Weiterhin ist der Einsatz von Brainstorming- und Kreativitätstechniken möglich. Bei letzteren sollte man die Stake\-holder allerdings nicht alleine lassen. Sinnvoll ist hier die gemein\-same Arbeit von Stake\-holdern und Requirements Engineering erfahrenen Personen (unternehmens\-intern oder externe Berater), um auch diesem kreativen Gedankenaustausch eine auf die Softwareproduktentwicklung zielführende Richtung zu geben. Eine Übersicht über konkrete Techniken aus diesen drei Bereichen finden Sie bei \cite[26-33]{poh15}.

\pagebreak %%% für Druck

Andere Techniken, die man zur Ermittlung und Dokumentation von Anforderungen einsetzen kann, entstammen spezifisch dem Softwareengineering-Bereich. Der folgende Abschnitt \ref{sec:Kap-6.3.1} stellt mit dem Anwendungsfalldiagramm eine überwiegend grafisch orientierte Möglichkeit aus dem UML-Umfeld vor.
\section{Anforderungsvalidierung}
\label{sec:Kap-6.4}

Es geht in diesem Teilprozess darum zu analysieren, ob die bislang dokumentierten Anforderungen als Grundlage für die zu entwickelnde Software geeignet sind. \mbox{Darüber} hinaus sind natürlich die Anforderungen zu ergänzen oder zu ändern, wenn dies nicht der Fall ist; wir erhalten also (wie so oft) einen zyklischen Vorgang, der schließlich mit der Prüfung endet, dass die Anforderungen die erforderlichen Qualitäts\-kriterien erfüllen. Manchmal sagt man auch \textit{Anforderungsprüfung} zu diesem Teilprozess; dies vermeiden wir hier, weil es suggeriert, dass im Standardfall diese Prüfung positiv ausfällt und nichts zu tun ist, der Negativfall also die Ausnahme wäre. Dies ist bei der Anforderungsvalidierung nicht so. Eine Überarbeitung der Anforderungen ist der übliche Ausgang und bedeutet auch nicht, dass die Anforderungs\-ermittlung fehlerhaft durchgeführt worden wäre.

Auch der Name \textit{Anforderungsvalidierung} ist nicht genau treffend, weil er zugleich für einen Teil dieses Teilprozesses verwendet wird, wie Sie weiter unten lesen werden. Wir verwenden ihn trotzdem, weil er auch in der Literatur so vorkommt und – im weiteren Sinne – Validierung doch grob das Ergebnis dieses Teilprozesses 
\linebreak %%% für Druck
umschreibt.

Wann ist nun eine Sammlung von Anforderungen geeignet als Grundlage für die Entwicklung eines Softwaresystems? Oder umgekehrt: Was könnten Gründe dafür sein, dass Anforderungen eine zu entwickelnde Software noch unzureichend spezifizieren? Bevor wir uns mit dieser Frage näher befassen, sollten wir das hier von uns betrachtete Szenario genauer beschreiben. Aus der Diskussion über verschiedene Vorgehensmodelle wissen Sie, dass Anforderungen in den meisten Fällen zu Beginn des Softwareentwicklungsprozesses noch nicht in Gänze bekannt sind. Das Wasserfallmodell ignoriert diesen Umstand bzw. berücksichtigt ihn nur notdürftig durch eigentlich nicht vorgesehene Rücksprünge. Agile Vorgehensmodelle gehen sogar davon aus, dass anfangs wenig über die Anforderungen bekannt ist. Wir werden im Folgenden unberücksichtigt lassen, wann die Anforderungen zusammengetragen werden bzw. wie die Anforderungsermittlung mit anderen Teilprozessen verschränkt ist – je nach Vorgehensmodell kann dies sehr unterschiedlich sein. Wir abstrahieren also von allen anderen Teilprozessen bzw. ihren Aktivitäten und konzentrieren uns rein auf die Anforderungssammlung, die ja bei jedem Vorgehen irgendwann das zu erstellende System vollständig spezifiziert (sofern Anforderungen überhaupt gesammelt werden).

Wofür ist es wichtig, dass Anforderungen das zu erstellende Softwaresystem relativ genau beschreiben, also quasi ein Modell dieses Systems darstellen? Der Charakter als Modell ist bei reinem Text weniger deutlich, bei anderen, bereits genannten Spezifikationssprachen wird er deutlicher.

Der wichtigste Grund ist natürlich, dass das System auf Grundlage der Anforderungen erstellt wird. Im Idealfall liegen dem Entwicklerteam nur die Anforderungen vor, und sie haben ihren Job gut gemacht, wenn sie ein System erstellen, dass alle vorgegebenen Anforderungen erfüllt (und dies auch nachweisbar so ist). Natürlich wird es in der Realität meist weitere Anforderungen oder Aspekte geben, die implizit sowohl von den Stakeholdern gemeint waren und auch vom Entwickler berücksichtigt werden, aber nie explizit in den Anforderungen genannt wurden (implizites \mbox{Wissen}). Doch darauf kann man sich nicht verlassen! Diese impliziten Anforderungen basieren oft auf einer vertieften Branchenkenntnis oder auf Detailwissen über ein Vorgängersystem. Einem neuen Entwickler fehlt dieses implizite Wissen und er wird die Erwartungen nicht erfüllen, ohne dass man ihm einen Vorwurf machen 
\linebreak %%% für Druck
könnte.

Ein weiterer Grund ist, dass nur mit einer klaren Vorstellung über das zu erstellende Softwaresystem ein erfahrenes Softwareunternehmen ein realistisches Angebot über Zeit- und Kostenaufwand für die Erstellung des Systems machen kann. Aussagen über Kosten und Lieferzeit sind aber umgekehrt für den Kunden wichtige Informationen, die zu seiner Entscheidung beitragen, ob er die Entwicklung gemäß seiner Visionen und Ziele überhaupt in Auftrag gibt. Oft ist es so, dass in einer Verhandlungsphase die Anforderungen und die resultierenden Entwicklungskosten und -zeiten Gegenstand einer Verhandlung zwischen Auftraggeber und Auftragnehmer sind. Es ist klar, dass dafür aussagekräftige Anforderungssammlungen möglichst früh zur Verfügung stehen müssen, am besten bevor ein Entwicklungsaufwand überhaupt entsteht.

Ein dritter Grund, der aber mit dem zweiten zusammenhängt: Für jedes Software\-entwicklungsprojekt muss der Ressourcenaufwand und ein entsprechender Zeitplan existieren, damit das Softwareunternehmen realistische Versprechungen zu Liefer\-zeiten und -kosten machen kann, aber auch sein -- unterschiedlich spezialisiertes -- Personal durchgehend gewinnbringend einsetzen kann. Grundlage dafür sind zunächst die Anforderungen, die durch die zu entwickelnde Software umgesetzt 
\linebreak %%% für Druck
werden.

Was gibt es nun für Gründe, dass Anforderungen nicht ausreichen oder nicht geeignet sind, ein Softwaresystem zu spezifizieren oder, genauer, das Softwaresystem zu spezifizieren, das die Wünsche des Kunden bzw. seiner Stakeholder tatsächlich erfüllt.
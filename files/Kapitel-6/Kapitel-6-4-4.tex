\subsection{Vollständigkeit}
\label{sec:Kap-6.4.4}

\vspace{-1mm} %%% für Druck

Schließlich müssen die Anforderungen daraufhin überprüft werden, ob sie vollständig sind. Vollständig bezogen auf was? Wenn ein Stakeholder einen Wunsch vergessen hat, wird das außer ihm kaum jemandem auffallen, diese Unvollständigkeit ist hier aber nicht gemeint. Wenn „vollständig“ im Zusammenhang mit „korrekt“ gesehen wird, können in einer unvollständigen Anforderungssammlung Angaben fehlen, die im Sinne der Korrektheit zur Interpretation anderer Anforderungen notwendig sind. Derartige Lücken mögen zwar bei der Verifikation auffallen, dies ist aber nicht notwendigerweise so. Auch nicht spezifiziertes implizites Wissen gehört zu unvollständigen Anforderungssammlungen, und es fällt bei keinem der bislang genannten Schritte zwingend auf. Im Endeffekt ist es notwendig aus dem Blickwinkel eines Entwicklers, der nicht über implizites Wissen verfügt, die Anforderungen zu betrachten und zu verstehen. Auch hier ist die Konstruktion eines Prototyps hilfreich, wenigstens gedanklich. Diesmal jedoch nicht zur Rückkopplung mit den Stakeholdern, sondern zur Prüfung für den Entwickler, ob er allein aufgrund der dokumentierten Anforderungen einen Prototyp erstellen könnte. Wenn die Konstruktion aufgrund fehlender Angaben nicht möglich scheint, ist es wahrscheinlich, dass wichtige Angaben in den Anforderungen fehlen.
\section{Kommentierte Literatur}
\label{sec:Kap-6.6}


\sttpKommLitItem{Sommerville}{2018}{Software Engineering}{som18}{Bilder/Buchcover/Buchcover_Sommerville.jpg}{}
{Kapitel 4 des Softwareengineering-Lehrbuchs beschäftigt sich überblicksartig mit vielen Aspekten des Requirements Engineering. Der Schwerpunkt liegt auf den unter\-schied\-lichen Möglichkeiten Anforderungen zu kategorisieren und den verschiedenen Techniken der Anforderungserhebung.}

\sttpKommLitItem{Broy/Kuhrmann}{2021}{Einführung in die Softwaretechnik}{bro21}{}{}
{Kapitel 5 des Lehrbuchs enthält einen knapp dreißigseitigen, gut strukturierten und verständlichen Überblick zu den wichtigsten Aspekten des Requirements Engineering. Für einen ersten Einblick in das Thema Requirements Engineering ist dieses Kapitel in Verbindung mit dem Requirements Engineering-Kapitel (Kap. 4) von \cite{som18} zu empfehlen, beide Kapitel ergänzen sich in ihren gesetzten Schwerpunkten. Kapitel 6 und 7 des Buchs von Broy/Kuhrmann vertieft auf weiteren 80 Seiten ausgewählte Aspekte und liegt vom Detailgrad zwischen \cite{som18} und \cite{poh15}.}

\sttpKommLitItem{Pohl/Rupp}{2015}{Basiswissen Requirements Engineering}{poh15}{}{}
{Begleitbuch zur Aus- und Weiterbildung zum „Certified Professional for Requirements Engineering“ von IREB (s.~S.~\pageref{sec:Kap-6:IREB}). Das Buch stellt das Thema Require\-ments Engineering sowie dessen Teilprozesse gut strukturiert und überblicksartig dar. Im Unterschied zu Lehrbüchern des Softwareengineering wie \cite{som18} und \cite{bro21} werden in diesem, knapp 150 Seiten umfassenden, rein auf Requirements Engineering fokussierten Buch, auch konkrete (praxisrelevante) Methoden in den Bereichen vorgestellt. Diese werden in der Regel zwar nicht umfassend behandelt, in der ausführ\-lichen Literaturliste des Buchs aber zahlreiche Verweise auf weiterführende \mbox{Literatur} zu diesen Methoden gegeben.}

\sttpKommLitItem{Rupp}{2014}{Requirements-Engineering und -Management}{rup14}{}{}
{Umfangreiche, sehr praxisnahe Darstellung von allen Arbeitsschritten des Requirements Engineering mit Checklisten, Schablonen und Lebensweltbeispielen. Für die konkrete Projektarbeit sehr hilfreich zur Inspiration und zum Nachschlagen verschiedener Requirements Engineering-Methoden. Um einen Überblick über das Thema Requirements Engineering insgesamt zu erhalten, ist der Inhalt zu ausführlich und zu kleinschrittig gestaltet, man verliert leicht den roten Faden. Hierfür eignet sich eher das Buch \cite{poh15}, an dem die Autorin ebenfalls mitgewirkt hat.}

\sttpKommLitItem{Bergsmann/Unterauer}{2018}{Requirements Engineering für die agile Software\-entwicklung}{ber18}{}{}
{Ausführliche und gut strukturierte Vorstellung von zahlreichen agilen Methoden für Requirements Engineering. Das Buch behandelt auch organisatorische und rechtliche Aspekte, unter anderem Aufwandsschätzung, Vertragsgestaltung und Management im agilen Requirements Engineering.}

\sttpKommLitItem{Bourque/Fairley (Hrsg.).}{2014}{SWEBOK V3.0}{swe14}{}{}
{Kapitel 1 des SWEBOK behandelt das Thema Requirements Engineering -- wie in anderen Kapiteln auf einem hohen Abstraktionsgrad. Die Begriffe dieser Lektion (Anforderung, Vision, Produktumfang etc.) kommen dort alle vor, SWEBOK kategorisiert aber anders und deutlich stärker als wir dies getan haben.}

\sttpKommLitItem{Kecher/Salvanos/Hoffmann-Elbern}{2018}{UML~2.5}{kec18}{Bilder/Buchcover/Buchcover_Kecher_Salvanos.jpg}{}
{Das schon aus anderen Lektionen bekannte UML-Buch behandelt in Kapitel 8 die UML-Anwendungsfalldiagramme. Wie immer in diesen Buch werden die Elemente des Diagramms an Lebensweltbeispielen dargestellt, die zudem teilweise schon aus den vorherigen Kapiteln des Buchs zu Objektdiagramm und Klassendiagramm bekannt sind.}
\subsubsection{Ziele}
\label{sec:Kap-6.1.2.2}

Die Darstellung der Vision beinhaltet, welche grundsätzlichen Verbesserungen durch den Einsatz des neuen Softwareprodukts erwartet werden. Die Beschreibung der Ziele konkretisiert diese Darstellung, indem genauer spezifiziert wird, in welchen Bereichen und bei welchen Aspekten, in welchem Umfang eine Verbesserung angestrebt wird. Sinnvoll ist es, nur Sollzustände zu beschreiben anstelle der Wege zur Erreichung dieser Sollzustände. \marginline{lösungsneutrale Ziele} Ziele sollten daher weitgehend \textit{lösungsneutral} formuliert werden, damit mögliche Lösungen durch die Formulierung des Ziels nicht von Beginn an ausgeschlossen werden. Zum Beispiel sollte das Ziel einer gemeinsamen Datenhaltung für die verschiedenen interagierenden Softwareprodukte des Unternehmens nicht durch die Vorgabe der Datenbanktechnologie (\zb relational, dokumenten\-orientiert) unnötig eingeschränkt werden. Es kann aber (gesetzliche, unternehmerische, arbeits\-organisa\-torische etc.) Rahmenbedingungen für den Betrieb des Softwareprodukts geben, die nicht ignoriert werden können und dementsprechend Teil der Ziel\-formulierung sein müssen, auch wenn sie die möglichen Lösungswege zur Erreichung eines Ziels einschränken. Im Beispiel von eben könnte es aus datenschutzrechtlichen oder sicherheitstechnischen Gründen notwendig sein, dass die Datenhaltung innerhalb des Unternehmens erfolgt und damit zum Beispiel externe Cloud-basierte Lösungen für die Erreichung des Ziels von vornherein wegfallen.

Neben der Lösungsneutralität bei gleichzeitiger Berücksichtigung relevanter Rahmen\-bedingungen sollte die Formulierung von Zielen für den Einsatz eines Software\-produkts bestimmten weiteren Qualitätskriterien genügen. Im Rahmen des Require\-ments Engineering werden dafür unterschiedliche Kriterienlisten aufgestellt (\zb \cite[S. 26 und S. 84]{rup14} oder \cite[457 \psqq]{bal09}), die verschiedene Schwer\-punkte setzen, sich im Großen und Ganzen aber sehr ähnlich sind. Verbreitet ist die Anwendung des ursprünglich aus dem Bereich der Mitarbeiterführung stammenden und auch im Projektmanagement bei Projekten außerhalb der IT häufig eingesetzten SMART-Katalogs. Das Akronym SMART 
\marginline{SMART} 
steht für \textbf{\sttpHervorhebung{S}}pecific, \textbf{\sttpHervorhebung{M}}easurable, \textbf{\sttpHervorhebung{A}}chievable, \textbf{\sttpHervorhebung{R}}easonable, \textbf{\sttpHervorhebung{T}}ime-Bound\footnote{Insbesondere für die Buchstaben A und R findet man in der Literatur teilweise auch andere Begriffe mit abweichender Bedeutung. Die inhaltliche Bedeutung des Gesamtakronyms ist aber im Großen und Ganzen unbestritten.}.  Ziele sollten danach folgende Kriterien aufweisen:

\begin{itemize}
	\item \textbf{Specific (spezifisch, gezielt, bestimmt)}: Ziele sollten eindeutig und so präzise wie möglich formuliert sein und vollständig beschrieben werden. Es muss deutlich werden, auf welchen Aspekt (\zb welcher Erfolgsfaktor) sich die angestrebte Verbesserung bezieht. Zum Beispiel wäre „die Relevanz des Zoos erhöhen“ nicht spezifisch genug, da unklar bleibt, an welchen Stellschrauben gedreht werden müsste. Spezifisch formulierte Ziele in diesem Zusammenhang wären zum Beispiel „das jährliche Besucheraufkommen erhöhen“ oder „Ausweitung der Öffentlichkeitsarbeit“. Das Kriterium spezifisch beinhaltet auch, dass ein Ziel für alle Stakeholder verständlich ist. Es kann und es wird Unstimmigkeiten zwischen verschiedenen Stakeholdern über die Priorität eines Ziels geben, es dürfen aber keine Missverständnisse über den Inhalt des entsprechenden Ziels bestehen.

	\vspace{2.3mm} %%% für Druck

	\item \textbf{Measurable (messbar)}: Die Erreichung eines Ziels sollte überprüfbar und im Idealfall der Grad der Zielerreichung messbar sein. Man sollte also angeben können, ab wann das Ziel als erreicht gelten soll. Das setzt voraus, dass die Formulierung des Ziels messbare Parameter enthält und entsprechende Metriken festgelegt werden. Um das Ziel „das jährliche Besucheraufkommen erhöhen“ messbar zu formulieren, benötigt man wenigstens die Angabe des Ausgangspunkts, zum Beispiel „im Vergleich zum letzten Jahr“ oder „im Vergleich zum durchschnittlichen Besucheraufkommen der letzten fünf Jahre“. Idealerweise sollte man in diesem Beispiel auch noch spezifizieren, wie stark das Besucheraufkommen steigen soll („um 10\% im Jahresdurchschnitt“, „um 20 Besucher pro Tag“ etc. ), denn ansonsten würde auch durch einen einzigen Besucher mehr pro Jahr das Ziel erfüllt werden. Und dafür würde sich der personelle und finanzielle Aufwand der Entwicklung des Softwareprodukts sicher nicht lohnen. 
	
	\vspace{2.3mm} %%% für Druck

	\item \textbf{Achievable (erreichbar, realisierbar, umsetzbar)}: Ein Ziel muss mit den vorhandenen Ressourcen, in der zur Verfügung stehenden Zeit (\su Kriterium Time-Bound) und unter den gegebenen Rahmenbedingungen erreichbar sein. Das Kriterium achievable beinhaltet zudem, dass das Ziel mit dem Aufgabenbereich des Softwareprodukts korrespondieren muss. So kann das Ziel „größere Gehege für Zebras und Löwen“ ein Ziel des Zoos sein. Vermutlich würde sich dieses Ziel allerdings nie durch den Einsatz der neuen Zooverwaltungssoftware erreichen lassen -- sondern nur durch Baumaßnahmen --, unabhängig davon wieviel Zeit und Ressourcen für die Softwareentwicklung zur Verfügung stünden.
	
	\vspace{2.3mm} %%% für Druck

	\item \textbf{Reasonable (angemessen, vernünftig, begründet, akzeptiert)}: Ein Ziel sollte von allen an der Zielbestimmung Beteiligten als sinnvoll angesehen und als Ziel der geplanten Softwareentwicklung akzeptiert sein. Die Gruppe der\-jenigen, die die Ziele einer Softwareentwicklung bestimmt, ist in der Regel nur eine Teilmenge der Stakeholder des Softwareentwicklungsprojekts. Auch deshalb ist es unrealistisch, dass alle (späteren) Anforderungssteller im Projekt die ausgegebenen Ziele vollumfänglich mittragen werden bzw. als relevant einschätzen. Aber zumindest unter denjenigen, die die Ziele definiert haben, sollte Einigkeit herrschen, da die Ziele der Maßstab für die Berücksichtigung bzw. Nichtberücksichtigung und Priorisierung der gesammelten Anforderungen sein werden.
	
	\vspace{2.3mm} %%% für Druck

	\item \textbf{Time-Bound (terminierbar, zeitbegrenzt, zeitgebunden)}: Ein Ziel sollte die Angabe beinhalten, zu welchem Zeitpunkt es erreicht sein muss. Es kann Ziele geben, die direkt zum Zeitpunkt des Ersteinsatzes des neuen Software\-produkts erreicht sein sollen (\zb die Ersetzung manueller Arbeitsabläufe durch automatisierte). Andere Ziele können längerfristiger Natur sein, zum Beispiel dass die Besucherzahl des Zoos innerhalb eines Jahrs nach Einsatz der Zooverwaltungssoftware um zehn Prozent zum Vergleich des Vorjahrs zugenommen hat.
\end{itemize}

Ziele \marginline{Beziehungen zwischen Zielen} können unabhängig voneinander sein, in Hierarchien zueinander stehen, sich gegenseitig ergänzen oder auch überschneiden. Nur widerstreitende Ziele versucht man zu vermeiden. Solche sogenannten Zielkonflikte sind allerdings nicht immer auf den ersten Blick ersichtlich, sondern zeigen sich vielleicht erst in der späteren Sammlung konkreter Anforderungen. Üblicherweise gibt es in einem Software\-entwicklungs\-projekt mehrere parallele Ziele sowie auch unterschiedliche Ebenen mit übergeordneten Zielen und daraus abgeleiteten feineren Teilzielen.

Die Kriterien des SMART-Katalogs für Zielformulierungen sind -- wie die Qualitäts\-kriterien für Ziele in anderen Katalogen auch -- Idealvorstellungen. In der Praxis wird es Zielkonflikte geben. Es wird im Softwareentwicklungsprojekt Ziele geben, die nicht messbar formuliert sind, oder Zielformulierungen, die missverständlich sind. Und nicht selten werden Ziele formuliert werden, die mit den gegebenen Ressourcen nicht erreicht werden können. Manchmal wird man auch erst bei der Formulierung oder Priorisierung konkreter Anforderungen merken, dass Ziele noch zu unspezifisch formuliert sind oder sogar Ziele vergessen wurden. Zudem können Ziele auch bewusst verändert werden, zum Beispiel weil sich das Umfeld des Unternehmens verändert hat. 

Die lineare Abfolge 
\marginline{die Bedeutung von Zielen} 
von Zielbestimmung und anschließender Anforderungs\-sammlung wird sich nicht immer einhalten lassen. Man sollte sich diesem Ideal jedoch soweit wie möglich annähern, denn wenn Ziele geändert werden müssen, werden sich auch Anforderungen oder zumindest die Priorität von Anforderungen ändern. Die Definition der Ziele des Produkteinsatzes gehört zu den wichtigsten Aktivitäten im Softwareentwicklungsprozess. Sie dient auch dazu, unter den Stakeholdern ein gemeinsames Verständnis zu etablieren, ab wann das Entwicklungsprojekt als erfolgreich (lohnender Einsatz der investierten Ressourcen) gewertet wird. In der Praxis hat der Auftraggeber meistens vor oder zum Beginn des Projekts schon Ziele formuliert. Häufig ist diese Zielbestimmung aber gleichzeitig nicht vollständig, zu unklar, nicht mit allen relevanten Stakeholdern abgestimmt, im Rahmen des Budgets nicht realisierbar etc., so dass hier zu Beginn des Projekts nochmal Ressourcen investiert werden müssen.

In \marginline{Ziele vs. Anforderungen} einem konkreten Softwareentwicklungsprojekt können die Übergänge zwischen Zielen und Anforderungen fließend sein. Je weiter ein Ziel in immer feinere Teilziele aufgegliedert wird, desto weniger ist ein solches Teilziel von einer Anforderung zu unterscheiden. Insofern sollte Ihnen zwar die konzeptionelle Trennung zwischen Ziel und Anforderung bekannt sein: Ein Ziel ist etwas, das ich \textbf{durch den Einsatz} des Softwareprodukts erreichen will. Eine Anforderung ist etwas, das ich mit dem Softwareprodukt tun will bzw. was das Softwareprodukt können, bereitstellen oder gewährleisten muss. Gleichzeitig sollten Ihnen aber auch die fließenden Grenzen beim praktischen Umgang mit beiden Konzepten bewusst sein.

Ziele entwickeln sich oft anhand vorhandener Unzufriedenheiten mit 
Arbeits- \linebreak %%% für Druck
prozessen oder Softwareprodukten. Um diese Unzufriedenheiten verschiedener Stake\-holder im Rahmen des Requirements Engineering zu erfassen und zu dokumentieren, kann man aus dem Bereich der Ist-Analyse des klassischen Projektmanagements bekannte Beobachtungs- und Befragungsmethoden anwenden (\zb \cite[103 \psqq]{rup14} und \cite[135 \psqq]{som18}). Für neue Produktarten, bei denen man sich kaum an vorhandenen eigenen oder Konkurrenzprodukten orientieren kann, können verschiedene Kreativitäts\-techniken hilfreich sein (\zb \cite[S.~76 und S.~99 \psqq]{rup14} und \cite[58 \psqq]{tre18}). 

Man kann Ziele in unterschiedlicher Weise aufschreiben, zum Beispiel in Prosa, in Form von Schablonen oder Formularen oder auch in Form von Grafiken, die neben den einzelnen Zielen auch Auskunft über die Beziehungen der Ziele geben. 
\subsection{Die Anforderungsspezifikation}
\label{sec:Kap-6.2.4}

Eine Anforderungsspezifikation ist eine systematisch organisierte Sammlung von Anforderungen. Sie bildet die Gesamtheit der Anforderungen an das zu erstellende Softwareprodukt ab. Sie enthält somit sowohl die Nutzeranforderungen als auch die Systemanforderungen, allerdings nicht zwangsläufig in einem Dokument. Wenn der Auftrag zur Systementwicklung an einen externen Auftragnehmer vergeben wird, hat die Anforderungsspezifikation als Teil der Vertrags auch eine formelle Funktion. 

In nach klassischen Vorgehensmodellen organisierten Projekten besteht die Anforderungsspezifikation in der Regel aus dem sogenannten Lastenheft und dem sogenannten Pflichtenheft. \marginline{Lastenheft} Das Lastenheft ist ein strukturiert aufgebautes Dokument, das beschreibt, welche Anforderungen das Softwareprodukt erfüllen soll und welche zusätzlichen Randbedingungen bezüglich des Produkts oder auch des Entwicklungsprozesses gelten müssen. Das Lastenheft wird vom Auftraggeber erstellt (ggf. mit externer Unterstützung) und enthält mindestens Nutzeranforderungen. Sofern es auch schon Systemanforderungen umfasst, werden diese in der Regel durch das Pflichtenheft noch detaillierter spezifiziert. Oft ist das Lastenheft die Grundlage für eine Angebotseinholung bei potenziellen Auftragnehmern. Auftraggeber können sich für die Struktur eines Lastenhefts an vorhandenen Standards von Normierungs\-organisationen oder Vorgaben aus Vorgehensmodellen orientieren. In der Regel enthält ein Lastenheft mindestens folgende thematische Abschnitte:

\begin{itemize}
	\item Einführende Informationen und eine allgemeine Beschreibung des Produkts inklusive der Motivation. Hier würden die Informationen zu Zielen, Systemkontext und Produktumfang Platz finden.
	\item die funktionalen und nichtfunktionalen Anforderungen
	\item produktbezogene Rahmenbindungen, ggf. auch projektbezogene Rahmen-
	\linebreak %%% für Druck
	bedingungen
	\item Glossar der im Lastenheft verwendeten Begriffe der Domäne
	\item eine Übersicht über alle in der Spezifikation referenzierten Dokumente
\end{itemize}

Das Pflichtenheft 
\marginline{Pflichtenheft} 
ist mindestens eine Konkretisierung der Anforderungen aus dem Lastenheft. Darüber hinaus kann es aber sehr unterschiedlich gestaltet sein. So kann es außer den Systemanforderungen auch schon sehr konkrete Umsetzungsbeschreibungen für die Anforderungen aus dem Lastenheft enthalten oder einen groben Überblick über die geplante Systemarchitektur. 

In agilen Projekten wird häufig kein formelles Spezifikationsdokument erstellt. Stattdessen wäre die Anforderungsspezifikation im Prinzip die Sammlung aller User \mbox{Stories} oder die Gesamtheit der Items aller Sprint Backlogs. Oft wird aber die Notwendigkeit einer Gesamtspezifikation grundsätzlich in Frage gestellt und eher auf die zeitlich kürzere Dimension der Iterationen fokussiert. Die in einer Iteration umzusetzenden Anforderungen würden dann die Spezifikation dieser Iteration bilden. Für die Gesamtheit der nichtfunktionalen Anforderungen findet man aber auch in agilen Projekten häufiger strukturierte Spezifikationsdokumente.





\subsection{Systemkontext und Produktumfang}
\label{sec:Kap-6.1.3}

Bei der Bestimmung von Systemkontext und Produktumfang geht es um Fest\-legungen, wofür das zu entwickelnde Softwaresystem zuständig sein soll und welche Verbindungen zu Elementen außerhalb des Systems bestehen, und damit um die Entscheidung, welche Aspekte es beeinflussen kann und welche außerhalb seiner Reichweite liegen werden.

Wir bezeichnen das Endergebnis eines Softwareentwicklungsprojekts in diesem Text als Softwareprodukt. Den Begriff Software\textbf{system} haben wir bisher synonym zu Softwareprodukt verwendet, ohne genauer auf die Definition einzugehen. Anlässlich des Begriffs \textbf{System}kontext in diesem Abschnitt möchten wir an dieser Stelle kurz den Systembegriff des Softwareengineering nachliefern.

\minisec{Der Systembegriff des Softwareengineering}

SEVOCAB definiert ein System als 

\sttpzitat{„combination of interacting elements organized to achieve one or more stated purposes“}{\url{https://www.computer.org/sevocab}; Eintrag: system (1).}

und ein Softwaresystem als 

\sttpzitat{„system for which software is of primary importance to the stakeholders“}{\url{https://www.computer.org/sevocab}; Eintrag: software system (1).}

IREB versteht unter einem System 

\sttpzitat{„a coherent, delimitable set of components that – by coordinated action – achieve some purpose.“ \cite[Eintrag: system]{ire23}}{}

\pagebreak %%% für Druck

Ein System besteht also aus Elementen, die für spezifizierte Zwecke in einer spezifizierten Weise zusammenarbeiten. Ein Systemelement kann dabei auch selbst wieder ein System sein. Die Menge der Elemente des Systems kann abgegrenzt werden gegenüber anderen Elementen, die nicht Teil des Systems sind. In einem \textbf{Software}\-system spielen die \textbf{Software}komponenten des Systems eine entscheidende Rolle. Das bedeutet nicht, dass jedes System, das Software enthält, als Softwaresystem bezeichnet werden würde. Zum Beispiel würde man für einen Kaffeevollautomaten vermutlich nicht den Begriff Softwaresystem wählen, auch wenn er Software\-komponenten enthält, da für die Nutzer die Softwareeigenschaften des Geräts nicht im Vordergrund stehen. Auf der anderen Seite findet sich die Bezeichnung Softwaresystem durchaus auch für Systeme, die neben der im Vordergrund stehenden Software auch Hardwarekomponenten (\zb Sensoren) umfassen. 

Aufgabengebiet des Softwareengineering sind \textbf{Software}systeme. Im Gegensatz \marginline{Produkt vs. System} zum Begriff des Software\textbf{produkts} legt der Begriff des Software\textbf{systems} dabei den Blickwinkel nicht nur auf das Gebilde im Ganzen, sondern auch auf seine Komponenten und deren Zusammenspiel sowie auf die Umgebung des Softwaresystems/-produkts. 

\sttpHinweiskasten{1.0}{erweiterter Systembegriff}{Der Begriff des Systems -- meistens ohne den Zusatz „Software“ -- wird im Softwareengineering auch noch in einer deutlich weiter gefassten Bedeutung verwendet, insbesondere wenn der Blickwinkel auf den Betrieb des Software\-systems ausgerichtet ist. Danach besteht ein System aus den Software\-komponenten, der Dokumentation der Komponenten und weiterer Dokumente, die für den Betrieb benötigt werden, zugehörigen technischen Elementen (\zb Geräte zur Tan-Generierung für Bankensoftware oder Chipkarten\-leser für Authentifizierungen, aber auch die Gesamtheit der Hardware, die für den Betrieb der Software benötigt wird) sowie dem Personal, das das System wartet und dessen betrieblichen Arbeitsprozessen. (\sa \url{https://www.computer.org/sevocab}; Eintrag: system(7)). Wir bleiben in diesem Text bei dem engeren Systembegriff, der synonym zu dem Begriff Softwareprodukt verwendet werden kann.}

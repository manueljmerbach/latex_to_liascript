\subsection{Validität}
\label{sec:Kap-6.4.1}

Wie bereits erwähnt, trägt dieser Aspekt denselben Namen wie der gesamte Teilprozess. Anforderungen sind valide, wenn sie das ausdrücken, was die Stakeholder tatsächlich gemeint haben. Man mag meinen, dass dies doch leicht zu erfüllen ist, denn die Stakeholder haben die Anforderungen ja meist direkt oder indirekt selbst formuliert. Tatsächlich geschehen aber auch dabei Fehler mit potentiell schweren Auswirkungen. Eine vergessene Negation oder eine fehlerhaft numerische Angabe können eine Anforderung stark verändern. Erfahrungsgemäß fallen derartige Fehler den Autoren der Anforderungen selbst nur selten auf. Maßnahmen der Anforderungsvalidierung sind 

\begin{itemize}
\item Gegenlesen der Anforderungen von verschiedenen Stakeholdern.
\item Plausibilitätskontrolle, also oft Einsatz des gesunden Menschenverstands.
\item Übersetzung der Anforderungen in möglichst klare, einfache Formulierungen (bei neuen Formulierungen mag auch der ursprüngliche Autor seinen Fehler leichter erkennen).

\pagebreak %%% für Druck

\item Erstellung eines Prototyps aus Teilen der Anforderungen. Dieser Ansatz wird indirekt bei agilen Verfahren durchgeführt, doch ist hier der Prototyp meist bereits der Kern des zukünftigen Softwaresystems. Ein Prototyp im eigentlichen Sinn ist aber ein Wegwerfprodukt, das hier nur der Anforderungsvalidierung dient, und meist nur die Benutzungsschnittstelle des Systems simuliert.
\item Überprüfung, ob Anforderungen sich gegenseitig widersprechen, siehe weiter unten bei „Konsistenz“.
\end{itemize}
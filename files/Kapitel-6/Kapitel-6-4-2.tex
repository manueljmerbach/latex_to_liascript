\subsection{Konsistenz}
\label{sec:Kap-6.4.2}

Es gibt in den meisten Fällen mehrere Stakeholder mit verschiedenen Zielen und entsprechend verschiedenen Anforderungen an ein Softwaresystem. Ein Software\-system, das all diese Anforderungen erfüllt, sollte für alle Stakeholder bezüglich ihrer Ziele zufriedenstellend sein. Dies ist aber leider nicht in jedem Fall möglich, denn Stake\-holder können recht unterschiedliche Vorstellungen von dem zu entwickelnden Softwaresystem haben. Dies einerseits deshalb, weil sie sich halt verschiedene Vorstellungen gemacht haben. Oft aber ist der Grund, dass sie auch verschiedene Interessen haben und die Formulierung der Anforderungen bewusst oder unbewusst interessengeleitet ist. Als Resultat entsteht eine in sich widersprüchliche Anforderungssammlung, kein Softwaresystem kann alle Anforderungen erfüllen. Als Grundlage für den Systementwurf ist eine nicht konsistente Anforderungssammlung untauglich, denn der Entwickler müsste entscheiden, welche Anforderungen erfüllt werden sollen und welche nicht, und dafür fehlt ihm jegliche Grundlage.

Man mag einwenden, dass sich alle Beteiligten doch vorher geeinigt haben sollen, um widersprüchliche Angaben zu vermeiden. Gerade dafür sind ja auch die Vision und die Formulierung der spezifischeren Ziele da. Trotzdem ist für eine „Einigung“ die explizite und genaue Formulierung von Anforderungen Voraussetzung, und eine Vermeidung jeglicher Inkonsistenz zwischen Stakeholdern würde eine zuvor durchgeführte Anforderungssammlung erfordern – es wäre also nichts gewonnen.

Inkonsistenzen sind nicht sehr leicht zu erkennen, denn verschiedene Stakeholder verwenden oft unterschiedliche Begriffe für dieselbe Sache und formulieren ihre Anforderungen aus recht unterschiedlichen Perspektiven. Zur Identifikation von inkonsistenten Anforderungen ist es deshalb notwendig, zunächst Anforderungen so umzuformulieren, dass Gleiches gleich genannt wird (Elimination von Synonymen), Ungleiches ungleich genannt wird (Elimination von Homonymen) und die Abstraktionsebene der Anforderungsformulierungen angeglichen wird. Anschließend fallen inkonsistente Anforderungen dann gleich auf, wenn sie sich im Sinne der Logik widersprechen. Dies ist nicht immer der Fall. Auch hier ist die Verwendung eines Prototyps sinnvoll, dem Anforderungen gegenübergestellt werden. Zu widersprüchlichen Anforderungen kann es auch keinen passenden Prototypen geben.

Widersprüche in Anforderungen stellen nicht notwendigerweise widersprüchliche Vorstellungen dar. Sie können auch Hinweise darauf geben, dass einzelne Anforderungen nicht das Gemeinte ausdrücken, also nicht valide sind (\so). In diesem Fall ist die Inkonsistenz geradezu als Glücksfall anzusehen, denn sie hilft, invalide Anforderungsformulierungen zu identifizieren.
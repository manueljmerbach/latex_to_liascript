\cleardoublepage
\chapter*{Einleitung zur Lektion}
\addcontentsline{toc}{chapter}{Einleitung zur Lektion}
\markboth{Einleitung zur Lektion}{Einleitung zur Lektion}

Sie haben in Lektion~1 %~\ref{sec:Lektion-1}
gelernt, \marginline{das letzte Mal}
dass sich ein Softwareentwicklungsprozess aus einzelnen Prozessen (wie Anforderungsermittlung oder Implementierung) zusammensetzt, in deren Rahmen verschiedene Aktivitäten ausgeführt werden. Sie wissen zudem, dass man diesen Entwicklungsprozess auf unterschiedliche Arten strukturieren kann, indem man Vorgehensmodelle einsetzt.
 
In  dieser Lektion 
\marginline{dieses Mal}
wird wieder der Begriff \textbf{Modell} fallen. Wir werden uns mit dem Modellbegriff im Softwareengineering, mit der Realweltorientierung als zentraler Idee der Objektorientierung und darauf aufbauend mit der \textit{objektorientierten Modellierung} beschäftigen. Letztere zeichnet sich dadurch aus, dass objektorientierte Prinzipien durchgängig in den Modellierungsprozessen der Anforderungsermittlung und -analyse und des Entwurfs eingesetzt werden und nicht nur bei der Implementierung eine objektorientierte Programmiersprache verwendet wird. Auf dieser Grundlage betrachten wir die Modellierung von Realweltzusammenhängen – in der Terminologie des objektorientierten Softwareengineering als \textit{Domänenmodellierung} bezeichnet. Wir zeigen, wie sich mithilfe der Modellierungssprache UML Objekte der Realwelt, ihre Eigenschaften und Beziehungen modellieren lassen und wie die für die Objekt\-orientierung wichtigen Klassen damit in Zusammenhang stehen. Diese Lektion ist im Unterschied zu späteren noch nicht einem spezifischen Prozess des Software\-engineering gewidmet, da die Themenbereiche Objektorientierung und Modellierung alle Prozesse betreffen. Im Zuge der Domänenmodellierung werden Sie aber schon auf Verbindungen zum Prozess der Anforderungsermittlung und -analyse treffen. Wir fokussieren unseren Blick in dieser Lektion auf die Modellierung von Realweltstrukturen. Dafür benötigen wir nur wenige basale Konzepte. Die weiteren, sehr mächtigen Konzepte der Objektorientierung, die die UML auch abbilden kann, spielen für Realweltmodellierungszwecke kaum eine oder gar keine Rolle. Sie werden manche von ihnen im weiteren Verlauf des Lerntextes kennenlernen.

Schlagen 
\marginline{Bezug zu den Vorgehens\-modellen}
wir zum Schluss dieser Einleitung noch kurz den Bogen zurück zu den Vorgehensmodellen aus der letzten Lektion. Wir haben in der aktuellen Lektion – und stärker noch in den späteren Lektionen – die Situation, dass manche Aspekte des Softwareengineering, die wir darstellen, in einigen Vorgehensmodellen sehr wichtig sind und in anderen gar keine Berücksichtigung finden. Den starken Fokus auf die objektorientierte Modellierung mithilfe der UML zum Beispiel, den wir hier setzen, findet man in agilen Softwareentwicklungsprojekten häufig so nicht. In Projekten, die nach wasserfallartigem Vorgehen arbeiten, trifft man dagegen nicht selten auf den Fall, dass zwar eine objektorientierte Programmiersprache eingesetzt wird, das zentrale Konzept der Realweltorientierung der Objektorientierung (und damit auch die Domänenmodellierung) aber in den der Implementierung vorgeschalteten Prozessen vernachlässigt wird. Wir versuchen bei Aspekten, in denen sich verschiedene Arten von Vorgehensmodellen sehr stark unterscheiden, diese Problematik explizit zu machen. Sie sollten sich aber insgesamt bewusst sein, dass nicht alle Methoden, die Sie hier kennenlernen, in jedem praktischen Softwareentwicklungsprojekt eingesetzt werden. Und das hat ausnahmsweise mal nicht nur damit zu tun, dass die universitäre Ausbildung teilweise andere Schwerpunkte setzt als man sie in betrieblichen Ausbildungs- und Arbeitszusammenhängen findet.

\sttpUniversalkasten{Lernziele zu Lektion 2}{Nach dieser Lektion
	\begin{itemize}
		\item können Sie erklären, was ein Modell im Softwareengineering ist und was in diesem Zusammenhang mit dem Begriff Abstraktion gemeint ist,
		\item besitzen Sie einen ersten Einblick in die Domänenmodellierung und ihre Anwendung in unterschiedlichen Vorgehensmodellen,
		\item können Sie die Begriffe Objekt und Klasse voneinander abgrenzen und die Bedeutung von Multiplizitäten erklären,
		\item können Sie mit UML-Elementen Objekte, Klassen, ihre Eigenschaften und Beziehungen unter dem Fokus der Realweltabbildung modellieren.	
	\end{itemize}
}

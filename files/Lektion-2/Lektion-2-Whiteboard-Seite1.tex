\newgeometry{left=25mm, right=15mm, top=20mm, bottom=20mm,
	marginparwidth=5mm, marginparsep=1mm}
%TODO @Maren: Falls diese Seite wieder eine Kopfzeile haben soll, muss die folgende Zeile - \thispagestyle{empty} - auskommenitert werden.
\thispagestyle{empty}

\phantomsection
\label{text:lektion2_whiteboard_s1}
\minisec{\hspace{1cm}Whiteboard mit Karteikarten}

%\begin{figure}[h!]

{ %% Scope beginnen (damit \renewcommand keine Auswirkung auf andere Karteikarten hat)
	\setlength{\unitlength}{1mm}
	\renewcommand{\sttpKarteikarteSkalierungsfaktor}{0.85}
	
	\begin{picture}(0,0)
		\put(-35,-44){
			\renewcommand{\sttpKarteikarteRotierungswinkel}{0}
			% Karteikarte Fallbeispiel Zoo

\sttpKarteikarte{8cm}{\sttpKarteikarteSkalierungsfaktor}{\sttpKarteikarteRotierungswinkel}{Frage zur Domäne}%
	{Thema:}{Tierärzte}%
	{Details:}{Müssen Tierärzte spezialisiert sein auf Tierarten oder kann jeder Tierarzt jede Tierart behandeln?}%
	{Paul (Softwareteam)}

		}
		\put(48,-45){
			\renewcommand{\sttpKarteikarteRotierungswinkel}{-4}
			% Karteikarte Fallbeispiel Zoo

\sttpKarteikarte{10cm}{\sttpKarteikarteSkalierungsfaktor}{\sttpKarteikarteRotierungswinkel}{Überlegungen zum Produkt}%
	{Thema:}{Standardsoftware}%
	{Details:}{Gibt es eigentlich schon Standard-Zoosoftware für bestimmte Aufgaben, die man integrieren könnte? Wie machen andere Zoos das? Wir sollten soweit wie möglich auf Standardkomponenten setzen.}%
	{Joris (Softwareteam)}

		}
		\put(12,-116){
			\renewcommand{\sttpKarteikarteRotierungswinkel}{-2}
			% Karteikarte Fallbeispiel Zoo

\sttpKarteikarte{16cm}{\sttpKarteikarteSkalierungsfaktor}{\sttpKarteikarteRotierungswinkel}{Information zur Domäne}%
	{Thema:}{Tierpfleger}%
	{Details:}{Der Zoo beschäftigt aktuell 15 Tierpfleger. Die Tierpfleger sind für die Versorgung der Tiere zuständig, das bedeutet in erster Linie, dass sie die Tiere füttern, die Gehege säubern sowie das Futter für die Tiere besorgen. Zudem setzen sie sich in Kontakt mit der Tierärztin, wenn es den Tieren nicht gut geht. Die Futterbesorgung ist der aufwändigste Prozess bei den Aufgaben der Tierpfleger, hier ist sehr viel bürokratische Arbeit notwendig. Urlaubsvertretungen zu organisieren ist auch ein großes Problem, die Tierpfleger der Nagetiere und der Fische können sich gut noch gegenseitig vertreten, bei den Tierpflegern der Geier und der Affen geht es noch mit Einschränkungen, aber der Tierpfleger der Löwen kann natürlich nicht die Tiger versorgen oder umgekehrt. Daher haben wir diese Stellen immer doppelt besetzt.}%
	{Helmut (Team Zoo, oberster Tierpfleger)}

		}
		\put(-41,-170){
			\renewcommand{\sttpKarteikarteRotierungswinkel}{2}
			% Diese Karteikarte kann als Vorlage dienen

\sttpKarteikarte{8cm}{\sttpKarteikarteSkalierungsfaktor}{\sttpKarteikarteRotierungswinkel}{Frage zur Domäne}%
{Thema:}{$\sim$$\sim$ $\sim$$\sim$$\sim$$\sim$$\sim$}%
{Details:}{$\sim$$\sim$$\sim$$\sim$ $\sim$$\sim$$\sim$$\sim$$\sim$$\sim$, $\sim$$\sim$$\sim$$\sim$$\sim$$\sim$$\sim$ $\sim$$\sim$ $\sim$$\sim$$\sim$$\sim$$\sim$$\sim$$\sim$$\sim$ $\sim$$\sim$$\sim$$\sim$$\sim$, $\sim$$\sim$ $\sim$$\sim$$\sim$$\sim$$\sim$.}
{Joris (Softwareteam)}
		}
		\put(42,-177){
			\renewcommand{\sttpKarteikarteRotierungswinkel}{-1}
			% Karteikarte Fallbeispiel Zoo

\sttpKarteikarte{11cm}{\sttpKarteikarteSkalierungsfaktor}{\sttpKarteikarteRotierungswinkel}{Frage zur Domäne}%
	{Thema:}{Käfige}%
	{Details:}{Was unterscheidet Gehege und Käfige? \newline Gibt es nur bestimmte Tierarten, die in Käfigen wohnen? \newline Werden Käfige immer ausgeliehen oder besitzt der Zoo auch eigene? \newline Sind Käfige auch zukünftig Teil der Domäne?}%
	{Paul (Softwareteam)}
		}
		\put(64,-232){
			\renewcommand{\sttpKarteikarteRotierungswinkel}{-3}
			% Diese Karteikarte kann als Vorlage dienen

\sttpKarteikarte{6.5cm}{\sttpKarteikarteSkalierungsfaktor}{\sttpKarteikarteRotierungswinkel}{Information zur Domäne}%
	{Thema:}{$\sim$$\sim$$\sim$$\sim$$\sim$ $\sim$$\sim$$\sim$ $\sim$$\sim$$\sim$$\sim$}%
	{Details:}{$\sim$$\sim$$\sim$$\sim$ $\sim$$\sim$$\sim$$\sim$$\sim$ $\sim$$\sim$$\sim$$\sim\sim\sim$ $\sim\sim\sim\sim$ $\sim$$\sim$ $\sim$$\sim$$\sim$$\sim$$\sim$ $\sim$$\sim$ $\sim$$\sim$$\sim$ $\sim$$\sim$$\sim$$\sim$$\sim$$\sim$$\sim$ $\sim$$\sim$.}
	{Ruth (Team Zoo, Fischpflegerin)}
		}
		\put(-20,-247){
			\renewcommand{\sttpKarteikarteRotierungswinkel}{1}
			% Karteikarte Fallbeispiel Zoo

\sttpKarteikarte{12.5cm}{\sttpKarteikarteSkalierungsfaktor}{\sttpKarteikarteRotierungswinkel}{Überlegungen zum Produkt}%
	{Thema:}{Prioritäten}%
	{Details:}{Am wichtigsten: Umfangreiche Informationen auf der Website zu unseren Tieren und eine Softwareunterstützung beim Futterbestellprozess. \newline Zweite Priorität: die Einbindung der Besucher bei der Namensgebung neugeborener Tiere, ein automatisiertes Patenschaftssystem und der Online-Eintrittskartenverkauf. \newline Weitere Wünsche: Terminverwaltung Tierärzte, komfortableres System zur Tier- und Käfigausleihe.}%
	{Fr. Walther (Zoodirektorin)}
	
	
	

		}
	\end{picture}
} %% Scope beenden (um \renewcommand lokal zu halten)

%\caption{Whiteboard}
%\label{fig:lektion2_whiteboard_s1}
%\end{figure}

\restoregeometry
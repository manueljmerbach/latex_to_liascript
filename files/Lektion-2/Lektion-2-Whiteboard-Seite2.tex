\newgeometry{left=25mm, right=15mm, top=20mm, bottom=20mm,
	marginparwidth=5mm, marginparsep=1mm}
%TODO @Maren: Falls diese Seite wieder eine Kopfzeile haben soll, muss die folgende Zeile - \thispagestyle{empty} - auskommenitert werden.
\thispagestyle{empty}

\phantomsection
\label{text:lektion2_whiteboard_s2}
% Keine Überschrift hier, aber selben Platz verbrauchen ;-)
\minisec{~~~}

%\begin{figure}[h!]

{ %% Scope beginnen (damit \renewcommand keine Auswirkung auf andere Karteikarten hat)
	\setlength{\unitlength}{1mm}
	\renewcommand{\sttpKarteikarteSkalierungsfaktor}{0.85}
	
	\begin{picture}(0,0)
		\put(-30,-54){
			\renewcommand{\sttpKarteikarteRotierungswinkel}{2}
			% Karteikarte Fallbeispiel Zoo

\sttpKarteikarte{10.5cm}{\sttpKarteikarteSkalierungsfaktor}{\sttpKarteikarteRotierungswinkel}{Information zur Domäne}%
	{Thema:}{Aufgaben der Tierärzte}%
	{Details:}{Ein Zootierarzt hat drei Aufgaben. Er behandelt und heilt (idealer\-weise) Tiere bei Krankheiten. Er impft die Tiere regelmäßig. Und er führt die Jahresuntersuchungen durch. Für diese Jahres\-untersuchung eines Tiers müssen immer zwei Tierärzte zur Verfügung stehen, das kann man nicht alleine machen.}%
	{Wiebke (Team Zoo, Tierärztin)}

		}
		\put(58,-55){
			\renewcommand{\sttpKarteikarteRotierungswinkel}{-2}
			% Diese Karteikarte kann als Vorlage dienen

\sttpKarteikarte{7.5cm}{\sttpKarteikarteSkalierungsfaktor}{\sttpKarteikarteRotierungswinkel}{Überlegungen zum Produkt}%
	{Thema:}{$\sim$$\sim$ $\sim$$\sim$$\sim$$\sim$ $\sim$$\sim$$\sim$$\sim$}%
	{Details:}{$\sim$$\sim$ $\sim$$\sim$$\sim$$\sim$$\sim$$\sim$$\sim$ $\sim$$\sim$$\sim$ $\sim$$\sim$$\sim$$\sim$ $\sim$$\sim$$\sim$ $\sim$$\sim$$\sim$$\sim$$\sim$$\sim$. $\sim$$\sim$$\sim$ $\sim$$\sim$$\sim$$\sim$$\sim$ $\sim$$\sim$$\sim$$\sim$, $\sim$$\sim$$\sim$$\sim$$\sim$$\sim$$\sim$ $\sim$$\sim$$\sim$$\sim$ $\sim$$\sim$$\sim$$\sim$. $\sim$$\sim$$\sim$ $\sim$$\sim$$\sim$$\sim$$\sim$$\sim$$\sim$ $\sim$$\sim$$\sim$ $\sim$$\sim$$\sim$ $\sim$$\sim$$\sim$$\sim$$\sim$ $\sim$$\sim$$\sim$ $\sim$$\sim$$\sim$$\sim$$\sim$$\sim$ $\sim$$\sim$ $\sim$$\sim$ $\sim$$\sim$$\sim$ $\sim$$\sim$$\sim$$\sim$$\sim$. $\sim$$\sim$$\sim$$\sim$$\sim$$\sim$$\sim$$\sim$ $\sim$$\sim$$\sim$ $\sim$$\sim$$\sim$$\sim$$\sim$ $\sim$$\sim$$\sim$$\sim\sim\sim$.}
	{Lars (Team Zoo, Marketing)}
		}
		\put(32,-123){
			\renewcommand{\sttpKarteikarteRotierungswinkel}{-3}
			% Karteikarte Fallbeispiel Zoo

\sttpKarteikarte{11.5cm}{\sttpKarteikarteSkalierungsfaktor}{\sttpKarteikarteRotierungswinkel}{Frage zur Domäne / Überlegungen zum Produkt}%
	{Thema:}{Patenschaften}%
	{Details:}{Können Kinder Patenschaften für mehrere Tiere übernehmen? \newline Wie ist der Prozess der Patenschaftsübernahme aktuell organisiert? \newline Sollen Patenschaften für alle Tierarten möglich sein? \newline Wie soll der Prozess idealerweise ablaufen? \newline Unterscheidet sich der Prozess für unterschiedliche Tierarten? \newline Der Vogelpark Hohenhausen verwendet ein Softwaresystem zur Organisation von Patenschaften. Prüfen, ob das für uns in Frage kommt.}%
	{Paul (Softwareteam)}

		}
		\put(10,-180){
			\renewcommand{\sttpKarteikarteRotierungswinkel}{1}
			% Karteikarte Fallbeispiel Zoo

\sttpKarteikarte{13cm}{\sttpKarteikarteSkalierungsfaktor}{\sttpKarteikarteRotierungswinkel}{Frage zur Domäne}%
	{Thema:}{Gehege}%
	{Details:}{Wie definiert sich ein Gehege? Ist jeder umzäunte Bereich im Zoo ein (potenzielles) Gehege? Wie viele Tiere können in einem Gehege leben?}%
	{Magnus (Softwareteam)}

		}
		\put(-28,-240){
			\renewcommand{\sttpKarteikarteRotierungswinkel}{3}
			% Karteikarte Fallbeispiel Zoo

\sttpKarteikarte{9.5cm}{\sttpKarteikarteSkalierungsfaktor}{\sttpKarteikarteRotierungswinkel}{Überlegungen zum Produkt}%
	{Thema:}{Informationen zu Tieren auf Website}%
	{Details:}{Unterscheiden sich die Informationen, die pro Tier auf der Website angezeigt werden sollen? Oder ist es spezifisch pro Tierart oder ist der Aufbau immer gleich?}%
	{Joris (Softwareteam)}

		}
		\put(-45,-120){
			\renewcommand{\sttpKarteikarteRotierungswinkel}{1}
			% Diese Karteikarte kann als Vorlage dienen

\sttpKarteikarte{5.5cm}{\sttpKarteikarteSkalierungsfaktor}{\sttpKarteikarteRotierungswinkel}{Frage zur Domäne}%
	{Thema:}{$\sim$$\sim$$\sim$$\sim$ $\sim$$\sim$ $\sim$$\sim$$\sim$}%
	{Details:}{$\sim\sim\sim$ $\sim\sim\sim\sim$ $\sim$$\sim$ $\sim$$\sim$$\sim$$\sim$$\sim$ $\sim$$\sim$ $\sim$$\sim$$\sim$ $\sim$$\sim$$\sim$$\sim$$\sim$$\sim$$\sim$ $\sim$$\sim$$\sim$$\sim$$\sim$ $\sim$$\sim$$\sim$.}
	{Magnus (Softwareteam)}
		}
		\put(57,-242){
			\renewcommand{\sttpKarteikarteRotierungswinkel}{-2}
			% Karteikarte Fallbeispiel Zoo

\sttpKarteikarte{7cm}{\sttpKarteikarteSkalierungsfaktor}{\sttpKarteikarteRotierungswinkel}{Frage zur Domäne}%
	{Thema:}{Tierpfleger}%
	{Details:}{Wer kann wen vertreten? \newline Gibt es Hierarchien bei den Tierpflegern? \newline Könnten Tierpfleger sich bei der Futter\-bestellung zusammen tun?}%
	{Paul (Softwareteam)}

		}
		%		\put(-20,-247){
			%			\renewcommand{\sttpKarteikarteRotierungswinkel}{1}
			%			\input{Karteikarten/xxx.tex}
			%		}
	\end{picture}
} %% Scope beenden (um \renewcommand lokal zu halten)

%\caption{Whiteboard}
%\label{fig:lektion2_whiteboard_s2}
%\end{figure}

\restoregeometry
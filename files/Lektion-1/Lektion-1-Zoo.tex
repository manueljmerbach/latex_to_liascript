\cleardoublepage
% Kapitel Fallbeispiel Zoo: über das Marko werden Kapitel, Eintrag Inhaltsverzeichnis und die Kopfzeilen konfiguriert.
\FallBeispielZoo
\label{sec:Lektion-1-Zoo}

Wir möchten hier am Ende dieser ersten Lektion mit einem Fallbeispiel beginnen, das Sie in den weiteren Lektionen immer mal wieder begleiten wird. Es dient dazu, aus einem praktischeren Blickwinkel ein paar Schlaglichter auf typische Tätigkeiten im Softwareengineering zu werfen. Die Simulation eines kompletten Software\-entwick\-lungs\-prozesses anhand eines Fallbeispiels ist im Rahmen einer solchen Lehrveranstaltung allerdings nicht möglich. Für das Fallbeispiel haben wir den Real\-weltbereich des Zoos gewählt und entschuldigen uns schon einmal im Vorfeld bei allen Zoo-Spezialistinnen und Zoo-Spezialisten für die Halbwahrheiten und erfundenen Informationen zur Lebenswelt Zoo in diesem natürlich für unsere Lehrzwecke so konstruierten Fallbeispiel.

\minisec{Die Ausgangslage}
Sie haben kürzlich Ihr Informatik- (wahlweise Wirtschaftsinformatik-) Studium abgeschlossen und Ihre erste Stelle in einem Unternehmen angetreten, das Softwarelösungen für kleine und mittelständische Betriebe entwickelt. Ihr neuer Chef, Herr Steiber, informiert Sie und Ihre Kolleg:innen über ein neues Projekt.

\textbf{Hr. Steiber:} „Was wissen Sie über Zoos? Wenig? Geht mir auch so. Aber das wird sich für uns alle wohl bald schon ändern: Einige von Ihnen kennen Frau Dr. Walther ja schon, die Direktorin des Zoos hier in der Stadt – und eine gute Bekannte von mir. Frau Dr. Walther ist vor einiger Zeit auf mich zugekommen, da der Zoo eine Zoo-Verwaltungssoftware benötigt. Nur, was genau gebraucht wird, in welchem Umfang, für welche Aufgaben usw. das scheint offensichtlich noch recht unklar zu sein. Frau Dr. Walther und einige ihrer Kolleginnen und Kollegen werden nächste Woche zu einem ersten Informationsgespräch bei uns vorbei kommen und ich möchte das gesamte Team bitten, dabei zu sein. Ich kann noch gar nicht einschätzen, in welchem Rahmen sich ein eventuelles Projekt bewegen würde und insofern müssen wir bei diesem ersten Gespräch auch herausfinden, inwiefern es sich für uns überhaupt lohnen würde, weitere Ressourcen zu investieren und auf eine Auftragserteilung hinzuarbeiten. Das zu lösende Problem darf einerseits natürlich nicht zu klein sein – ich sag mal, nur Eingabemasken zu erstellen, in der die Namen der Zootiere eingeben werden können, ist für uns nicht interessant. Auf der anderen Seite können wir mit unserem doch recht kleinen Team natürlich auch keine allumfassende Automatisierung sämtlicher Zooabläufe liefern - Ich fürchte ein wenig, dass es genau das ist, was der Zoo sich vorstellt. Wir werden sehen!“

\minisec{Beim ersten Treffen mit dem Zoo-Team}

\textbf{Fr. Walther (Zoodirektorin):} „Hallo. Schön, dass wir heute hier bei Ihnen sein dürfen. Herr Steiber hatte mich gebeten, etwas zu unserem Zoo zu sagen. Das ist kein Problem! Außerdem soll ich ja schon mal skizzieren, was die von uns gewünschte Zoo-Software können muss. Aber woher soll ich das wissen? Sie sind doch die Experten in der Softwareentwicklung. Sie wissen viel besser, was Softwareprodukte können müssen. Also, unser Zoo: Wir haben jährlich ca. 150.000 Besucher, im Jahresverlauf ist die Besucherzahl stark schwankend. 80 Prozent unserer Besucher sind Kinder. Natürlich hat irgendwie jedes Kind sein eigenes Lieblingstier, was sie aber alle auch mögen, ist unser Streichelzoo. Da haben wir nicht nur die üblichen Kaninchen, Meerschweinchen, Ziegen und Schafe, sondern auch unser Warzenschwein Rudi. Rudi war das erste Tier, für das wir vor ein paar Jahren die Möglichkeit eingerichtet haben, für einen jeweils begrenzten Zeitraum eine Patenschaft zu übernehmen. Mittlerweile existieren weitere Patenschaften für vier unserer Schlangen, unsere zwei Axolotl und den Kormoran. Eigentlich möchten wir das Patenschaftssystem auf alle unsere Tiere ausweiten und auch die Möglichkeit unterschiedlich langer Patenschaften bieten, aber ohne technische Unterstützung ist der Verwaltungsaufwand zu hoch. Kinder, die eine Patenschaft übernommen haben, erhalten für diesen Zeitraum freien Eintritt in den Zoo und können an bestimmten Tagen beim Füttern ihres Patentiers helfen. Wir beschäftigen fünfzehn Tierpflegerinnen und Tierpfleger und einen Tierarzt als Angestellte. Oft reicht ein Tierarzt nicht aus, sodass wir in diesen Fällen externe Tierärzte beauftragen müssen, was so kurzfristig oft nicht immer einfach ist. Wir planen langfristig eine zweite feste Tierarztstelle im Zoo einzurichten, vor allem da wir aktuell unseren Tierbestand erweitern. Bisher haben wir 26 verschiedene Tierarten mit entsprechend vielen Unterarten, \zb haben wir drei verschiedene Affenarten im Zoo. Natürlich hat jedes Tier einen Namen, und wir versuchen immer die Besucher einzubinden, wenn ein neugeborenes Tier einen Namen benötigt. Da kommt dann auch mal Shaggy raus als Name für unsere gestern geborene Testudo Graeca. Wir würden gerne alle Informationen zu unseren Tieren – also wann sie geboren sind, wer die Eltern sind, wie alt sie werden können, was sie fressen, aus welchem Zoo sie gegebenenfalls ausgeliehen sind und wie lange und ganz vieles mehr – unseren Besuchern auch über das Internet zur Verfügung stellen, aber das ist sicher aufwändig. Andererseits könnte man dann vielleicht die Eintrittskarten auch gleich online verkaufen, vielleicht würden wir dann noch mehr Besucher bekommen. Wir nutzen schon ein Onlinesystem, wenn wir Käfige oder Tiere aus anderen Zoos für eine bestimmte Zeit ausleihen und eine Software, mit der wir die Dienstpläne und Aufgaben unserer Tierpfleger verwalten, das wär schon gut, wenn das alles ein System wäre. Wir möchten langfristig nur noch fest installierte Gehege für die Tiere haben, in denen sie auch genug Platz und artgerechte Bedingungen haben, aber im Moment müssen wir auch Tiere in Käfigen halten, da sich die Bauarbeiten für die Zoo-Erweiterung verzögern. Aber wenn das mal fertig ist, werden wir endlich auch Eisbären und Nashörner haben, da fragen die Besucher nach. Seit letztem Jahr haben wir zwei Giraffen, da erhoffen wir uns dieses Jahr Nachwuchs. Tja, was müssen Sie noch wissen über unseren Zoo? Ja, genau: wir haben auch mehrere Spielplätze und natürlich Eis- und Pizzastände. Die Essensstände haben wir verpachtet. Das läuft ganz gut. Um das Essen für die Tiere kümmern sich die jeweiligen Tier\-pfleger, die kaufen entsprechend ein. Das müsste man sich auch mal angucken, ob es nicht sinnvoller wäre, die Futterbestellungen zentral zu machen, das müsste doch mit Softwareunterstützung auch viel leichter zu realisieren sein, und dann könnten wir vielleicht auch gleich noch für die Besucher im Internet Informationen über die Tierpfleger zur Verfügung stellen. Gerade unsere Stammgäste mit den Jahreskarten finden es wichtig den Tierpfleger zu kennen, der ihr Lieblingstier versorgt. Was meinen Sie, wie lange brauchen Sie für unsere Zoo-Software?“
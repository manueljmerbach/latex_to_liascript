% Anmerkung zur Schriftgröße:
% Voreingestellt ist bei scrbook, scrreprt und scrartcl fontsize=11pt. (s. scrguide.pdf, S. 61)

\documentclass[
paper=a4, % default-Einstellung
twoside, % default-Einstellung
fontsize = 11pt, % default-Einstellung
%%%BCOR=8mm, % Bindungs-Korrektur. Wird nicht benötigt, da Seitenränder mit "newgeometry"(s.u.) konfiguriert werden
%
% In KOMA-Script wird bei Verwendung des Gliederungsbefehls \part auf Gliederungsnummern mit abschließendem Punkt umgeschaltet.
% Dies kann mit der folgenden Einstellung abgeschaltet werden (s. scrguide.pdf, S. 105).
% numbers=noendperiod
numbers=noendperiod,
% Breite der Kopfzeile auf die Breite des Textbereichs einschließlich der Marginalienspalte und natürlich des Abstandes zwischen den beiden setzen (s. scrguide.pdf, S. 277)
headwidth=textwithmarginpar,
% Trennlinie anzeigen und auf 1pt setzen (s. srcguide.pdf, S. 278)
headsepline=1pt,
% Einschalten von 'chapterprefix' (s. srcguide.pdf, S. 102)
chapterprefix=true,
% Abbildungsverzeichnis als nicht nummerierter Eintrag im Inhaltsverzeichnis (s. scrguide.pdf, S. 153)
listof=totoc,
% Literaturverzeichnis als nicht nummerierter Eintrag im Inhaltsverzeichnis (s. scrguide.pdf, S. 159)
bibliography=totoc,
%%-----------------------------------------
%% Index vorerst auf Kommentar
% Stichwortverzeichnis als nicht nummerierter Eintrag im Inhaltsverzeichnis (s. scrguide.pdf, S. 160)
%index=totoc,
%%-----------------------------------------
% Vakatseiten unterdrücken. Das brauchen wir für die Rückseite des Deckblatts. Dafür müssen wir aber vor jedem "Part" und vor jedem "Chapter" ein "\cleardoublepage" einfügen.
open=any
]{scrbook}

%%%%%%%%%%%%%%%%%%%%%%%%%%%%%%%%%%%%%%%%%%%%%%%%%%%
%%%%% Seitenformat hier zentral festlegen
%%%%% ... am einfachsten mit newgeometry
%%%%%%%%%%%%%%%%%%%%%%%%%%%%%%%%%%%%%%%%%%%%%%%%%%%
\usepackage[pass]{geometry}
% schmaler Seitenrand
\newgeometry{
	left=25mm, right=45mm, top=30mm, bottom=40mm,
	marginparwidth=30mm
}

%%%%%%%%%%%%%%%%%%%%%%%%%%%%%%%%%%%%%%%%%%%%%%%%%%%
%%%%% Packages
%%%%%%%%%%%%%%%%%%%%%%%%%%%%%%%%%%%%%%%%%%%%%%%%%%%

% für Umlaute und Encoding
\usepackage[utf8]{inputenc}
\usepackage[T1]{fontenc}

% für 'Deutsch'
\usepackage[ngerman]{babel}

% ??? wird das noch gebraucht --- ja, sonst gibt es eine Warning
\usepackage[babel, german=quotes]{csquotes} % einfache Handhabung von quotations

% für eigene verwendete Farben
% xcolor wird für die tcolorbox benötigt (tcolorbox wird bei den Autoren-Kästen verwendet)
% xcolor muss vor dem Definieren der eigenen Farben geladen werden.
\usepackage[dvipsnames]{xcolor}

% Kopf- und Fußzeilen konfigurieren
\usepackage{scrlayer-scrpage}

%% Index vorerst auf Kommentar
%% Package für einen Index
%% Quellen: 
%% https://www1.ku.de/urz/schriften/makeidx.pdf
%% https://www.namsu.de/Extra/pakete/Makeidx.html
%\usepackage{makeidx}
%\makeindex

% Mit Hilfe dieses Packages kann ein Seitenumbruch verhindert werden.
% Quelle: https://golatex.de/viewtopic.php?t=2070
% Quelle: http://ctan.org/pkg/needspace
% Eine interessante Alternative ist das needspace-Paket, damit kann man den zumindest benötigten Platz angeben, so dass nach Bedarf umgebrochen wird. Damit lassen sich beispielsweise Einleitung und ein oder zwei folgende Aufzählungspunkte zusammenhalten, ohne dass die komplette Umgebung auf eine Seite gezwungen ist, also in der Folge auch Seitenumbrüche enthalten darf. Die Dokumentation zu needspace findet sich in needspace.sty selbst.
% needspace habe ich in der kommentierten Literaturliste verwendet, um die Autor/Titelangabe mit den ersten ein bis zwei Zeilen des Kommentars zusammenzuhalten (s. Makro \kommlititem).
\usepackage{needspace}

% tcolorbox wird bei den Autoren- und anderen Kästen verwendet
\usepackage[most]{tcolorbox}

% mit dem Paket können wir die Sprechblasen gut als Aufzählungspunkte verwenden
\usepackage{enumitem}

% Dieses Package unterstützt das Verbinden von Zeilen in Tabellen ...
\usepackage{multirow}
% ...und dieses Umbrüche, wenn große Tabellen über mehrere Seiten gehen.
\usepackage{longtable}

\usepackage{latexsym}

% Dieses Package benötigen wir für die Darstellung der mathematischen Symbole für Zahlenmengen (\mathbb{}).
\usepackage{amsfonts} 

%--------------------------------------------------
% Fußnoten innerhalb von tcolorbox wurden nicht am Seitenende angezeigt, sondern ebenfalls innerhalb der tcolorbox.
% Das kann man mit dem folgenden Code verhindern.
% Quelle: https://tex.stackexchange.com/questions/558709/tcolorbox-footnotes-at-end-of-each-page/558922#558922
% Anmerkung: funktioniert nicht mit "breakable"
%--------------------------------------------------
\usepackage{footnotehyper}
\makesavenoteenv{tcolorbox}
%--------------------------------------------------

% wird für die Deckblätter verwendet
\usepackage{tikz}

% Für Listings werden die beiden folgenden Pakete benötigt.
\usepackage{sourcecodepro}
\usepackage{listings}

% weiters Paket für die Darstellung der Algorithmen in Kapitel Qualitätssicherung
\usepackage[linesnumberedhidden,ruled,vlined,algochapter,german]{algorithm2e}

% Seiten komplett bebildern
% Wird hier gebraucht: Zoo-Grafik auf ganzer Seite
% Tipp von hier: https://www.julo.ch/blog/latex-bilder-komplette-seite/
\usepackage{graphicx}
\usepackage[absolute]{textpos}

% Workaround gegen die verpixelte Schrift
\usepackage{lmodern}

% Das hyperref-Paket wird üblicherweise als letztes Paket in der Präambel 
% geladen, da es sich so auf alle zuvor geladenen Pakete einstellen kann.
\usepackage{hyperref}

% https://ctan.mc1.root.project-creative.net/macros/latex/contrib/oberdiek/hypcap.pdf
% Links auf Gleitumgebungen springen nicht zur Beschriftung,
% sondern zum Anfang der Gleitumgebung
\usepackage[figure]{hypcap}

% Das Package wird u.a. fuer das Makro \zb benoetigt.
\usepackage{xspace}

%%%%%%%%%%%%%%%%%%%%%%%%%%%%%%%%%%%%%%%%%%%%%%%%%%%
%%%%% Angaben für das Glossar
%%%%%%%%%%%%%%%%%%%%%%%%%%%%%%%%%%%%%%%%%%%%%%%%%%%
% Vorerst auf Kommentar, da kein Glossar verwendet wird.
%%%%%%%%%%%%%%%%%%%%%%%%%%%%%%%%%%%%%%%%%%%%%%%%%%%
%% Glossaries
%% http://ftp.fau.de/ctan/macros/latex/contrib/glossaries/glossariesbegin.pdf
%% As with all packages, you need to load glossaries with \usepackage, but there are certain packages that must be loaded before glossaries, if they are required:
%% hyperref, babel, polyglossia, inputenc and fontenc.
%% (You don’t have to load these packages, but if you want them, you must load them before glossaries.)
%
%% für die Verwendung eines Glossars
%\usepackage[
%%acronym,     % ein Abkürzungsverzeichnis erstellen
%nonumberlist, % keine Seitenzahlen anzeigen
%toc,          % Einträge im Inhaltsverzeichnis
%nopostdot,    % kein extra Punkt am (Satz-)Ende des Glossartextes
%%section,     % im Inhaltsverzeichnis auf section-Ebene erscheinen
%%numberedsection,
%style=altlistgroup      % alternativ "style=altlist" verwenden, dann werden die Begriffe nicht nach Anfangsbuchstaben gruppiert.
%]
%{glossaries}
%\makeglossaries
%
%% Datei mit den Glossar-Einträgen einbinden
%%%%%%%%%%%%%%%%%%%%%%%%%%%%%%%%%%%%%%%%%%%%%%%%%%%%
%%%%% Glossar vorerst auf Kommentar
%%%%%%%%%%%%%%%%%%%%%%%%%%%%%%%%%%%%%%%%%%%%%%%%%%%

% Diese Datei enthält alle Glossar-Einträge für den gesamten Kurs.

% Im Moment müssen Umlaute und das ß noch gesondert ausgezeichnet werden!!!

\newglossaryentry{Anwendungsfalldiagramm}
{
	name=Anwendungsfalldiagramm,
	description={TODO: }
}

\newglossaryentry{Anwendungsprogrammierer}
{
    name=Anwendungsprogrammierer,
    description={TODO: Glossartext f\"ur den Begriff Anwendungsprogrammierer}
}

\newglossaryentry{CASE_Tools}
{
    name=CASE Tools,
    description={TODO: Glossartext f\"ur den Begriff CASE Tools}
}

\newglossaryentry{Domaenenexperte}
{
    name=Dom\"anenexperte,
    description={Eine Person, die das fachliche Anwendungsgebiet (die Dom\"ane), f\"ur das das Softwareprodukt erstellt werden soll, sehr gut kennt und daher beurteilen kann, ob die technische Umsetzung den fachlichen Erfordernissen entspricht.}
}

\newglossaryentry{Kapselung}
{
    name=Kapselung,
    description={TODO: Glossartext f\"ur den Begriff Kapselung}
}

\newglossaryentry{Klassendiagramm}
{
	name=Klassendiagramm,
	description={TODO: }
}

\newglossaryentry{Komponente}
{
	name=Komponente,
	description={TODO: }
}

\newglossaryentry{Komponentendiagramm}
{
	name=Komponentendiagramm,
	description={TODO: }
}

\newglossaryentry{Konsistenz}
{
    name=Konsistenz,
    description={TODO: Glossartext f\"ur den Begriff Kapselung}
}

\newglossaryentry{Kontextdiagramm}
{
	name=Kontextdiagramm,
	description={TODO: kommt aus der Strukturierten Analyse}
}

\newglossaryentry{Library}
{
    name=Library,
    description={TODO: Glossartext f\"ur den Begriff Library}
}

\newglossaryentry{Objektdiagramm}
{
	name=Objektdiagramm,
	description={TODO: }
}

\newglossaryentry{Prototyp}
{
	name=Prototyp,
	description={TODO: Glossartext f\"ur den Begriff Prototyp}
}

\newglossaryentry{Rollen}
{
    name=Rollen,
    description={TODO: Glossartext f\"ur den Begriff Rollen}
}

\newglossaryentry{Schnittstelle}
{
    name=Schnittstelle,
    description={TODO: Glossartext f\"ur den Begriff Schnittstelle}
}

\newglossaryentry{Skalierbarkeit}
{
    name=Skalierbarkeit,
    description={TODO: Glossartext f\"ur den Begriff Skalierbarkeit}
}

\newglossaryentry{Softwarearchitekt}
{
    name=Softwarearchitekt,
    description={TODO: Glossartext f\"ur den Begriff Softwarearchitekt}
}

\newglossaryentry{Softwareprodukt}
{
    name=Softwareprodukt,
    description={Das Ergebnis eines Softwareentwicklungsprozesses.}
}

\newglossaryentry{Softwaretests}
{
    name=Softwaretests,
    description={TODO: Integrationstests überprüfen das Zusammenspiel verschiedener Komponenten.
    	im Systemtest wird geprüft, ob die Software die Gesamtheit der definierten Anforderungen erfüllt}
}

\newglossaryentry{Systemarchitektur}
{
    name=Systemarchitektur,
    description={TODO: Glossartext f\"ur den Begriff Systemarchitektur}
}

\newglossaryentry{Validierung}
{
    name=Validierung,
    description={TODO: Glossartext f\"ur den Begriff Validierung}
}

\newglossaryentry{Verifikation}
{
    name=Verifikation,
    description={TODO: Glossartext f\"ur den Begriff Verifikation}
}

\newglossaryentry{Vision}
{
	name=Produktvision,
	description={TODO: Glossartext f\"ur den Begriff Vision/Produktvision}
}

\newglossaryentry{Wartbarkeit}
{
    name=Wartbarkeit,
    description={Ein Qualit\"atskriterium f\"ur Software. Es gibt an, mit wieviel Aufwand und mit welchem Erfolg \"Anderungen an der bestehenden Software vorgenommen werden k\"onnen.}
}


%%%%%%%%%%%%%%%%%%%%%%%%%%%%%%%%%%%%%%%%%%%%%%%%%%%
%%%%% Angaben für das Literaturverzeichnis
%%%%%%%%%%%%%%%%%%%%%%%%%%%%%%%%%%%%%%%%%%%%%%%%%%%

\usepackage[
	backend=biber,
	style=alphabetic,
	%citestyle=authoryear
]{biblatex} % biblatex mit biber laden

\DeclareFieldFormat{labelalpha}{\thefield{entrykey}}
\DeclareFieldFormat{extraalpha}{}
%\DeclareFieldFormat*{volume}{Vol.~#1} %% auf diese Weise könnte man Bd. durch Vol. ersetzen

\ExecuteBibliographyOptions{
% Beim Eintrag sorting=nyt funktionierte die Sortierung nicht wie gewünscht.
% Der Eintrag "swe14" wurde unter "b" einsortiert (Autorname Bourque).
% sorting=debug -> jetzt stimmt die Sortierung, aber bitte nochmal kontrollieren
%	sorting=nyt,   % Sortierung Autor, Jahr, Titel
	sorting=debug, % Eigentlich nur fuer die Fehlersuche vorgesehen.
	               % Sortierung nach den Eintragsschlüsseln
	bibwarn=true,  % Probleme mit den Daten, die Backend betreffen anzeigen
	isbn=false,    % keine isbn anzeigen
	doi=true,      % erstmal doch DOI anzeigen
	url=true       % erstmal doch URL anzeigen
}

% Definieren der Ueberschrift fuer die Literatur-Verzeichnisse der einzelnen Lektionen
\defbibheading{subbibliography}{%
	\chapter*{Literatur für Lektion \ref{refsection:\therefsection}}
	\addcontentsline{toc}{chapter}{Literatur für Lektion \ref{refsection:\therefsection}}
	\markboth{Literatur für Lektion \ref{refsection:\therefsection}}{Literatur für Lektion \ref{refsection:\therefsection}}
}

% bib-Datei laden
% Die bib-Datei erzeugt Maren aus der Citavi-Datenbank.
\addbibresource{Kurs_1793_Literatur.bib}

%%%%%%%%%%%%%%%%%%%%%%%%%%%%%%%%%%%%%%%%%%%%%%%%%%%
%%%%% Für zusätzliche Inhaltsverzeichnisse
%%%%% pro Kapitel
%%%%%%%%%%%%%%%%%%%%%%%%%%%%%%%%%%%%%%%%%%%%%%%%%%%
% Idee stammt von hier
% https://texwelt.de/fragen/6162/kapiteltitel-und-kapitel-inhaltsverzeichnis
\usepackage{etoc}
\newcommand{\chaptertoc}[1][Inhalt der \partname~\thepart]{%
	\etocsettocstyle{\addsec*{#1}}{}%
	\localtableofcontents%
	\cleardoublepage%
}


%%%%%%%%%%%%%%%%%%%%%%%%%%%%%%%%%%%%%%%%%%%%%%%%%%%
%%%%% Angaben für ein zusätzliches Verzeichnis
%%%%% für die Quellenangaben verschiedener Abbildungen
%%%%%%%%%%%%%%%%%%%%%%%%%%%%%%%%%%%%%%%%%%%%%%%%%%%
\DeclareNewTOC[%
type=figreference,%
types=figreferences,%
float,%
floattype=4,%
name=Quelle,%
floatpos=ht,%
listname={Verzeichnis der Quellenangaben für Abbildungen}%
]{lofr}

\setuptoc{lofr}{chapteratlist}

%%%%%%%%%%%%%%%%%%%%%%%%%%%%%%%%%%%%%%%%%%%%%%%%%%%
%%%%% Einstellungen
%%%%%%%%%%%%%%%%%%%%%%%%%%%%%%%%%%%%%%%%%%%%%%%%%%%

%--------------------------------------------------
% Konfiguration für hyperref
%--------------------------------------------------
% Die Optionen für hyperref besser erst mittels \hypersetup konfigurieren, da sonst
% die Leerzeichen in title, subject, keywords und author verschluckt werden.
\hypersetup{
	bookmarks=true,%
	bookmarksopen=false,% Klappt die Bookmarks in Acrobat aus
	pdftitle   = {Modul 63812 - Software Engineering},
	pdfsubject = {Software Engineering},
	pdfauthor  = {Prof. Dr. Jörg Desel und Maren Stephan},
	%pdfkeywords = {Stichwort1, Stichwort2 ...},
	%pdfcreator  = {Mit welcher Anwendung i.d.R. pdflatex},
	%pdfproducer = {LaTeX with hyperref}
	%
	% Durch die folgenden Optionen werden alle Links farblich mit einem Rand markiert (Default-Einstellung).
	colorlinks=false,
	linkbordercolor={1 0 0},% red
	urlbordercolor={0 1 1},% cyan
	citebordercolor={0 1 0},% green
	%
	% Durch die folgenden Optionen werden alle Links nicht farblich hervorgehoben, sondern stattdessen in Schwarz angezeigt.
	% (ACHTUNG: die Texte sind dann IMMER schwarz, auch da, wo man eigentlich andere Farben gesetzt hat.)
	%colorlinks=true,
	%linkcolor=black,
	%urlcolor=black,
	%citecolor=black,	
	%
	% Evtl. ist das unsere gewünschte/bevorzugte Einstellung.
	% Alle Links werden mit normalem Text angezeigt - keine Hervorhebungen. Damit klappt es auch mit den ursprünglich eingestellten Farben von Texten, die jetzt zu Links werden (z.B. Inhaltsverzeichnis).
	% hidelinks, 
}
%--------------------------------------------------

%--------------------------------------------------
% Definieren des Grüns der Fakultät M+I (RGB-Wert: #006666)
%--------------------------------------------------
\definecolor{FernUni-MI-green}{RGB}{0,102,102}
\definecolor{FernUni-MI-green-light}{RGB}{224,236,236}
%--------------------------------------------------

%--------------------------------------------------
% Kopf- und Fußzeilen konfigurieren
%--------------------------------------------------
\ihead{\headmark}
\ohead{\thepage}
\ofoot[\thepage]{}
\pagestyle{scrheadings} % (s. scrguide.pdf, S. 258)

% Schriftart für Kopf- und Fußzeilen:
% grün, ohne Serifen und fett
\setkomafont{pagehead}{\normalfont\color{FernUni-MI-green}}
\addtokomafont{pagehead}{\sffamily\bfseries}
\setkomafont{pagefoot}{\normalfont\color{FernUni-MI-green}}
\addtokomafont{pagefoot}{\sffamily\bfseries}
% Fußzeilen werden nur auf der ersten Seite eines Kapitels angezeigt
%--------------------------------------------------

%--------------------------------------------------
% Formatierung der Gliederungsebenen 
% (Part / Chapter / Section / Subsection / Subsubsection / Paragraph / SubParagraph / Minisec)
%--------------------------------------------------
%--- Schriftart für alle sieben Gliederungsebenen bestimmen (s. scrguide.pdf, S. 110)
\setkomafont{disposition}{\normalfont\color{FernUni-MI-green}}
\addtokomafont{disposition}{\bfseries}
\addtokomafont{disposition}{\sffamily}
%--- Schriftart für die Gliederungsebene 'Kapitel' bestimmen
\setkomafont{chapterprefix}{\Huge} % für 'Kapitel X'
\setkomafont{chapter}{\Large} % für den Titel
%--- Schriftart für die Gliederungsebene 'Part' bestimmen
\setkomafont{partnumber}{\normalsize} % für 'Lektion X'
\setkomafont{part}{\normalsize} % für den Untertitel (gleiche Formatierung wie 'partnumber', damit auch der Anhang entsprechend ausgezeichnet wird)

%--- part ---
% bei Part soll 'Lektion X' (arabische Nummerierung) ausgegeben werden, auch im Inhaltsverzeichnis
\renewcaptionname{ngerman}{\partname}{Lektion}
\renewcommand\thepart{\arabic{part}}
\renewcommand*{\partformat}{\partname~\thepart\autodot}

% Einstellung 'Part' für das Inhaltsverzeichnis
%\addtokomafont{partentry}{\normalsize \bfseries} % Font für den Eintrag 'Part' im Inhaltsverzeichnis
\addtokomafont{partentrypagenumber}{\nullfont} % keine Seitenangabe für den Eintrag 'Part' im Inhaltsverzeichnis

\renewcommand*{\addparttocentry}[2]{
	\addtocentrydefault{part}{}{\partname~#1~\\#2}% Ausgabe von 'Lektion X', neue Zeile, Untertitels im Inhaltsverzeichnis. Funktioniert aber nicht für den Anhang, daher wird für den Anhang dieser Befehl neu konfiguriert
}

% Keine Kopf- und Fußzeile auf der 'Part'-Seite
\renewcommand\partpagestyle{empty}

% 'Part'-Seite mit FerUni-Deckblatt
\renewcommand*{\partlineswithprefixformat}[3]{%
	\sttpDeckblatt{\Kursautor}{\Modulnummer}{\Modulname}{\partname~\thepart}{#3}{\Zeitstempel}
}

%--- chapter ---
% bei Chapter wird der Text 'Kapitel X' rechtsbündig ausgegeben und
% beim Kapiteltext werden zusätzlich zwei horizontale Linien ausgegeben
\renewcommand{\chapterlineswithprefixformat}[3]{
	{\raggedleft #2}
	\noindent\rule[1ex]{\textwidth}{5pt}
	#3
	\noindent\rule{\textwidth}{5pt}
}

% Einstellung 'Chapter' in der Kopfzeile
% im Kolumnentitel (Heading) nur die Kapitelnummer, aber nicht 'Kapitel' ausgeben (s. scrguide.pdf, S. 120)
\renewcommand*\chaptermarkformat{\thechapter\autodot\enskip}

%--- Nummerierung ---
% Überschriften werden bis Ebene subsubsection nummeriert (s. scrguide.pdf, S. 121)
\setcounter{secnumdepth}{\subsubsectiontocdepth}
% Überschriften werden bis Ebene subsubsection ins Inhaltsverzeichnis aufgenommen
\setcounter{tocdepth}{\subsubsectiontocdepth}
%--------------------------------------------------

%--------------------------------------------------
% Formatierung der Fußnoten
%--------------------------------------------------
% durchgängige Nummerierung der Fußnoten 
% Idee von hier: https://golatex.de/viewtopic.php?t=2232
\usepackage{chngcntr}
\counterwithout{footnote}{chapter}

% Formatierung der Fußnoten (s. scrguide.pdf, S. 98)
\deffootnote{1.5em}{1em}{%
	\textsuperscript{\thefootnotemark}%
}
%--------------------------------------------------

%--------------------------------------------------
% Marginalien in farbig, fett und ohne Serifen
%--------------------------------------------------
\let\MLine\marginline
\renewcommand{\marginline}[1]{\MLine{\textcolor{FernUni-MI-green}{\textbf{\textsf{#1}}}}}
%--------------------------------------------------

%--------------------------------------------------
% Textformatierung
%--------------------------------------------------
% Textformatierung: Kein Einzug, etwas Abstand zwischen den Absätzen (s. scrguide.pdf, S. 506)
\setparsizes{0pt}{1ex}{0pt plus 1fil}

% Textformatierung: Schriftart Frutiger (aus feultr.cls - Briefvorlage FernUni)
%\renewcommand{\rmdefault}{lfrs}
%\renewcommand{\seriesdefault}{l}
%\renewcommand{\shapedefault}{n}

% Für die serifenlose Schrift verwenden wir Frutiger.
\renewcommand{\sfdefault}{lfrs}

%--------------------------------------------------

% Einstellungen zum "Debuggen"
%\usepackage{layout}
%\usepackage{showframe}

%--------------------------------------------------
% zentrale Trennhilfe für Silbentrennung
% Für bestimmte Wörter, die häufig im Text vorkommen und immer wieder falsch 
% getrennt werden, sollten wir hier eine zentrale Trennhilfe konfigurieren.
% Insbesondere "Software" wird gerne falsch getrennt (Softwa-re). Das gilt auch für die
% mit dem Begriff "Software" zusammengesetzten Wörter.
% Wir werden hier aber nur die Trennhilfen für Wörter vorgeben, die im Text häufig falsch getrennt werden.
%--------------------------------------------------
\hyphenation{Soft-ware Soft-ware-archi-tek-tur Soft-ware-engi-nee-ring Soft-ware-ent-wick-lung Soft-ware-ent-wick-lungs-pro-jekt Soft-ware-ent-wick-lungs-pro-jek-ten Soft-ware-ent-wick-lungs-pro-jekts Soft-ware-pro-dukt Soft-ware-pro-duk-te Soft-ware-pro-dukts Soft-ware-ent-wick-lungs-team Soft-ware-sys-tem Soft-ware-sys-te-me 
	Soft-ware-test Soft-ware-tests
	Petri-netz Petri-netze Petri-net-zen
	Model-lierung
	pa-ral-lel pa-ral-le-le
	IEEE
	SEVOCAB
}

%%%%%%%%%%%%%%%%%%%%%%%%%%%%%%%%%%%%%%%%%%%%%%%%%%%
%%%%% Deklarationen und Macros
%%%%%%%%%%%%%%%%%%%%%%%%%%%%%%%%%%%%%%%%%%%%%%%%%%%

%%-----------------------------------------------------
%%---------- Makros für deutsche Abkürzungen ----------
%%-----------------------------------------------------
\newcommand{\dasHeisst}{d.h.\@\xspace}
\newcommand{\sa}{s.a.\@\xspace}
\newcommand{\so}{s.o.\@\xspace}
\newcommand{\su}{s.u.\@\xspace}
\newcommand{\zb}{z.B.\@\xspace}

% Über dieses Makro wird die Kapitelüberschrift für das Fallbeispiel erzeugt.
% Darin enthalten sind auch der Eintrag für das Inhaltsverzeichnis sowie die Angaben für linke und rechte Kopfzeile.
\newcommand{\FallBeispielZoo}{
	\chapter*{Fallbeispiel Zoo ~~~ \raisebox{-0.35cm}{\includegraphics[height=1.0cm]{Bilder/elephant_emoji.png}}}
	\addcontentsline{toc}{chapter}{Fallbeispiel Zoo ~~~ \raisebox{-0.25cm}{\includegraphics[height=0.8cm]{Bilder/elephant_emoji.png}}}
	\markboth{Fallbeispiel Zoo ~ \raisebox{-0.15cm}{\includegraphics[height=0.6cm]{Bilder/elephant_emoji.png}}}{\raisebox{-0.15cm}{\includegraphics[height=0.6cm]{Bilder/elephant_emoji2.png}} ~ Fallbeispiel Zoo}
}

% neues Kommando für farblich hervorgehobenen Text 
% Im Kurstext wird diese Hervorhebung verwendet, wenn der erste Satz eines Absatzes hervorgehoben werden soll. Funktioniert aber auch an jeder beliebigen Stelle.
% Hervorgehobener Text wird in FernUni-MI-green dargestellt.
% Das Kommando erhält 1 Parameter
%   - Text, der farblich hervorgehoben wird
\newcommand{\sttpHervorhebung}[1]{\textcolor{FernUni-MI-green}{#1}}

%% Glossar vorerst auf Kommentar
%% neues Kommando für einen STTP-eigenen Glossareintrag (sttpgls -> gls => Glossar)
%% Der Glossar-Begriff zu dem Glossar-Schlüsselwort wird im Seitenrand angezeigt und mit einem entsprechenden Icon versehen.
%\newcommand{\sttpgls}[1]{\marginline{\includegraphics[height=0.3cm]{Bilder/gluehbirne.png} \gls{#1}}}

% neues Kommando für einen Kapitelverweis
% Ein Kapitelverweis wird im Seitenrand positioniert und sieht wie folgt aus: "`Begriff"' -> "`Kap. 1.1"' (mit Verlinkung).
% Der Kapitelverweis erhält 2 Parameter
%   - Begriff
%   - Verweis (der eigentliche Verweis soll allerdings beim Aufruf gemacht werden)
\newcommand{\sttpkapitelverweis}[2]{\marginline{#1 $\rightarrow$~#2}}

%--------------------------------------
% neues Kommando für einen Eintrag in der 'Kommentierten Literatur' (KommLitItem -> Komm(entierte) Lit(eratur) Item (Eintrag))
% enthält 7 Parameter
%   - Autor(en)
%   - Jahr
%   - Titel
%   - Key für die Literatur-Referenz
%   - #5 wird derzeit nicht verwendet
%     Dateinamen der Grafik für das Buchcover
%     Wenn der Parameter leer bleibt ('{}'), wird kein Buchcover angezeigt.
%   - #6 wird derzeit nicht verwendet
%     Seitenangabe, wenn es nur ein Artikel aus dem Buchcover ist.
%     Wenn der Parameter leer bleibt ('{}'), wird der Text nicht angezeigt.
%     Wird auch nur angezeigt, wenn ein Buchcover angegeben ist (Parameter #5).
%   - Kommentar zu diesem Eintrag der Literaturliste
%--------------------------------------
\newcommand{\sttpKommLitItem}[7]{
	\needspace{2\baselineskip} \noindent 
	\textit{#1 
%		\ifthenelse{\equal{#5}{\empty}}
%			{} % kein Buchcover angegeben
%			{
%				\ifthenelse{\equal{#6}{\empty}}
%				{\sttpMarginPicture{#5}} % Anzeige Buchcover ohne Text
%				{\sttpMarginPictureWithText{#5}{#6}} % Anzeige Buchcover mit Text
%			} % else
		(#2). #3
	}~\cite{#4}\smallskip\\#7\\
}
%--------------------------------------
% Dieses Kommando ist ein Helfs-Befehl, mit dem eine zusätzliche die Fußnote hinter der Lietratur-Referenz in der 'kommentierten Literaturliste' erzeugt werden kann (dies kommt beim ersten Eintrag in der Einleitung vor).
%   - Fußnoten-Text (8. Parameter)
\newcommand{\sttpKommLitItemMitFussnote}[8]{
	\needspace{2\baselineskip} \noindent 
	\textit{#1 
%		\ifthenelse{\equal{#5}{\empty}}
%			{} % kein Buchcover angegeben
%			{
%				\ifthenelse{\equal{#6}{\empty}}
%				{\sttpMarginPicture{#5}} % Anzeige Buchcover ohne Text
%				{\sttpMarginPictureWithText{#5}{#6}} % Anzeige Buchcover mit Text
%			} % else
		(#2). #3
	}~\cite{#4}\footnote{#8}\smallskip\\#7\\
}
%--------------------------------------

%--------------------------------------
% Ich verwende ein neues Kommando für Bilder, die im Seitenrand (marginpar) an
% der oberen Kante eines Absatzes positioniert werden.
% Der Befehl hat 1 Parameter
%   - Dateinamen der Grafik (Buchcover)
% Benutzt wird dies für die Buchcover der Werke, die in den kommentierten Literaturlisten vorgestellt werden.
% Quelle: Bild in Marginpar an oberer Kante des Absatzes positionieren
% https://texwelt.de/fragen/11095/bild-in-marginpar-an-oberer-kante-des-absatzes-positionieren
%--------------------------------------
\newcommand{\sttpMarginPicture}[1]{%
	\sttpMarginPictureWithText{#1}{}
}
%--------------------------------------
% Den Befehl gibt es auch noch in einer etwas abgewandelten Form, so dass eine zusätzliche Seitenangabe gemacht werden kann.
% Der Befehl hat 2 Parameter
%   - Dateinamen der Grafik (Buchcover)
%   - Text für die Seitenangabe
%     Diese wird unterhalb des Buchcovers angezeigt.
%--------------------------------------
\newcommand{\sttpMarginPictureWithText}[2]{%
  \leavevmode\marginline{%
    \raisebox{\dimexpr-\totalheight+\ht\strutbox\relax}%
      [\ht\strutbox][\dp\strutbox]{
					\parbox[b]{2.6cm}{\centering \includegraphics[width=2.2cm]{#1}\\#2}
		}
	}
}
%--------------------------------------
% Das folgende Makro ist für ein Bild im Seitenrand (marginpar) gedacht.
% Das Makro sttpMarginPicture kann dafür nicht verwendet werden, da das für die Buchcover gebastelt wurde.
% Auch hier sollen Bilder im Seitenrand (marginpar) an der oberen Kante eines Absatzes positioniert werden.
% Der Befehl hat 1 Parameter
%   - komplette Angabe für das einzufügende Bild (samt "\includegraphics[width...]{}")
% Benutzt wird dies erstmal für die kleinen bunten Kästen, die aus der Abbildung 3.1 (MindMap) stammen.
%
% Quelle: Bild in Marginpar an oberer Kante des Absatzes positionieren
% https://texwelt.de/fragen/11095/bild-in-marginpar-an-oberer-kante-des-absatzes-positionieren
%--------------------------------------
\newcommand{\sttpForImageInMargin}[1]{%
	\leavevmode\marginline{%
		\raisebox{\dimexpr-\totalheight+\ht\strutbox\relax}%
		[\ht\strutbox][\dp\strutbox]{
			\parbox[b]{\marginparwidth}{#1}
		}
	}
}
%--------------------------------------

% neues Kommando für einen Autoren-Kasten
% In einem Autorenkasten wird das Bild des Autors, der Name, das Geburtsdatum und ein erklärender Text angezeigt.
% enthält 5 Parameter
%   [1] - Name des Autors
%   [2] - Geburtsjahr
%   [3] - Todesjahr (wird nur angezeigt, wenn nicht leer)
%   [4] - erweiterter Text
%   [5] - Bilddatei
%   [6] - Datum der Aufnahme (nur Jahr)
%   [7] - Quellenangabe
\newcommand{\sttpAutorenkasten}[7]{
	\vspace{2ex}
	\phantomsection % damit hyperref korrekt funktioniert
	\addcontentsline{lofr}{section}{Foto #1 #6. #7} % Eintrag fuer das Foto in das Quellenverzeichnis
	\begin{tcolorbox}[
			center,
			width=0.8\textwidth,
			enhanced,
			%--Beginn Bildquelle--Angaben für die Bildquelle im "title"
			flip title={sharp corners},
			coltitle=black,
			colbacktitle=gray!5,
			fonttitle=\tiny,% andere mögliche Größen: \small, \footnotesize, \scriptsize, \tiny
			title={Bildquelle: #7 (#6)},
			%--Ende Bildquelle--
			sidebyside, % upper and lower part nebeneinander statt untereinander
			segmentation empty, % keine Trennlinie
			lefthand width=4.0cm,
			boxrule=2.4pt,
			left=2mm, right=2mm, top=2mm, bottom=2mm, middle=2mm,
			titlerule=0pt,
			colback=gray!5,
			colframe=FernUni-MI-green!30,
			sharp corners,
			drop fuzzy shadow
		]
		% -------------------------------------------------
		\begin{center}
			\includegraphics[width=2.5cm]{#5}%
			\\
			\vspace{2mm}
			\textbf{\mbox{#1}}
			\\
			\textbf{\footnotesize{\mbox{(\**#2\ifthenelse{\equal{#3}{\empty}}{}{,~\dag#3})}}}
		\end{center}
		\tcblower % -------------------------------------------------
		#4
		% -------------------------------------------------
	\end{tcolorbox}
	\vspace{2ex}
}

%%--------------------------------------
% neues Kommando für einen Definitions-Kasten
% In einem Definitions-Kasten wird ein Begriff, ein Kurztext (kursiv) und ein Langtext (normal) angezeigt.
% Das Kommando enthält 4 Parameter
%   - relative Breite des Kastens zur Textbreite
%   - Begriff, wird fett in einer farblich unterlegten Box dargestellt
%   - Kurztext, wird kursiv dargestellt (bei leerem Parameter entfällt dieser Text)
%   - Langtext, wird normal dargestellt (bei leerem Parameter entfällt dieser Text)
%%--------------------------------------
% Alle Definitionskästen haben eine Breite von 0.8\textwidth.
% In Kapitel 5 (Petrinetze / PN) werden die Kästen mit 0.9 skaliert.
\newcommand{\sttpDefinitionskastenSkalierungsfaktor}{0.8}
\newcommand{\sttpDefinitionskastenSkalierungsfaktorKapPN}{0.9}
%%--------------------------------------
\newcommand{\faktorBreite}{\sttpDefinitionskastenSkalierungsfaktor}
\newcommand{\sttpDefinitionskasten}[4]{
	\vspace{2ex}
	\begin{tcolorbox}[
		center,
		width=#1\textwidth,
		enhanced,
		adjusted title=#2,
		% Das kleine "i"-Symbol in der Titelzeile des Kastens entfernen.
		%after title={\hfill\includegraphics[height=0.3cm]{Bilder/i.png}},
		fonttitle=\sffamily\bfseries,
		coltitle=FernUni-MI-green!80!black,
		enlarge top initially by=0mm,
		enlarge bottom finally by=2mm,
		left=2mm,right=2mm,middle=0.5mm,
		segmentation hidden,
		sharp corners,
		boxrule=2.4pt,
		colback=gray!5,
		colframe=FernUni-MI-green!30,
		drop fuzzy shadow,
	]
% TODO AF: nochmal prüfen
%		% -------------------------------------------------
%		% Kurztext nur anzeigen, wenn nicht leer
%		\ifthenelse{\equal{#2}{\empty}}{}{\textit{#2}}
%		% -------------------------------------------------
%		% den Wechsel "`tcblower"' nur dann einfügen, wenn beide Elemente nicht leer sind
%		\ifthenelse{\equal{#2}{}}{}{\ifthenelse{\equal{#3}{}}{}{\tcblower}}
%		% -------------------------------------------------
%		% Langtext nur anzeigen, wenn nicht leer
%		\ifthenelse{\equal{#3}{\empty}}{}{#3}
%		% -------------------------------------------------
%
		% -------------------------------------------------
		% Kurztext nur anzeigen, wenn nicht leer
		\ifx&#3&\empty\else\textit{#3}\fi
		% -------------------------------------------------
		% Den Wechsel "`tcblower"' nur dann einfügen, wenn beide Elemente nicht leer sind
		\ifx&#3&\empty\else\ifx&#4&\empty\else\tcblower\fi\fi
		% -------------------------------------------------
		% Langtext nur anzeigen, wenn nicht leer
		\ifx&#4&\empty\else#4\fi
		% -------------------------------------------------
	\end{tcolorbox}
	\vspace{2ex}
}

%%--------------------------------------
% neues Kommando für einen Kasten mit Theorem
% In einem Kasten mit Theorem wird der angegebene Text angezeigt, zusammen mit dem String "Theorem:".
% Das Kommando enthält 2 Parameter
%   - relative Breite des Kastens zur Textbreite
%   - Textinhalt des Theorems
%%--------------------------------------
% Ein Kasten für ein Therorem hat standardmäßig die Breite 0.9\textwidth.
\newcommand{\sttpTheoremSkalierungsfaktor}{0.9}
%%--------------------------------------
\newcommand{\sttpTheorem}[2]{
	\vspace{1ex}
	\begin{tcolorbox}[
		center,
		width=#1\textwidth,
		colback=FernUni-MI-green!5!white,
		colframe=FernUni-MI-green!40!white
	]
	\textbf{Theorem:} #2
	\end{tcolorbox}
	\vspace{1ex}
}

% neues Kommando für einen Hinweis-Kasten
% In einem Hinweis-Kasten wird - ähnlich wie in einem Definitions-Kasten - ein Kurztext (fett) und ein Langtext (normale Schriftgröße) angezeigt.
% Zusätzlich erhält der Kasten 'Hinweis' als Überschrift.
% Das Kommando enthält 3 Parameter
%   - relative Breite des Kastens zur Textbreite
%   - Kurztext, wird fett dargestellt (bei leerem Parameter entfällt dieser Text)
%   - Langtext, wird normal dargestellt (bei leerem Parameter entfällt dieser Text)
\newcommand{\sttpHinweiskasten}[3]{
	\vspace{2ex}
	\begin{tcolorbox}[
		center,
		width=#1\textwidth,
		enhanced,
		title=Hinweis,
		attach boxed title to top left={xshift=-1mm,yshift=-\tcboxedtitleheight+1mm},
		boxed title style={colframe=FernUni-MI-green,colback=FernUni-MI-green!50},
		fonttitle=\sffamily\bfseries,
		top=\tcboxedtitleheight, % Abstand oben, gleiche Höhe wie 'height of the boxedtitle'
		middle=0.5mm,
		segmentation hidden,
		colback=gray!5,
		colframe=FernUni-MI-green!30,
		drop fuzzy shadow
	]
		% -------------------------------------------------
		% Kurztext (fett) nur anzeigen, wenn nicht leer
		\ifthenelse{\equal{#2}{\empty}}{}{\textbf{#2}}
		% -------------------------------------------------
		% den Wechsel Upper/Lower (tcblower) nur dann einfügen, wenn beide Elemente nicht leer sind
		\ifthenelse{\equal{#2}{}}{}{\ifthenelse{\equal{#3}{}}{}{\tcblower}}
		% -------------------------------------------------
		% Langtext (normal) nur anzeigen, wenn nicht leer
		\ifthenelse{\equal{#3}{\empty}}{}{{#3}}		
		% -------------------------------------------------
	\end{tcolorbox}
	\vspace{2ex}
}

% neues Kommando für einen Kasten
% In einem Kasten wird - ähnlich wie in einem Hinweis-Kasten - ein Text angezeigt.
% Der Kasten geht allerdings über die gesamte Textbreite, die Hintergrundfarbe ist weiß und der Rahmen ist etwas schmaler.
% Das Kommando enthält 1 Parameter
%   - Text, der angezeigt werden soll
\newcommand{\sttpKasten}[1]{
	\vspace{1ex}
	\begin{tcolorbox}[width=\textwidth,
		left=2mm,right=2mm,middle=0.5mm,
		% breakable, % breakable führt dazu, dass die Fußnoten innerhalb einer tcolorbox dargestellt werden und nicht am Seitenende
		bicolor,sharp corners,boxrule=1.5pt,colback=white,colbacklower=white,colframe=FernUni-MI-green!50]
		#1
	\end{tcolorbox}
	\vspace{1ex}
}
% Bei diesem Kasten ist zudem Seitenumbruch innerhalb des Kastens erlaubt.
% Aber Achtung: dann sind keine "normalen" Fußnoten möglich.
\newcommand{\sttpKastenBreakable}[1]{
	\vspace{1ex}
	\begin{tcolorbox}[width=\textwidth,
		left=2mm,right=2mm,middle=0.5mm,
		breakable, % breakable führt dazu, dass die Fußnoten innerhalb einer tcolorbox dargestellt werden und nicht am Seitenende
		bicolor,sharp corners,boxrule=1.5pt,colback=white,colbacklower=white,colframe=FernUni-MI-green!50]
		#1
	\end{tcolorbox}
	\vspace{1ex}
}

% neues Kommando für ein Zitat (wichtiges Zitat, dass direkt zum Text gehört)
% Das Zitat selbst wird kursiv dargestellt.
% Es soll kein Einzug der ersten Zeile vorgenommen werden (\noindent).
% Die Quelle erscheint rechtsbündig und in etwas kleinerer Schriftart darunter.
% Vor und nach dem Zitat wird ein vertikaler Abstand zum restlichen Text eingefügt.
% ---
% Anmerkung von Maren:
% Quelle kommt direkt hinter das Zitat in den ersten Parameter, 
% zweiter Parameter bleibt leer, wenn nur Quelle. 
% Zweiter Parameter für inhaltliche Zusätze 
% ---
% Das Kommando enthält 2 Parameter
%   - Zitattext (wird kursiv dargestellt)
%   - Quelle (wird rechtsbündig dargestellt) (kann auch leer bleiben)
\newcommand{\sttpzitat}[2]{
	\vspace{1ex}
	\noindent
	\textit{#1}

	\ifthenelse{\equal{#2}{\empty}}
	{} % keine Ausgabe, wenn #2 leer ist
	{
		{\raggedleft
			\small #2\\ % Anmerkung: die abschließenden "`\\" werden benötigt, damit der Text wirklich rechtsbündig ausgegeben wird
		}
	}
	\vspace{1ex}
}

% neues Kommando für ein Seitenrand-Zitat
% Ein Seitenrand-Zitat reicht nicht über die gesamte Seitenbreite.
% Es wird in einem Kasten dargestellt, der "`Außenrand-bündig"' ausgerichtet ist.
% Das Zitat selbst wird kursiv dargestellt.
% Die Quelle erscheint rechtsbündig und in etwas kleinerer Schriftart darunter.
% Vor und nach dem Zitat wird ein vertikaler Abstand zum restlichen Text eingefügt.
% Das Kommando enthält 2 Parameter
%   - Zitattext (wird kursiv dargestellt)
%   - Quelle (wird rechtsbündig dargestellt)
\newcommand{\sttpseitenrandzitat}[2]{
	\begin{tcolorbox}[
		if odd page={flush right}{flush left}, % ACHTUNG: die Unterscheidung linke Seite / rechte Seite geschieht manchmal erst nach dem 2. TeX-Durchlauf!
		width=1.0\textwidth,
		%width=0.75\textwidth, % alte Einstellung
		top=1.5mm,bottom=1.5mm,left=1.5mm,right=1.5mm,middle=0mm,
		bicolor,sharp corners,boxrule=1pt,colback=white,colbacklower=white,colframe=FernUni-MI-green!30,
		halign lower=right]
		% \tcbifoddpage{Odd}{Even} page!\\
		\textit{#1}
		\tcblower
		{\small #2}
	\end{tcolorbox}
}

%%% TODO kommentieren
% neues Kommando für die Leserführung (Einsatz erstmal nur in Kapitel 2)
% Die beiden Grafiken werden mittels zweier Minipages nebeneinander positioniert.
% Das Kommando enthält 2 Parameter
%   - Dateiname der linken Grafik
%   - Dateiname der rechten Grafik
\newcommand{\sttpLeserfuehrung}[2] {
	\vspace{1ex}
	\begin{center}
		\begin{minipage}[c]{.34\textwidth} 
			\includegraphics[scale=0.25]{#1}
		\end{minipage}
		\begin{minipage}[c]{.65\textwidth}
			\centering
			\includegraphics[scale=0.9]{#2}
		\end{minipage}
	\end{center}
	\vspace{1ex}
}


%--------------------------------------
% Das Kommando \sttpUMLText ist ein zentrales Marko, mit dem
% Text auf eine bestimmte Art (Typewriter-Schriftart ohne 
% Umbruch) dargestellt werden kann.
% Aktuell wird der Text mit \texttt{...} und ohne Umbruch
% (\mbox{...}) formatiert.
%
% Das Kommando enthält 1 Parameter
%   - Text, der entsprechend dargestellt werden soll
%--------------------------------------
% Achtung: es dürfen keine Leerzeichen und Zeilenumbrüche innerhalb
% des folgenden Makros enthalten sein.
\newcommand{\sttpUMLText}[1]{\mbox{\texttt{#1}}}
%--------------------------------------

%--------------------------------------
% neues Kommando für einen Kasten, in dem eine Anforderung dargestellt wird
% Der Text wird in Schreibmaschinenschrift dargestellt.
% Der Text wird zentriert dargestellt.
% Der Text wird in einem Kasten (100% Textbreite) mit hellgrünem Rand dargestellt.
%
% Das Kommando enthält 1 Parameter
%   - Text, der entsprechend dargestellt werden soll
%--------------------------------------
\newcommand{\sttpAnforderungText}[1]{
	\begin{tcolorbox}[
		width=\textwidth,
		top=1.5mm,bottom=1.5mm,left=1.5mm,right=1.5mm,middle=0mm,
		sharp corners,
		boxrule=10pt,
		colback=white,
		colframe=FernUni-MI-green!30,
		halign=flush center,
		]
		\texttt{#1}
	\end{tcolorbox}
}


%--------------------------------------
% neues Kommando für eine farbige kleine Box mit enthaltenem Text
% TODO kommentieren ...
%--------------------------------------
\newtcbox{\sttpMindMapText}[1][red]{
	on line,
	arc=0pt,
	outer arc=2pt,
	colback={#1},
	colframe={#1},
	coltext=white,
	boxsep=0pt,
	left=4pt,right=4pt,top=2pt,bottom=1pt,
	boxrule=1pt, 
}
% folgende Farben werden für die Texte benötigt
\definecolor{colMindMap1}{RGB}{153,0,0}
\definecolor{colMindMap2}{RGB}{74,74,186}
\definecolor{colMindMap3}{RGB}{0,102,204}
\definecolor{colMindMap4}{RGB}{200,112,4}
\definecolor{colMindMap5}{RGB}{38,115,77}


%--------------------------------------
% neues Kommando für eine Karteikarte
% Das Kommando enthält die folgenden 9 Parameter:
%   - Breite der Karteikarte (3cm oder ..., bitte nicht in "0.8\textwidth" angeben, das funktioniert z.B. in Tabellen nicht)
%   - Skalierungsfaktor
%   - Rotierungswinkel
%   - Kategorie (Kopfzeile des Kastens)
%   - Überschrift im 1. Teil
%   - Text zum 1. Teil
%   - Überschrift im 2. Teil
%   - Text zum 2. Teil (wenn leer, dann ohne Anzeige vom 2. Teil (auch ohne Überschrift #7))
%   - Ersteller-Name (wenn leer, dann ohne Anzeige dieser Zeile)
%--------------------------------------
% fuer die Karteikarte verwenden wir eine Schreibschrift
\newcommand*\schreibschrift{\fontfamily{wela}\selectfont}
% Für die Karteikarte können diese "Variablen" für den Skalierungsfaktor und den Rotierungswinkel verwendet werden.
% Das macht insbesondere dann Sinn, wenn ein und dieselbe Karteikarte an verschiedenen Stellen in verschiedenen
% Skalierungsgrößen und verschieden rotiert verwendet werden soll.
\newcommand{\sttpKarteikarteSkalierungsfaktor}{1.0}
\newcommand{\sttpKarteikarteRotierungswinkel}{0}
%--------------------------------------
\newcommand{\sttpKarteikarte}[9]{
	\smallskip
	\begin{tcolorbox}[
		skin=bicolor,
		center,
		width=#1,
		%enhanced,
		adjusted title=#4,
		fonttitle=\large\sffamily\bfseries,
		toptitle=2pt,
		bottomtitle=2pt,
		coltitle=FernUni-MI-green!80!black,
		colback=white,%gray!5,
		colframe=FernUni-MI-green!30,
		colbacklower=FernUni-MI-green!10,
		boxrule=1pt,
		rounded corners,
		segmentation style={solid,line width=1pt}, % Trennlinie 
		% drop fuzzy shadow, % Schatten
		%
		% mit folgenden Zeilen kann man Karopapier in Hintergrund anzeigen
		%underlay={
		%    \begin{tcbclipinterior}
		%        \draw[help lines,step=4mm,FernUni-MI-green!10,shift={(interior.north west)}]
		%        (interior.south west) grid (interior.north east);
		%    \end{tcbclipinterior}
		%},
		scale=#2,
		rotate=#3,
		]
		% -------------------------------------------------
		\textsf{\textbf{#5}}\\
		\schreibschrift{\textbf{#6}}
		% -------------------------------------------------
		\ifthenelse{\equal{#8}{\empty}}
		{
			% Text zum 2. Teil leer -> keine Anzeige
		}
		{
			\tcbline
			% -------------------------------------------------
			\textsf{\textbf{#7}}\\
			\schreibschrift{\textbf{#8}}
		}
		% -------------------------------------------------
		% Wenn der Parameter für den Ersteller-Namen leer ist, fällt die Anzeige dieser zeile weg.
		\ifthenelse{\equal{#9}{\empty}}
		{
			% Ersteller-Name leer -> keine Anzeige
		}
		{
			% Ersteller-Name nicht leer -> Anzeige mittels tcblower
			\tcblower
			% -------------------------------------------------
			\small \textsf{\textbf{Ersteller:}} #9
			% -------------------------------------------------
		}
	\end{tcolorbox}		
	\smallskip
}

%--------------------------------------
% neues Kommando für einen Universal-Kasten
% In einem Universal-Kasten wird eine Überschrift (mittig in der Kopfzeile) und 
% ein Text (normal im Kasten) angezeigt. Der Kasten ist so breit wie der Text.
% Das Kommando enthält 2 Parameter
%   - Überschrift, wird fett und zentriert in einer farblich unterlegten Box dargestellt
%   - Text, wird normal dargestellt
% Ergänzung: Wenn der erste Parameter leer bleibt (keine Überschrift),
%            dann wird die Titelzeile im Kasten gar nicht angezeigt.
%--------------------------------------
\newcommand{\sttpUniversalkasten}[2]{
	\ifthenelse{\equal{#1}{\empty}}
	{
		% Ausgabe ohne Titelzeile, wenn #1 leer ist
		\begin{tcolorbox}[
			center,
			width=\textwidth,
			enhanced,
			%adjusted title=#1,
			%halign title=center,
			% Das kleine "i"-Symbol in der Titelzeile des Kastens entfernen.
			%after title={\hfill\includegraphics[height=0.3cm]{Bilder/elephant_emoji.png}},
			%fonttitle=\sffamily\bfseries,
			%coltitle=FernUni-MI-green!80!black,
			enlarge top initially by=0mm,
			enlarge bottom finally by=2mm,
			left=2mm,right=2mm,middle=0.5mm,
			segmentation hidden,
			sharp corners,
			boxrule=2.4pt,
			colback=gray!5,
			colframe=FernUni-MI-green!30,
			drop fuzzy shadow,
		]
		#2
		\end{tcolorbox}
	}
	{
		\vspace{2ex}
		% Ausgabe mit Titelzeile, wenn #1 NICHT leer ist
		\begin{tcolorbox}[
			center,
			width=\textwidth,
			enhanced,
			adjusted title=#1,
			halign title=center,
			% Das kleine "i"-Symbol in der Titelzeile des Kastens entfernen.
			%after title={\hfill\includegraphics[height=0.3cm]{Bilder/elephant_emoji.png}},
			fonttitle=\sffamily\bfseries,
			coltitle=FernUni-MI-green!80!black,
			enlarge top initially by=0mm,
			enlarge bottom finally by=2mm,
			left=2mm,right=2mm,middle=0.5mm,
			segmentation hidden,
			sharp corners,
			boxrule=2.4pt,
			colback=gray!5,
			colframe=FernUni-MI-green!30,
			drop fuzzy shadow,
		]
		#2
		\end{tcolorbox}
		\vspace{2ex}
	}
}


%--------------------------------------
% Die folgenden Einstellungen sind für die "algorithm"-Umgebung,
% die in Kurseinheit 7 / Kapitel 11 verwendet wird.
% z.B. Farben in der algorithm-Umgebung
%--------------------------------------
% Text-Style in Conditions (If, While, Repeat, ...)
% Wird gebraucht für "if odd(n)". Ansonsten ist "odd" immer kursiv.
% Quelle: https://tex.stackexchange.com/questions/301793/in-algorithm2e-how-to-force-non-italic-font-in-the-condition-block-of-while
\SetArgSty{textnormal}
%--------------------------------------
% Einzelne Zeilen können farblich hervorgehoben werden.
% Quelle: https://tex.stackexchange.com/questions/149779/how-can-i-colourfuly-highlight-some-lines-of-an-algorithm-using-algorithm2e
\def\HiLi{\leavevmode\rlap{\hbox to \hsize{\color{FernUni-MI-green!5}\leaders\hrule height .8\baselineskip depth .5ex\hfill}}}
%--------------------------------------
% Titelzeile wird farblich unterlegt.
% Quelle: https://tex.stackexchange.com/questions/176057/algorithm2e-and-color
\makeatletter
\renewcommand{\algocf@makecaption@ruled}[2]{%
	\global\sbox\algocf@capbox{\colorbox{FernUni-MI-green!20}{\hskip\AlCapHSkip% .5\algomargin%
			\parbox[t]{\hsize}{\algocf@captiontext{\strut#1}{\strut#2\strut}}\hskip 1.5\algomargin}}% then caption is not centered
}%
\setlength{\interspacetitleruled}{0pt}%
\makeatother
%--------------------------------------
% "ggt" wird in Formeln damit nicht kursiv dargestellt
\newcommand{\ggt}{{\text{ggT}}}
%--------------------------------------

%%-----------------------------------------------------
%%---------- Makros für Lektion 3 (Petrinetze) --------
%%-----------------------------------------------------

% wird für die Darstellung der Petrinetze in Lektion 3 benötigt
\usetikzlibrary{petri, positioning, angles, quotes, decorations.pathreplacing, calc, arrows.meta}

% zentrales Setzen von Farben
\colorlet{colDummyLine}{white} % Dummy-Linie über diesen Schalter sichtbar (rot) bzw. unsichtbar (weiß) machen

\colorlet{COLTRANS}{black}
\definecolor{COLPLACE}{RGB}{0,102,102} % FernUni-Grün
\colorlet{COLRED}{red!80!black}

\newlength{\PNTransWidth}
\newlength{\PNTransHeight}
\newlength{\PNPlaceSize}

\setlength{\PNTransWidth}{2.2cm}
\setlength{\PNTransHeight}{1.5cm}
\setlength{\PNPlaceSize}{0.9cm}

%--------------------------------------
% Makro für Petrinetze
%--------------------------------------
% Das folgende Kommando verwenden wir für die Darstellung der Petrinetze in Lektion 3.
% Voraussetzung hierfür ist die Verwendung des Package "tikz" und einiger Libraries.
%
% \usepackage{tikz}
% \usetikzlibrary{petri, positioning, angles, quotes, decorations.pathreplacing, calc, arrows.meta}
%--------------------------------------
\newcommand{\petrinetz}[1]{
	\begin{tikzpicture}[
		>=Stealth, % Latex-Pfeilspitzen verwenden
		line width=0.9pt, % Dicke der Linien
		every place/.style={minimum size=0.75cm, inner sep=0pt}, % Größe der Stellen, ohne Polsterung
		every transition/.style={minimum size=0.75cm, inner sep=0pt}, % Größe der Transitionen, ohne Polsterung
		every node/.append style={font=\boldmath}, % Fett und größer für alle Labels
		shorten >=0pt,
		shorten <=0pt, % Pfeile verkürzen
		post/.style={->, line width=0.9pt, arrows={-Stealth[length=8pt,width=6pt]}} % Pfeilspitzen größer
		]
		#1 % Zeichnung
	\end{tikzpicture}
}


%%%%%%%%%%%%%%%%%%%%%%%%%%%%%%%%%%%%%%%%%%%%%%%%%%%
%%%%% temporäre Pakete und Einstellungen (nur für Entwicklungszeit)
%%%%%%%%%%%%%%%%%%%%%%%%%%%%%%%%%%%%%%%%%%%%%%%%%%%

%% In dieser Datei sammeln wir die Pakete, die wir nur für die Erstellung des Kurses während der Entwicklungszeit benötigen.
% Die Idee ist, dieses Kapitel dann vor der Erzeugung der End-Version umzubenennen,
% um alle Einstellungen, die nur zur Entwicklungszeit benötigt wurden, sicher auszuschließen.

% Auf der Titelseite wird die aktuelle Uhrzeit (wann PDF erzeugt wurde) ausgegeben.
\usepackage{scrtime}

% für 'Blindtext'
\usepackage{lipsum}
% Die Datei muss aus der Vorlage kopiert und entsprechend angepasst werden.
%\input{tmp_Inhalt/tmp_Meine_Einstellungen.tex}


%%%%%%%%%%%%%%%%%%%%%%%%%%%%%%%%%%%%%%%%%%%%%%%%%%%
%%%%% für die Titelseite
%%%%%%%%%%%%%%%%%%%%%%%%%%%%%%%%%%%%%%%%%%%%%%%%%%%

%%%%%%%%%%%%%%%%%%%%%%%%%%%%%%%%%%%%%%%%%%%%%%%%%%%
%%%%% für die Titelseite
%%%%%%%%%%%%%%%%%%%%%%%%%%%%%%%%%%%%%%%%%%%%%%%%%%%

%\title{Kurs 1793}
%\author{Autoren: Jörg Desel, Maren Stephan}
%\date{übersetzt von \MeinName, \today{}, \thistime~Uhr}

\newcommand\Kursautor{Prof. Dr. Jörg Desel, Maren Stephan}
\newcommand\Modulnummer{63812}
\newcommand\Modulname{Software Engineering}
\newcommand\Lehrgebiet{Softwaretechnik und Theorie der Programmierung}
%\newcommand\Zeitstempel{übersetzt von \MeinName, \today{}, \thistime~Uhr}

% Fonts, die gewoehnlich auf Titelseiten vorkommen
% (übernommen von C. Icking / ProPra SS 2022)
\newcommand\feuletbold{\fontfamily{lfrs}\fontseries{b}\fontshape{n}}
\newcommand\feuletlight{\fontfamily{lfrs}\fontseries{l}\fontshape{n}}
\newcommand\fruttitel{\feuletbold\fontsize{24}{15pt}\selectfont}
\newcommand\frutsubtitel{\feuletbold\fontsize{12}{14pt}\selectfont}
\newcommand\frutsubsubtitel{\feuletbold\fontsize{10}{11pt}\selectfont}
\newcommand\frutfakultaet{\feuletlight\fontsize{18}{12pt}\selectfont}
\newcommand\frutfakultaetb{\feuletbold\fontsize{18}{12pt}\selectfont}
\newcommand\fruturheberrecht{\feuletlight\fontsize{9}{10pt}\selectfont}
\let\frutautor\frutsubtitel

\newcommand\frutzeitstempel{\feuletlight\fontsize{9}{10pt}\selectfont}

%--------------------------------------
% Für das Deckblatt gibt es extra ein neues Kommando / Makro,
% damit es sowohl für die erste Seite einer Lektion (Parts)
% als auch für die Titelseite genutzt werden kann.
% Der Befehl hat 5 Parameter
%   [1] Autor(en)
%   [2] Modulnummer
%   [3] Modulname
%   [4] Lektion
%   [5] Titel der Lektion
%   [6] Hinweis (wird zur Zeit nicht verwendet)
%--------------------------------------
\newcommand\sttpDeckblatt[6]{
\begin{tikzpicture}[remember picture,overlay]
	% Zeichnen der farbigen Hintergründe
	% hellgrüner großer Bereich
	\fill[FernUni-MI-green-light] (current page.north west) rectangle ([xshift=-3cm, yshift=-19cm] current page.north east);
	% die dunkelgrüne Form setzt sich aus 2 Rechtecken und einem Kreis zusammen
	\fill[FernUni-MI-green] ([xshift=2cm, yshift=-19cm] current page.north west) rectangle ([xshift=9cm, yshift=-14cm] current page.north west);
	\fill[FernUni-MI-green] ([xshift=2cm, yshift=-21cm] current page.north west) rectangle ([xshift=7cm, yshift=-19cm] current page.north west);
	\fill[FernUni-MI-green] ([xshift=7cm, yshift=-19cm] current page.north west) circle (2cm);
	%
	% jetzt kommt der Text ...
	% Autoren
	\draw[black] node[anchor=west, xshift=2cm, yshift=-6.5cm] at (current page.north west) {\frutautor\selectfont {#1}};
	% Modulnummer
	\draw[black] node[anchor=west, xshift=2cm, yshift=-8.2cm] at (current page.north west) {\fruttitel\selectfont {Modul #2}};
	% Modulname
	\draw[black] node[anchor=west, xshift=2cm, yshift=-9.5cm] at (current page.north west) {\fruttitel\selectfont {#3}};
	% Lektion X
	\draw[black] node[anchor=west, xshift=2cm, yshift=-11cm] at (current page.north west) {\frutsubtitel\selectfont {#4}};%{\partname~\thepart}};
	% Untertitel
	\draw[black] node[anchor=west, xshift=2cm, yshift=-11.6cm] at (current page.north west) {\frutsubtitel\selectfont {#5}};
	%
	% weißer Schriftzug: Fakultät M+I
	\draw[white] node[anchor=west, xshift=2.7cm, yshift=-15.2cm] at (current page.north west) {\frutfakultaet\selectfont Fakultät für};
	\draw[white] node[anchor=west, xshift=2.7cm, yshift=-16.1cm] at (current page.north west) {\frutfakultaetb\selectfont Mathematik und};
	\draw[white] node[anchor=west, xshift=2.7cm, yshift=-17.0cm] at (current page.north west) {\frutfakultaetb\selectfont Informatik};
	%
	% FernUni-Logo
	\node[black] [anchor=east, xshift=18cm, yshift=-27.0cm] at (current page.north west) {\includegraphics[width=65mm]{Bilder/feulogo.pdf}};
	%
	% Zeitstempel
	%\draw[black] node[anchor=east, fill=white, xshift=20cm, yshift=-1.0cm] at (current page.north west) {\frutzeitstempel\selectfont {#5}};
\end{tikzpicture}%
}


%--------------------------------------
% Für die Rückseite des Deckblatts gibt es ebenfalls ein neues Kommando / Makro,
% damit es sowohl für die erste Seite einer Lektion (Parts)
% als auch für die Titelseite genutzt werden kann.
% Der Befehl hat 2 Parameter
%   [1] Lektion
%   [2] Text zu dieser Lektion
%--------------------------------------
\newcommand\sttpDeckblattRueckseite[2]{
	\newgeometry{left=25mm, right=15mm, top=80mm, bottom=15mm}
	\thispagestyle{empty}
	{
		\sffamily % Text der Rückseite ohne Serifen
		
		Desel/Stephan: \Modulname\\
		#1, Hagen 2024
		
		\vspace{1cm}
		#2
		
		\vfill
		Layout und Satz: Andrea Frank
		
		\vspace{3cm}
		
		\begin{minipage}{1.1cm}
			\includegraphics[width=1.0cm]{Bilder/elephant_emoji2.png}
		\end{minipage}
		-- Bildquelle: Sofie Ascherl, \href{https://creativecommons.org/licenses/by-sa/4.0}{CC BY-SA 4.0}, via OpenMoji
		
		\noindent\rule[1ex]{\textwidth}{1pt}
		{
			\scriptsize
			Das Werk ist urheberrechtlich geschützt. Die dadurch begründeten Rechte, insbesondere das Recht der Vervielfältigung und Verbreitung sowie der Übersetzung und des Nachdrucks, bleiben, auch bei nur auszugsweiser Verwertung, vorbehalten. Kein Teil des Werkes darf in irgendeiner Form (Druck, Fotokopie, Mikrofilm oder ein anderes Verfahren) ohne schriftliche Genehmigung der FernUniversität reproduziert oder unter Verwendung elektronischer Systeme verarbeitet, vervielfältigt oder verbreitet werden. Wir weisen darauf hin, dass die vorgenannten Verwertungsalternativen je nach Ausgestaltung der Nutzungsbedingungen bereits durch Einstellen in Cloud-Systeme verwirklicht sein können. Die FernUniversität bedient sich im Falle der Kenntnis von Urheberrechtsverletzungen sowohl zivil- als auch strafrechtlicher Instrumente, um ihre Rechte geltend zu machen.
			
			Der Inhalt dieses Studienbriefs wird gedruckt auf Recyclingpapier (80 g/m$\mathsf{^2}$, weiß), hergestellt aus 100\,\% Altpapier.
		}
	}

	\restoregeometry
}
